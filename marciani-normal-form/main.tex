% Marciani Normal Form, On semidecidability of context-free language regularity
% Authors: Giacomo Marciani
%
% Style: acmsmall.cls v 1.3

\documentclass[prodmode,acmtecs]{acmsmall}

\usepackage[ruled]{algorithm2e}
\renewcommand{\algorithmcfname}{ALGORITHM}
\SetAlFnt{\small}
\SetAlCapFnt{\small}
\SetAlCapNameFnt{\small}
\SetAlCapHSkip{0pt}
\IncMargin{-\parindent}

% METADATA INFORMATION
\acmVolume{9}
\acmNumber{4}
\acmArticle{39}
\acmYear{2016}
\acmMonth{3}

% COPYRIGHT
%\setcopyright{acmcopyright}
%\setcopyright{acmlicensed}
%\setcopyright{rightsretained}
%\setcopyright{usgov}
%\setcopyright{usgovmixed}
%\setcopyright{cagov}
%\setcopyright{cagovmixed}

% DOI
\doi{0000001.0000001}

%ISSN
\issn{1234-56789}

\begin{document}

%PAGE HEADS
\markboth{G. Marciani et al.}{Marciani Normal Form}

%TITLE
\title{Marciani Normal Form}
\author{Giacomo Marciani
\affil{University of Rome Tor Vergata}
Alberto Pettorossi
\affil{University of Rome Tor Vergata}
}

% ABSTRACT
\begin{abstract}
We prove the semidecidability of the problem of saying wheter or not a context-free
grammar generates a regular language.
We introduce the notion of context-free grammars in Marciani Normal Form.
We prove that a context-free grammar in Marciani Normal Form always generates a
regular language.
\end{abstract}

% CATEGORIES
\begin{CCSXML}
	<ccs2012>
	<concept>
	<concept_id>10003752.10003766.10003771</concept_id>
	<concept_desc>Theory of computation~Grammars and context-free languages</concept_desc>
	<concept_significance>500</concept_significance>
	</concept>
	</ccs2012>
\end{CCSXML}

\ccsdesc[500]{Theory of computation~Grammars and context-free languages}

\acmformat{Giacomo Marciani and Alberto Pettorossi, 2016. Marciani Normal Form.}

\begin{bottomstuff}
This work is supported by the National Science Foundation, under
grant CNS-0435060, grant CCR-0325197 and grant EN-CS-0329609.

Author's addresses: G. Marciani {and} A. Pettorossi, Computer Science Department, University of Rome Tor Vergata.
\end{bottomstuff}

\maketitle

\section{Introduction}

In Section 2, we give some preliminary definitions about language
equations and regularity problem. In particular, we introduce the
class of both-linear equations, the pseudo-regular partition and the
looking forward property.

In Section 3, we state and prove the Marciani's Rule, that exposes
a method to solve the class of both-linear language equations.

In Section 4, we introduce the notion of Marciani Normal Form for
context-free grammars, and we prove that a context-free grammar in
Marciani Normal Form always generates a regular language.

In Section 5, we prove the semidecidability and the undecidability
of the context-free regularity problem.

In Section 6, we give some example of the Marciani's Rule and the
Marciani Normal Form.

\section{Definitions}
\begin{definition}[Left-Linear Equations]
	A left-linear language equation in
	the unknown $r$ is a language equation of the form
	\begin{equation}
	r=ar+s
	\end{equation}
\end{definition}

\begin{definition}[Right-Linear Equations]
	A right-linear language equation
	in the unknown $r$ is a language equation of the form
	\begin{equation}
	r=ra+s
	\end{equation}
\end{definition}

\begin{definition}[Both-Linear Equations]
	A both-linear language equation in
	the unknown $r$ is a language equation of the form
	\begin{equation}
	r=ar+rb+s
	\end{equation}
\end{definition}

\begin{definition}[Looking Forward Property]
	A grammar $G=(V_{T},V_{N},S,P)$
	enjoys the looking forward property iff the digraph $D_{G}$ constructed
	as follows does not have any cicle.

	Let $D_{G}$ be a digraph such that for all nonterminal symbol in
	$V_{T}$ there exists a node in $D_{G}$, and for all production $A\rightarrow\alpha$
	in $P$ there exists an arc from $A$ to every nonterminal symbol
	in $\alpha$, with the exception of $A$ itself.
\end{definition}

\begin{definition}[Pseudo-Regular Partition]
	Given a grammar $G=(V_{T},V_{N},S,P)$,
	a symbol $A\in V_{N}$ and the subset $P_{A}\subseteq P$ of productions
	for the nonterminal $A$, the pseudo-regular partition induced by
	$A$ is the partition $\Upsilon_{A}$ of $P_{A}$ defined as follows:
	\begin{equation}
	\Upsilon_{A}=\{\{p|p\in P_{A},A\rightarrow\alpha A\},\{p|p\in P_{A},A\rightarrow A\beta\},\{p|p\in P_{A},A\rightarrow\gamma\}\}
	\end{equation}
	where $\alpha,\beta,\gamma\in\left(V_{T}\cup V_{N}\setminus\{A\}\right)^{*}$.
\end{definition}

\begin{definition}[CF Regularity Problem]
	The CF Regularity Problem is the
	subset $\Re$ of $\{0,1\}^{*}$ defined as follows:
	\begin{equation}
	\Re=\{\underline{G}|G\in[G_{CS}]\wedge L(G)\in[L_{REG}]\}
	\end{equation}
	where $\underline{G}$ is the encoding of a context-free grammar $G$
	as a string in $\{0,1\}^{*}$ \cite{Pettorossi13}, $[G_{CS}]$ is the class
	of context-free grammars, and $[L_{REG}]$ is the class of regular
	languages.
\end{definition}

\section{Marciani's Rule}
It is well known that the Arden's Rule permits the resolution of left-linear
and right-linear language equations \cite{Pettorossi13}.
Now, we state and prove a theorem that permits the resolution of both-linear
language equations.

\begin{theorem}[Marciani's Rule]
	Given the language equation $r=ar+rb+s$
	in the unknown $r$, its least solution is $a^{*}sb^{*}$.
\end{theorem}

\begin{proof}
	Let us divide the proof in the following three point (a), (b), (c).
	\\
	(a) We will first show that the language equation $r=ar+rb+s$ is
	equivalent to the language equation $r=a^{*}(rb+s)$. \\
	(b) Then, we will show that $a^{*}sb^{*}$ is a solution for $r$
	of the language equation $r=a^{*}(rb+s)$. \\
	(c) Finally, we will show that $a^{*}sb^{*}$ is the minimal solution
	for $r$ of the language equation $r=a^{*}(rb+s)$, that is, for
	any other solution $z$ we have that $L(a^{*}sb^{*})\subseteq L(z)$.\\

	Proof (a): By the application of the Arden Rule on $r=ar+rb+s$ ,
	we obtain $r=a^{*}(rb+s)$. So $ar+rb+s=a^{*}(rb+s)$\cite{Pettorossi13}.\\

	Proof (b): Notice that $a^{*}((a^{*}sb^{*})b+s)=a^{*}a^{*}sb^{*}b+a^{*}s=a^{*}sb^{*}b+a^{*}s$,
	so we have to show that (b.1) $a^{*}sb^{*}\subseteq a^{*}sb^{*}b+a^{*}s$
	and (b.2) $\beta.2$ $a^{*}sb^{*}b+a^{*}s\subseteq a^{*}sb^{*}$.\\

	Proof (b.1): The following inclusion holds $a^{*}sb^{*}=a^{*}s(b^{+}+\varepsilon)\subseteq a^{*}s(b^{*}b+\varepsilon)=a^{*}sb^{*}b+a^{*}s$.\\

	Proof (b.2): The following inclusions holds $a^{*}sb^{*}b\subseteq a^{*}sb^{*}$
	and $a^{*}s\subseteq a^{*}sb^{*}$.\\

	Proof (c): We assume that $z$ is a solution of $r=a^{*}(rb+s)$,
	that is $z=a^{*}(zb+s)$, and we show that $a^{*}sb^{*}\subseteq z$,
	that is $\bigcup_{i,j\geq0}a^{i}sb^{j}\subseteq z$. The proof can
	be done by induction on $i,j\geq0$.\\
	\textsl{(Basis $i,j=0$):} $s\subseteq z$ holds because
	$z=a^{*}(zb+s)$.\\
	\textsl{(Step: $i\geq0,j=0$):} We have to show the following implication
	$\bigcup_{i\geq0}a^{i}s\subseteq z\rightarrow\bigcup_{i\geq0}a^{i+1}s\subseteq z$.
	This holds because $\bigcup_{i\geq0}a^{i+1}s\subseteq a\bigcup_{i\geq0}a^{i}s\subseteq az\subseteq z$.
	\\
	\textsl{(Step: $i,j\geq0$):} We have to show the following implication
	$\bigcup_{i,j\geq0}a^{i}sb^{j}\subseteq z\rightarrow\bigcup_{i,j\geq0}a^{i}sb^{j+1}\subseteq z$.
	This holds because $\bigcup_{i,j\geq0}a^{i}sb^{j+1}\subseteq\bigcup_{i,j\geq0}a^{i}sb^{j}b\subseteq zb\subseteq z$.
\end{proof}

\section{Marciani Normal Form}
We introduce the notion of context-free grammars in Marciani Normal
Form. We prove that every grammar in Marciani Normal Form generates
a regular language.

\begin{definition}[Grammar in MNF]
	A context-free grammar $G=(V,V_{N},S,P)$
	is said to be in Marciani Normal Form iff $G$ enjoys the looking
	forward property and for all symbol $A\in V_{T}$ there exists a pseudo-regular
	partition $\Upsilon_{A}$.
\end{definition}

\begin{theorem}[Regularity of context-free grammar in MNF]
	A context-free grammar $G=(V_{T},V_{N},S,P)$ in Marciani Normal Form always generates
	a regular language.
\end{theorem}

\begin{proof}
	If $G$ is in Marciani Normal Form, then for each nonterminal $A$
	there exists a pseudo-regular partition $\Upsilon_{A}$. Notice that
	to every pseudo-regular partition $\Upsilon_{A}=\{\{p|p\in P_{A},A\rightarrow\alpha A\},\{p|p\in P_{A},A\rightarrow A\beta\},\{p|p\in P_{A},A\rightarrow\gamma\}\}$
	can be associated a both-linear language equation $A=\alpha A+A\beta+\gamma$.
	By the application of the Marciani's Rule, we know that the least
	solution of the previous equation is $\alpha^{*}\gamma\beta^{*}$,
	so $L(A)=L(\alpha^{*}\gamma\beta^{*})$.

	As a conseguece of the previous results and the looking forward property,
	there exists a regular expression $e$ such that $L(S)=L(e)$, then
	$L(S)$ is a regular language, that is $G$ is regular.
\end{proof}


\section{Semidecidability of the CF Regularity Problem}
We know that the context-free regularity problem is undecidable \cite{Pettorossi13},
due to its reduction to the undecidable Post Correspondence Problem
\cite{Hopcroft06}.

We prove the semidecidability and the undecidability of the context-free
regularity problem.

\begin{theorem}
	Given a context-free grammar, the problem of saying wheter or not
	there exists an equivalent grammar in Marciani Normal Form is semidecidable
	and undecidable.
\end{theorem}

\begin{proof}
	Let us consider a context-free grammar in Chomsky Normal Form. Let
	us derive an equivalent grammar by unfolding every production, until
	getting the productions for the axiom only. Now, it' easy to check
	if the derived grammar is in Marciani Normal Form. So, given any context-free
	grammar in Chomsky Normal Form, it is always possible to check if
	there exists an equivalent context-free grammar in Marciani Normal
	Form.

	As this possibility holds for grammars in Chomsky Normal Form, then
	it holds for every context-free grammar \cite{Pettorossi13}.
\end{proof}

\begin{theorem}[Semidecidability of the CF Regularity Problem]
	The context-free regularity problem is semidecidable and undecidable.
\end{theorem}

\begin{proof}
	Follows from the semidecidability and undecidability of the problem
	of saying wheter or not, given a context-free grammar, there exists
	an equivalent grammar in Marciani Normal Form, and from the regularity
	of the language generated by a context-free grammar in Marciani Normal
	Form.
\end{proof}

\section{Conclusions}
Lorem ipsum dolor sit amet, consectetur adipiscing elit, sed do eiusmod tempor incididunt ut labore et dolore magna aliqua.
Ut enim ad minim veniam, quis nostrud exercitation ullamco laboris nisi ut aliquip ex ea commodo consequat.
Duis aute irure dolor in reprehenderit in voluptate velit esse cillum dolore eu fugiat nulla pariatur.
Excepteur sint occaecat cupidatat non proident, sunt in culpa qui officia deserunt mollit anim id est laborum.

\appendix
\section*{APPENDIX}
\setcounter{section}{1}

\begin{example}
	Let us consider the context-free grammar $G$ with axiom $S$ and
	the following productions

	\[
	S\rightarrow abcS|Sdef|ghi|\varepsilon
	\]


	The language generated by $G$ is denoted by the regular expression
	\[
	(abc)^{*}(ghi+\varepsilon)(def)^{*}
	\]

\end{example}

\begin{example}
	Let us consider the context-free grammar $G$ with axiom $S$ and
	the following productions

	\[
	S\rightarrow aAS|SBdef|CD|\varepsilon
	\]
	\[
	A\rightarrow uA|Av|m
	\]
	\[
	B\rightarrow xB|By|n
	\]
	\[
	C\rightarrow gC|Ch|i
	\]
	\[
	D\rightarrow pD|Dq|r
	\]


	The language generated by $G$ is denoted by the regular expression
	\[
	(au^{*}mv^{*})^{*}(g^{*}ih^{*}p^{*}rq^{*}+\varepsilon)(x^{*}ny^{*}def)^{*}
	\]
\end{example}

\begin{example}
	Let us consider the context-free grammar $G$ in \textsl{Chomsky Normal
		Form} with axiom $S$ and the following productions

	\[
	S\rightarrow AS|SB|CD|z|\varepsilon
	\]
	\[
	A\rightarrow UA|AU|m
	\]
	\[
	B\rightarrow XB|BX|n
	\]
	\[
	C\rightarrow UC|CU|i
	\]
	\[
	D\rightarrow XD|DX|r
	\]


	\[
	U\rightarrow u
	\]
	\[
	X\rightarrow x
	\]

	The language generated by $G$ is denoted by the regular expression
	\[
	(u^{*}mu^{*})^{*}(u^{*}iu^{*}x^{*}rx^{*}+z+\varepsilon)(x^{*}nx^{*})^{*}
	\]
\end{example}

\appendixhead{ZHOU}

\begin{acks}
Lorem ipsum dolor sit amet, consectetur adipiscing elit, sed do eiusmod tempor incididunt ut labore et dolore magna aliqua.
Ut enim ad minim veniam, quis nostrud exercitation ullamco laboris nisi ut aliquip ex ea commodo consequat.
Duis aute irure dolor in reprehenderit in voluptate velit esse cillum dolore eu fugiat nulla pariatur.
Excepteur sint occaecat cupidatat non proident, sunt in culpa qui officia deserunt mollit anim id est laborum.
\end{acks}

% Bibliography
\bibliographystyle{ACM-Reference-Format-Journals}
\bibliography{references}

% History dates
\received{February 2007}{March 2009}{June 2009}

% Electronic Appendix
\elecappendix

\medskip

\section{This is an example of Appendix section head}

Lorem ipsum dolor sit amet, consectetur adipiscing elit, sed do eiusmod tempor incididunt ut labore et dolore magna aliqua.
Ut enim ad minim veniam, quis nostrud exercitation ullamco laboris nisi ut aliquip ex ea commodo consequat.
Duis aute irure dolor in reprehenderit in voluptate velit esse cillum dolore eu fugiat nulla pariatur.
Excepteur sint occaecat cupidatat non proident, sunt in culpa qui officia deserunt mollit anim id est laborum.

\section{Appendix section head}

Lorem ipsum dolor sit amet, consectetur adipiscing elit, sed do eiusmod tempor incididunt ut labore et dolore magna aliqua.
Ut enim ad minim veniam, quis nostrud exercitation ullamco laboris nisi ut aliquip ex ea commodo consequat.
Duis aute irure dolor in reprehenderit in voluptate velit esse cillum dolore eu fugiat nulla pariatur.
Excepteur sint occaecat cupidatat non proident, sunt in culpa qui officia deserunt mollit anim id est laborum.

\end{document}
