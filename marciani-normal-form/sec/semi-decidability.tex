\section{Semi-decidability of the CF Regularity Problem}
\label{sec:semi-decidability}

We know that the context-free regularity problem is undecidable \cite{Pettorossi13}, due to its reduction to the undecidable Post Correspondence Problem \cite{Hopcroft06}.

We prove the semi-decidability and the undecidability of the context-free regularity problem.

\begin{theorem}
	\label{thm:semi-decidability-mnf}	
	Given a context-free grammar, the problem of saying wheter or not there exists an equivalent grammar in Marciani Normal Form is semidecidable and undecidable.
	
	\begin{proof}
		Let us consider a context-free grammar in Chomsky Normal Form. Let us derive an equivalent grammar by unfolding every production, until getting the productions for the axiom only. Now, it' easy to check if the derived grammar is in Marciani Normal Form. So, given any context-free grammar in Chomsky Normal Form, it is always possible to check if there exists an equivalent context-free grammar in Marciani Normal Form.
		
		As this possibility holds for grammars in Chomsky Normal Form, then it holds for every context-free grammar \cite{Pettorossi13}.
	\end{proof}
\end{theorem}

\begin{theorem}
	\label{thm:semi-decidability}
	The context-free regularity problem is semidecidable and undecidable.
	
	\begin{proof}
		Follows from the semidecidability and undecidability of the problem of saying wheter or not, given a context-free grammar, there exists an equivalent grammar in Marciani Normal Form, and from the regularity of the language generated by a context-free grammar in Marciani Normal Form.
	\end{proof}
\end{theorem}