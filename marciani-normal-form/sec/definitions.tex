\section{Definitions}
\label{sec:definitions}

We recall definitions known in the literature, for convenience of the reader.

It is well known that a language equation can be classified, at first instance,
according to the position of its unknown.
In  particular, we tipically have the following definitions.

\begin{definition}
	\label{dfn:ll-language-equation}
	A left-linear (ll) language equation in the unknown $r$ is a language
	equation of the form

	\begin{equation}
	\label{eqn:ll-language}
	r=ar+s
	\end{equation}
\end{definition}

\begin{definition}
	\label{dfn:rl-language-equation}
	A right-linear (rl) language equation in the unknown $r$ is a language
	equation of the form

	\begin{equation}
	\label{eqn:rl-language}
	r=ra+s
	\end{equation}
\end{definition}

A grammar in wich all productions for a given non-terminal identifiy ll or rl
language equations, always generates a regular language.

The problem of saying whether or not a context-free grammar $G$ generates a
regular language is very important.
In fact, a context-free grammar has greater expressiveness than a regular one.
On the other hand, a regular grammar is much lighter in terms of computational
complexity.
We call this problem \textit{Context-Free (CF) Regularity Problem}.

\begin{definition}
	\label{dfn:cf-regularity-problem}
	The \textit{Context-Free (CF) Regularity Problem} is the subset
	$\Re$ of $\{0,1\}^{*}$ defined as follows:

	\begin{equation}
	\label{eqn:cf-regularity-problem}
	\Re=\{\underline{G}|G\in[G_{CS}]\wedge L(G)\in[L_{REG}]\}
	\end{equation}

	where $\underline{G}$ is the encoding of a context-free grammar $G$ as a
	string in $\{0,1\}^{*}$, $[G_{CS}]$ is the class of context-free grammars,
	and $[L_{REG}]$ is the class of regular languages.
\end{definition}

We now introduce the new definitions, which will be essential in subsequent
sections.

\begin{definition}
	\label{dfn:bl-language-equation}
	A bilateral-linear language equation in the unknown $r$ is a language
	equation of the form

	\begin{equation}
	\label{eqn:ll-rl-language}
	r=ar+rb+s
	\end{equation}
\end{definition}

\begin{definition}
	\label{dfn:looking-forward-property}
	A grammar $G=(V_{T},V_{N},S,P)$ enjoys the looking forward property if and
	only if the digraph $D_{G}$, constructed as follows, does not have any cycle.

	Let $D_{G}$ be a digraph such that for all non-terminal symbol in $V_{T}$
	there exists a node in $D_{G}$, and for all production $A\rightarrow\alpha$
	in $P$ there exists an arc from $A$ to every non-terminal symbol in $\alpha$,
	with the exception of $A$ itself.
\end{definition}

\begin{definition}
	\label{dfn:pseudo-regular-partition}
	Given a grammar $G=(V_{T},V_{N},S,P)$, a symbol $A\in V_{N}$ and the subset
	$P_{A}\subseteq P$ of productions for the non-terminal $A$, the
	\textit{pseudo-regular partition} induced by $A$ is the partition
	$\Upsilon_{A}$ of $P_{A}$, defined as follows:

	\begin{equation}
	\label{eqn:pseudo-regular-partition}
	\Upsilon_{A}:=
	\biggl\{p\in P_{A} \;\bigg|\;
	\{A\rightarrow\alpha A\},
	\{A\rightarrow A\beta\},
	\{A\rightarrow\gamma\}
	\biggl\}
	\end{equation}

	where $\alpha,\beta,\gamma\in\left(V_{T}\cup V_{N}\setminus\{A\}\right)^{*}$.
\end{definition}
