\section{Marciani's Rule}
\label{sec:marciani-rule}

It is well known that the Arden's Rule permits the resolution of left-linear and right-linear language equations \cite{Pettorossi13}.
In particular, by the Arden's Rule, given the language equation $r=ar+s$ in the unknown $r$, its least solution is $a^{*}s$. Likewise, given the language equation $r=ra+s$ in the unknown $r$, its least solution is $sa^{*}$. 

Now, we state and prove a theorem that permits the resolution of both-linear language equations.

\begin{theorem}
	\label{thm:marciani-rule}
	
	Given the language equation $r=ar+rb+s$ in the unknown $r$, its least solution is $a^{*}sb^{*}$.
	
	\begin{proof}
		Let us divide the proof in the following three point (a), (b), (c).
		\\
		(a) We will first show that the language equation $r=ar+rb+s$ is equivalent to the language equation $r=a^{*}(rb+s)$. \\
		(b) Then, we will show that $a^{*}sb^{*}$ is a solution for $r$ of the language equation $r=a^{*}(rb+s)$. \\
		(c) Finally, we will show that $a^{*}sb^{*}$ is the minimal solution for $r$ of the language equation $r=a^{*}(rb+s)$, that is, for
		any other solution $z$ we have that $L(a^{*}sb^{*})\subseteq L(z)$.\\
		
		Proof (a): By the application of the Arden Rule on $r=ar+rb+s$, we obtain $r=a^{*}(rb+s)$. 
		So $ar+rb+s=a^{*}(rb+s)$\cite{Pettorossi13}.\\
		
		Proof (b): Notice that $a^{*}((a^{*}sb^{*})b+s)=a^{*}a^{*}sb^{*}b+a^{*}s=a^{*}sb^{*}b+a^{*}s$,
		so we have to show that 
		(b.1) $a^{*}sb^{*}\subseteq a^{*}sb^{*}b+a^{*}s$ and 
		(b.2) $\beta.2$ $a^{*}sb^{*}b+a^{*}s\subseteq a^{*}sb^{*}$.\\
		
		Proof (b.1): The following inclusion holds 
		$a^{*}sb^{*}=a^{*}s(b^{+}+\varepsilon)\subseteq  a^{*}s(b^{*}b+\varepsilon)=a^{*}sb^{*}b+a^{*}s$.\\
		
		Proof (b.2): The following inclusions holds 
		$a^{*}sb^{*}b\subseteq a^{*}sb^{*}$ and $a^{*}s\subseteq a^{*}sb^{*}$.\\
		
		Proof (c): We assume that $z$ is a solution of $r=a^{*}(rb+s)$, that is $z=a^{*}(zb+s)$, and we show that $a^{*}sb^{*}\subseteq z$,
		that is $\bigcup_{i,j\geq0}a^{i}sb^{j}\subseteq z$. The proof can be done by induction on $i,j\geq0$.\\
		\textsl{(Basis $i,j=0$):} $s\subseteq z$ holds because
		$z=a^{*}(zb+s)$.\\
		\textsl{(Step: $i\geq0,j=0$):} We have to show the following implication
		$\bigcup_{i\geq0}a^{i}s\subseteq z\rightarrow\bigcup_{i\geq0}a^{i+1}s\subseteq z$.
		This holds because $\bigcup_{i\geq0}a^{i+1}s\subseteq a\bigcup_{i\geq0}a^{i}s\subseteq az\subseteq z$.
		\\
		\textsl{(Step: $i,j\geq0$):} We have to show the following implication
		$\bigcup_{i,j\geq0}a^{i}sb^{j}\subseteq z\rightarrow\bigcup_{i,j\geq0}a^{i}sb^{j+1}\subseteq z$.
		This holds because $\bigcup_{i,j\geq0}a^{i}sb^{j+1}\subseteq\bigcup_{i,j\geq0}a^{i}sb^{j}b\subseteq zb\subseteq z$.
	\end{proof}
\end{theorem}