\chapter{Linguaggio naturale}
\label{chp:natural-language}

Un linguaggio formale è generato da una grammatica formale. 

Il linguaggio naturale si attiene alle regole grammticali, ma non è un linguaggio formale a causa dell'ambiguità semantica.

Il linguaggio naturale è flessibile, ambiguo e dinamico.

Il linguaggio formale è ripetitivo: ad uno stesso input, stesso output.

Il linguaggio naturale non è ripetitivo: ad uno stesso input, può corrispondere un output diverso.

Esistono linguaggi di conoscenza e linguaggi di ragionamento.

La comprensione del linguaggio naturale richiede la rappresentazione del significato oltre ai collegamenti alla conoscenza esplicita posseduta dal sistema.

Una macchina sarà considerata veramente intelligente solo quando sarà in grado di elaborare il linguaggio naturale come fa un umano. (test di Turing).

L'ingegneria del linguaggio naturale si occupa delle tecniche e applicazioni della modellazione del linguaggio naturale. 
Si occupa dello sviluppo di algoritmi per il parsing, la generazione, e la acquisizione di conoscenza linguistica.

Sono necessarie varie tipologie di conoscenza linguistica per elaborare un testo nella sua completezza oltre a conoscenza eventuale sul contesto (applicativo) (e.g.: l'interpretazione frasale presuppone l'indentificazione delle parole).

Il processamento del linguaggio naturale prevede le seguenti conoscenze linguistiche:

\begin{enumerate}

	\item morfologia: fornisce informazioni sulla composizione, flessione e coniugazione delle parole.
	
	\item sintassi: definisce le relazioni strutturali tra le parole, ovvero la correttezza di una sequenza di parole.
	
	\item semantica: definisce le regole di attribuzione del significato delle espressioni
	
	\begin{enumerate}
		
		\item semantica lessicale: significato di una singola parola;
		
		\item semantica composizionale: significato in funzione della combinazione di parole;
		
	\end{enumerate}
	
	\item pragmatica: definisce usi e convenzioni linguistiche al fine di valutare lo stile delle espressioni (e.g.: distinguere una espressione di cortesia, da una espressione di ostilità). Permette dunque di studiare come il linguaggio è usato per raggiungere un obiettivo.
	
	\item analisi del discorso: definisce le relazioni strutturali di entità linguistiche più ampie di una frase. Definisce la forma del discorso necessaria per poter valutare e formulare correttamente risposte a domande (e.g.: una domanda riguardo X prevede una risposta riguardo X, non una risposta riguardo Y).
	
\end{enumerate}

\subsection{Ambiguità}
L'elaborazione del linguaggio naturale prevede la presenza di tecniche di risoluzione di ambiguità interpretative. Un testo è ambiguo quando ad una stessa frase possono essere assegnati significati diversi. L’ambiguità è il motivo per cui le macchine non riescono ad elaborare completamente il linguaggio naturale.

Il significato di una frase non è dato solo dal significato delle parole. È necessario conoscere come il significato sia determinato dalla combinazione delle parole all'interno di una stessa frase, e dalle relazioni di queste parole con le frasi esterne.

L'ambiguità genera un'esplosione combinatoriale di pattern semantici, aumentando così la complessità computazionale dell'elaborazione.

L'ambiguità può essere:

\begin{enumerate}
	
	\item ambiguità lessicale: si risolve con l'analisi del contesto.
	
	\item ambiguità di anafora: si risolve conoscenza del dominio e senso comune.
	
	\item ambiguità categoriale: una parola può appartenere a più categorie. Questa ambiguità viene risolta mediante analisi sintattica.
	
	\item ambiguità strutturale: ambigua nell'analisi logica della frase.
	
	\item ambiguità referenziale: ambiguità dovuta ad anafore. Si risolve conoscenza del dominio e senso comune.
	
	\item ambiguità per ellissi: dovuta a mancanza di riferimenti nella frase. (e.g.: Giovanni aveva studiato molto e superò l’esame. Anche
	Mario).
	
	\item ambiguità proposizionale: in assenza di punteggiatura nella frase.
	
	\item ambiguità locale: una parte di una frase può avere più di una interpretazione. Questa ambiguità può essere risolta mediante analisi sintattica (e.g.: the old trains {the young|leave the station})
	
	\item ambiguità globale: una intera frase può avere più di una interpretazione. Questa ambiguità può essere risolta mediante analisi semantica e pragmatica. (e.g.: I saw the Gran Canyon flying to New York)
\end{enumerate}


\subsection{Conoscenza}
Gran parte del significato non risiede all’interno delle parole, per cui il sistema di NLP deve avere la conoscenza del senso comune, o un modello del mondo usato da chi ha scritto il testo.
Non è necessario capire tutto, ma capire ciò che serve per il dominio applicativo.


\subsection{Sintassi}
La sintassi definisce il ruolo delle parole all'interno di una frase, mediante classificazione Part-Of-Speech e relazioni interne ed esterne alla frase.

Vi sono almeno le seguenti classi POS: nome, verbo, avverbio, pronome, articolo, preposizione, congiunzione, participio.
Il tagging in classi POS permette di associare un significato alle parole, permettendo di riconoscere il ruolo delle parole vicine.

Una classe POS può essere (i) open class, se è un insieme in continua evoluzione, o (ii) closed class, se è un insieme fisso di parole.

La sintassi è definita da una grammatica formale, il cui insieme di produzioni viene rappresentato da una lista. Il processamento di una lista ha un costo computazionale minore di quello di un albero.

Nella grammatica formale di una sintassi vanno bene produzioni differenti per una frase, ma non va bene se esistono più simboli terminali per una stessa parola. Ciò creerebbe ambiguità.

Un testo processato secondo una sintassi, genera un albero sintattico. Questo albero è generato eseguendo una strategia di ricerca sul testo che può essere: 

(i) depth first, se attivazione della produzione una per volta, e torna indietro se fallisce,

(ii) breadth first: se attivazione delle produzioni in parallelo


\subsection{Parsing}
Il parser è un algoritmo che una testa la correttzza di un testo rispetto ad una specifica grammatica.

Il lessico mostra a quale simbolo non terminale appartiene una parola del linguaggio.


\subsection{Information Extraction}
An information extraction system is a cascade of transducers or modules that, at each step, add structure and often lose information, hopefully irrelevant, by applying rules that are acquired manually and/or automatically.

L'information retrieval ricerca testi e li presenta all'utente.

L'information extraction analizza testi e fornisce all'utente solo le informazioni richieste.

I sistemi IE sono più costosi computazionalmente,  richiedono conoscenza (sono knowledge intensive) e sono specifici di un dominio applicativo. I sistemi IR no.

I task dell'IE sono:

\begin{enumerate}
	
	\item named entity recognition (NE): individuazione di entità (nomi propri, comuni, date, quantità);
	
	\item coreference resolution (CO): individuazione di relazioni di identità tra entità (identità NE e referenze anaforiche);
	
	\item template element construction (TE): aggiunge informazioni descrittive a NE e CO; 
	
	\item template relation construction (TR): individua relazioni tra entità TE; l'insieme dei TE costituisce una prima base di conoscenza applicabile al testo.
	
	\item scenario template production:
	
\end{enumerate}

La complessità del task IE dipende da (i) tipo di testo, (ii) dominio (argomento del testo) e (iii) scenario (evento di interesse)

Una architettura IE è costituita dai seguenti moduli:

\begin{enumerate}

	\item Text Zoner: il testo è diviso in frammenti;
	
	\item Pre-processor: identificazione frasi e POS;
	
	\item Filter: eliminazione di frasi irrilevanti per l'applicazione;
	
	\item Pre-parser: identificazione di strutture frasali minime;
	
	\item Parser: produce albero sintattico di ogni frase;
	
	\item Fragment Combiner: si combinano gli alberi sintattici di frasi diverse;
	
	\item Semantic Interpreter: produzione di un albero semantico;
	
	\item Lexical Disambiguation: risoluzione delle ambiguità lessicali;
	
	\item Coreference Resolution: risoluzione delle ambiguità referenziali;
	
	\item Template Generator: produzione dello scenario  template a partire dagi alberi semantici.
	
\end{enumerate}


\subsection{Sistemi Q/A}
Un sistema di Question Answering (Q/A) risponde a domande utilizzando la conoscenza ed il ragionamento nel processamento del linguaggio naturale.

Risponde alle domande del tipo 5W (who, which, where, when, why).

Un task di Q/A prevede le seguenti fasi:

\begin{enumerate}

	\item question analysis
	
	\item information retrieval
	
	\item answer extraction
	
	\item answer merging
	
\end{enumerate}


\subsection{Text Summarization}
Un sistema di Text Summarization seleziona le informazioni più importanti da sorgenti di linguaggio naturale, producendo una visione sintetica di queste informazioni per uno specifico task.

Questo sistema prende un testo, produce un estratto (frammento rilevante del testo originario), e da questo un abstract (risultato della summarization).

Questi sistemi possono essere 

\begin{enumerate}
	
	\item indicativi: selezionano i documenti rilevanti.
	
	\item informativi: selezionano le informazioni rilevanti da uno stesso documento.
	
	\item valutativi: valutano la qualità della sorgente.
\end{enumerate}


\subsection{Textual Entailment}
Il textual entailment è la relazione tra un testo $T$ ed una ipotesi $H$. Si dice che sussiste un textual entailment tra $T$ ed $H$ se il significato di $H$ può essere inferito dal significato di $T$.

Ad esempio, sussiste un textual entailment tra il testo "A acquired B" e l'ipotesi "A owns B".

Una funzione di entailment $e:(T,H)\rightarrow [0,1]$ è una funzione probabilistica che esprime la confidenza con cui può essere affermato uno specifico textual entailment.

Il textual entailment può essere:

\begin{enumerate}

	\item paraphrase entailment: la frase esprime lo stesso fatto dell'ipotesi (a meno di sinonimi).
	Ad esempio: "A acquired B" e "A bought B".
	
	\item strict entailment: la frase contiene più fatti, dai quali può essere inferita l'ipotesi.
	Ad esempio: "A acquired B, now A owns B"
	
\end{enumerate}