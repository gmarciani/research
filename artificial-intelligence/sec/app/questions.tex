\chapter{Domande}
\label{chp:questions}

In riferimento agli appunti:

\begin{enumerate}
	
	\item "un sistema tradizionale è implementato con if...then...else". Un sistema intelligente no?
	
	\item quale è la differenza tra first-order predicate logic e description-logic?
	
	\item abbiamo dato tante definizioni di ontologia: quale è quella di riferimento?
	
	\item che differenza c'è tra un linguaggio di conoscenza ed un linguaggio di ragionamento? Esempi?
	
	\item che differenza c'è tra semantica ed interpretazione?
	
\end{enumerate}

In riferimento alle slide:

\begin{enumerate}
	
	\item (pag 1.2) che differenza c'è tra descrivere e strutturare la realtà?

	\item (pag 1.4) Che differenza c'è tra "definire chiaramente idee poco chiare" e "concettualizzare situazioni fluide"?
	
	\item (pag 1.6) quale è la differenza tra iponimia e relazione tassonomica?
	
	\item (pag 1.12) quale è la differenza tra rappresentazione del significato e conoscenza esplicita?
	
	\item (pag 1.13) che differenza c'è tra semantica e pragmatica?
	
	\item (pag 1.15) come fa una ontologia a (i)rappresentare un processo di interpretazione, (ii) quali sono i diversi tipi di semantica?
	
	\item (pag 1.19) riconoscere nuovi collegamenti ontologici non fa parte del processo di acquisizione (vedi pag 18)? 
	
	\item (pag 1.22) quale è la differenza tra simulazione e applicazione di casi simili?
	
	\item (pag 1.23) quale è un esempio di ragionamento di meta-livello?
	
	\item (pag 1.26) ontologia e teoria ontologica sono sinonimi? Possiamo dire che ontologia:teoriaontologica=grammatica formale:linguaggio formale?
	
	\item (pag 1.31) una regola di meta-ragionamento è una regola di valenza generale a prescindere dal dominio applicativo?
	
	\item (pag 1.42) Per un sistema intelligente ciò che esiste è ciò che può essere rappresentato, o ciò che è rappresentato?
	
	\item (pag 1.43) Nella ipotesi open-world, ciò che non è dichiarato è (i) indeterminato o (ii) falso?
	
	\item (pag 1.46) cosa si intende per ereditarietà multipla? non è un aspetto intrinseco della tassonomia? L'inferenza è esterna all'ontologia o interna ad essa?
	
	\item (pag 1.66) quale è la definizione formale di linguaggio ontologico logic-based?
	
	\item (pag 1.67) quindi il sistema è libero di inferire ed esplicitare ulteriori iponimie non dichiarate?
	
	\item (pag 1.72) che differenza c'è tra ambiguità di significato e conflitto di interpretazione?
	
	\item (pag 1.74) che altri oggetti ci stanno oltre alle classi, le relazioni e le funzioni?
	
	\item (pag 1.75) in parole povere, commitment ontologico vuol dire "attenersi una ontologia"?
	
	\item (pag 1.76) una funzione di interpretazione associa dunque ad un vocabolo un/una concetto/relazione?
	
	\item (pag 1.77) che vuol dire che I assegna estensioni rispetto a D, di tutti gli elementi del vocabolario? che vuol dire che M rappresenta una situazione corrispondente ad un mondo possibile?
	
	\item (pag 1.78) che differenza c'è tra teoria logica e ontologia? che differenza c'è tra vocabolario e lessico?
	
	\item (pag 1.79) spiegare tutto.
	
	\item (pag 1.82) che differenza c'è tra conoscenza di dominio e conoscenza operazionale?
	
	\item (pag 1.88) fare esempio.
	
	\item (pag 1.109) perchè le ontologie decentralizzano il semantic web?
	
	\item (pag 1.121) quando si fa data integration nelle ontologie, al mapping ontologico segue sempre il merging dell'ontologia?

	\item (pag 2.25) l'analisi del contesto è un tipo di analisi semantica? farsi spiegare il punto sulla pragmatica.
	
	\item (pag 2.28) possiamo dire che l'ambiguità dovuta ad anafora è un'ambiguità referenziale?
	
	\item (pag 2.48) non dovrebbe essere simbolo non terminale?
	
	\item (pag 2.53) secondo questa definizione anche un algoritmo che conta le parole è un sistema di information extraction?
	
	\item (pag 2.54) perchè con grandi volumi di dati, IE è più efficiente di IR?
	
	\item (pag 2.60) più del 90\% di cosa?
	
	\item (pag 2.63) come fa ad essere consapevole delle descrizioni se ha solo raccolto entità?
	
	\item (pag 2.63) i TEs costituiscono una base di conoscenza, nel senso di una ontologia, o nel senso di istanziazione di una ontologia pregressa?
	
	\item (pag 2.66) spiegare ST.
	
	\item (pag 2.71) come identifico un frammento? come faccio a dire che una frase è irrilevante?
	
	\item (pag 2.84) strict entailment, chi è la frase e chi è l'ipotesi?	
	
\end{enumerate}

In riferimento al libro di testo \cite{cimiano2014ontology}:

\begin{enumerate}

	\item cosa può essere al centro del processo interpretativo, se non le ontologie? (abstract)
	
	\item quale è la differenza tra formalization fo domain knowledge e construction of meaning representation? (abstract)	
	
	\item quale è la differenza tra inferenza semantica ed inferenza pragramtica? (pag. 1)
	
	\item in cosa consistono i metodi semi-automatici per la creazione di ontology lexica? (pag. 4)
	
	\item quale è la differenza tra knowledge-based e knowledge-driven NLP? (pag. 10)	

\end{enumerate}
				