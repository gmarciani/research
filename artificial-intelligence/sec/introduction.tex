\chapter{Introduction}
\label{chp:introduction}

Un sistema intelligente usa la conoscenza per ragionare.

I sistemi di reasoning basati su conoscenza motivano l'output, i sistemi numerici no.
Il ragionamento può essere dunque esplicitato.

La conoscenza è rappresentata in linguaggio naturale, sia in forma scritta che orale.  Il linguaggio naturale è lo strumento di cui disponiamo per comunicare la conoscenza. La maggior parte della conoscenza è scritta. Per questo è importante saper interpretare il linguaggio naturale.

Il processo cognitivo dell'essere umano è volto alla identificazione e definizione di modelli mentali per la descrizione della realtà.

La conoscenza è formalizzata per essere condivisibile.

La conoscenza di un sistema intelligente si riferisce a oggetti, eventi e processi.

La conoscenza di un sistema intelligente consta di strutture dati per la rappresentazione dell'informazione e procedure interpretative. 

La conoscenza condivisa permette all'essere umano di eseguire il processo di interpretazione frasale, andando oltre al significato letterale arricchendolo di inferenza semantica e pragmatica.

La conoscenza deve essere modulare.

La conoscenza è utile se permette ad un agente di fornire una soluzione migliore di quella che avrebbe scelto altrimenti.