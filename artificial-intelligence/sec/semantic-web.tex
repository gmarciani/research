\chapter{Semantic Web}
\label{chp:semantic-web}

Il semantic web è un'estensione del Web in cui all'informazione è associato un significato. Ciò permette una migliore interazione uomo-macchina, macchina-macchina.

I dati sul Web sono rappresentati come triple $(soggetto,predicato,oggetto)$.

L’informazione nel SW è rappresentata in RDF (Resouce Description Framework), un data model per descrivere
in maniera strutturata tutto ciò che possa essere identificato con una URI (Uniform Resource Identifier).

Un nodo RDF può essere una URI, un blankNode o un literal.

Le ontologie supportano il Semantic Web fornendo conoscenza e regole di inferenza condivise.

L’aspetto peculiare del semantic web è la decentralizzazione: le strutture locali possono essere modellizzate da ontologie.

La natura distriuita del Semantic Web rende necessario il mapping semantico tra le diverse ontologie ed il merging di ontologie distinte in un'unica ontologia.