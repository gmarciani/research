\chapter{Cifrari di blocco}
\label{chp:block-ciphers}

Un \textit{cifrario di blocco} è un crittalgoritmo operante su unità simboliche di dimensione fissa.

Un cifrario di blocco può essere:
\begin{description}
  \item[a sostituzione] sostituisce unità simboliche con altre unità simboliche.
  \item[a permutazione] permuta i simboli di uno stesso blocco. È vulnerabile ad attacchi di crittanalisi statistica in quanto ha una bassa diffusione. Inoltre, essendo vulnerabile ad attacchi basati su anagrammi, una chiave vicina a quella corretta può rivelare sezioni del plaintext.
  \item[a prodotto] esegue una composizione di sostituzioni e permutazioni.
\end{description}

\section{Tipi di sostituzione}
La sostituzione può aderire ad una alle seguenti discipline:

\begin{description}
  \item[monoalfabetica] sostituisce un simbolo in chiaro con un simbolo cifrato appartenente allo stesso alfabeto.

  \item[polialfabetica] sostituisce un simbolo in chiaro con un simbolo cifrato appartenente ad un alfabeto differente\footnote{oppure uno stesso alfabeto ordinato diversamente}.   La resistenza alla crittanalisi statistica dipende dal numero di alfabeti adottati.

  \item[poligrammica] sostituisce gruppi r-dimensionali con gruppi r-dimensionali.

  \item[omofonica] sostituisce un simbolo in chiaro con un simbolo cifrato preso casualmente da un sottinsieme di simboli di cardinalità proprozionale alla sua frequenza. In questo modo la si raggiunge un'alta diffusione, rendendo crittanalisi statistica più difficoltosa. Ciò non è più vero per plaintext molto lunghi, nei quali la legge dei grandi numeri neutralizza la diffusione. Comporta inoltre l'espansione dei dati.
\end{description}


\section{Forma generale}
La forma generale di un cifrario di blocco è definita dal cosiddetto \textit{cifrario affine}

\begin{equation}
\begin{cases}
  E_{k}(m)=k_{1}m+k_{2} \mod n \\
  D_{k}(c)=k_{1}^{-1}(c-k_{2}) \mod n
\end{cases}
\end{equation}

Qualora $k_{1}=k$ e $k_{2}=0$, si ha il cosiddetto \textit{cifrario semplice}.
Qualora $k_{2}=k$ e $k_{1}=0$, si ha il cosiddetto \textit{cifrario shift}.


\section{Vigenere}
Il \textit{Vigenére} è un cifrario di blocco con sostituzione polialfabetica.
Dato un alfabeto ordinato $\Sigma$ con $N$ simboli, un periodo $t\in\mathbf{N}$, una chiave $k=k_{1}...k_{t}$ e un plaintext $m=m_{1}m_{2}m_{3}...$, il ciphertext $c=c_{1}c_{2}c_{3}...$ è definito come:

\begin{equation}
  c_{i}=m_{i}+k_{i \mod t}(\mod N)
\end{equation}

dove l'indice di posizione $i$ individua l'$i$-esima permutazione $\Sigma_{i}$ dell'alfabeto $\Sigma$ da utilizzare nella sostituzione monoalfabetica.

\subsection{Varianti}
Del cifrario di Vigenere esistono numerose varianti. Le più famose sono:
\begin{description}
  \item[Beaufort-Vigenere] la trasformazione è $c_{i}=m_{i}-k_{i \mod t}(\mod N)$;

  \item[Vigenere composto] ottenuto componendo $R$ cifrari di Vigenere con periodi diversi, ovvero con chiavi $k^{(r)}=k_{1}^{(r)}...k_{t^{(r)}}^{(r)}$, in cui
  $c_{i}=m_{i}+\sum_{r=1}^{R} k_{i \mod t_{r}}^{(r)}(\mod N)$; questo cifrario è equivalente ad un cifrario di Vigenere di periodo $t=lmc(t^{1},...,t^{(R)})$;

  \item[running-key Vigenere] versione running-key del Vigenere classico.
\end{description}

\subsection{Vulnerabilità}
Il cifrario di Vigenère può essere rotto individuando la lunghezza della chiave ed eseguendo un analisi statistica del ciphertext parametrizzata dalla lunghezza della chiave.

\begin{description}
  \item[Metodo Charles-Babbage] determina la lunghezza $t$ della chiave come fattore della \textit{distanza} tra le ripetizioni di una stessa sequenza; determina la chiave identificando lo \textit{shift} dei simboli mediante crittanalisi statistica.
  \item[Kasiki Test] determina la lunghezza $t$ della chiave come massimo comune divisore delle \textit{distanze} tra le ripetizioni di una stessa sequenza; determina la chiave mediante crittanalisi statistica considerando il cifrario come $t$ \textit{cifrari di Cesare} intercalati.
  \item[Kappa Test] ipotizza la lunghezza $t$ della chiave; dispone il ciphertext in una matrice a $t$ colonne; calcola l'\textit{indice di coincidenza} medio delle colonne; $t$ è corretta se l'indice di coincidenza medio è prossimo a quello di un linguaggio naturale.
\end{description}


\section{German ADFGVX}
Il \textit{German ADFGVX} è un cifrario di blocco a prodotto sviluppato dai tedeschi nel 1918, riconosciuto come uno dei più sofisticati cifrari classici.

Il cifrario realizza la cifratura mediante:
\begin{description}
  \item[sostituzione monoalfabetica] dato l'alfabeto ordinato $\Sigma=(ADFGVX)$, e una matrice $M_{1}:\Sigma \times \Sigma$, esegue la sostituzione $s \rightarrow (ij).M_{1}[i,j]=s$.

  \item[trasposizione] data la chiave segreta $k=k_{1}...k_{t}$, il testo sostituito viene inserito per righe in una matrice $M_{2}:\mathbf{N}_{t} \times t$, dove $k$ è la chiave segreta; le colonne di $M_{2}$ vengono trasposte secondo l'ordine alfabetico di $k$; il ciphertext è il contenuto di $M_{2}$ letto per colonne.
\end{description}


\section{Block modes}
Un \textit{block mode}\footnote{anche detto \textit{modalità di esecuzione}.} definisce una composizione di cifrari di blocco. Un block mode permette dunque di impiegare un cifrario di blocco per cifrare un messaggio più lungo di un blocco.

Un block mode è valutato in termini di \textit{condizioni di identità dei messaggi}, \textit{dipendenza dei blocchi}, \textit{propagazione dell'errore} ed \textit{efficienza}.

\begin{description}
  \item[Electronic Codebook (ECB)] ogni blocco è cifrato indipendente dagli altri.

  \begin{equation}
  \begin{cases}
    c_{i}=E_{k}(p_{i}) \\
    p_{i}=D_{k}(c_{i})
  \end{cases}
  \end{equation}

  Non nasconde i data patterns; l'errore non si propaga; necessita padding; vulnerabile ad attachi replay e known-plaintext; cifratura/decifratura parallelizzabile; utilizzato per cifrare piccole quantità di dati.


  \item[Cipher Block Chaining (CBC)] ogni blocco è cifrato utilizzando il blocco ciphertext precedente. Inizializzato con IV;

  \begin{equation}
  \begin{cases}
    c_{i}=E_{k}(p_{i} \oplus c_{i-1}) & c_{0}=IV \\
    p_{i}=D_{k}(c_{i}) \oplus c_{i-1} & c_{0}=IV
  \end{cases}
  \end{equation}

   un errore nel plaintext compromette quel blocco, un errore nel ciphertext compromette quel blocco e quello successivo, la perdita di un bit ciphertext compromette tutta la decifratura; necessita padding; necessita messa in sicurezza del IV; decifratura parallelizzabile; utilizzato per grandi quantità di dati.


  \item[Cipher Feedback (CFB)] ogni blocco è cifrato con uno shift-register che scorre con gli $r$ bit del blocco ciphertext precedente; shitf-register inizializzato con IV;

  \begin{equation}
  \begin{cases}
    c_{i}=p_{i} \oplus SelectLeft_{r}(E_{k}(ShiftLeft_{r}(S_{i-1})c_{i-1})) & S_{0}=IV \\
    p_{i}=c_{i} \oplus SelectLeft_{r}(E_{k}(ShiftLeft_{r}(S_{i-1})c_{i-1})) & S_{0}=IV
  \end{cases}
  \end{equation}

  L'errore è propagato per tutta la dimensione dello shuft-register (self-synchonization); cifratura/decifratura con la stessa funzione; non necessita padding; decifratura parallelizzabile; utilizzato come cifrario di flusso.

  \item[Output Feedback (OFB)] ogni blocco è cifrato con uno shift-register che scorre con gli $r$ bit dello shift-register precdente; lo shift register è inizializzato a IV.

  \begin{equation}
  \begin{cases}
    c_{i}=p_{i} \oplus SelectLeft_{r}(E_{k}(ShiftLeft_{r}(S_{i-1}))) & S_{0}=IV \\
    p_{i}=c_{i} \oplus SelectLeft_{r}(E_{k}(ShiftLeft_{r}(S_{i-1}))) & S_{0}=IV
  \end{cases}
  \end{equation}

  Gli shift-register possono essere precaricati, in quanto non dipendono dai blocchi ciphertext prodotti precedentemente; cifratura/decifratura non parallelizzabili; non necessita di padding; la chiave non deve esere riutilizzata; utilizzato come cifratore a flusso in canali rumorosi.

  \item[Counter (CTR)] ogni blocco è cifrato con un contatore inizializzato a IV;

  \begin{equation}
  \begin{cases}
    c_{i}=p_{i} \oplus (E_{k}(S_{i})) & S_{i}=IV+i\\
    p_{i}=c_{i} \oplus (D_{k}(S_{i})) & S_{i}=IV+i
  \end{cases}
  \end{equation}

  Cifratura/decifratura parallelizzabili, in quanto i blocchi sono cifrati in modo indipendente l'uno dall'altro; necessita di padding; la chiave non deve essere riutilizzata; usato come cifrature a flusso in comunicazioni ad alta efficienza.

\end{description}
