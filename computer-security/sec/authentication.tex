\chapter{Autenticazione}
\label{chp:authentication}

L'autenticazione è il processo di identificazione \textit{identificazione unilateralmente/bilateralmente} di un'entità A con un'entità B. L'identificazione deve essere \textit{non trasferibile} \footnote{B non può reciclare l'identificazione di A per identificarsi come A con C.} e \textit{non impersonificabile}\footnote{Z(A) non può utilizzare l'identificazione di A con B per identificarsi come A con B.}.

Un protocollo di autenticazione è basato sul fornire in real-time \textit{qualcosa che si conosce}, \textit{qualcosa che si ha}, \textit{qualcosa che si è} o \textit{un luogo dove si è}.
Un protocollo di \textit{autenticazione forte} prevede la combinazione di due o più di questi meccanismi.

I protocolli di autenticazione venogno confrontati in base a \textit{reciprocità dell'autenticazione}, \textit{efficienza computazionale}, \textit{efficienza di comunicazione}, \textit{costo di implementazione}, \textit{coinvolgimento TTP}, \textit{garanzie di sicurezza} e \textit{gestione delle chiavi}.


\section{Passwords}
Una \textit{password fissa} è vulnerabile ad \textit{attacchi replay, brute-force e dictionary}.
Per aumentare la sicurezza di una password fissa sono necessari: \textit{alta entropia della password}, \textit{salting}, \textit{cifratura dei file password con funzioni hash non lineari} e \textit{rallentamento della verifica}.

Una \textit{passphrase} è una password realizzata da una frase di senso compiuto. Una \textit{passkey} è una passphrase mappata in una chiave pseudo-random.
Una \textit{graphical password} è una password definita da una sequenza ordinata di click su supporti grafici.

Una \textit{one-time password} è una password che permette l'autenticazione una sola volta all'interno di una finestra temporale di riferimento. Può essere implementata (i) condividendo la successione $\{p_{n}\}_{n}$ di password tra le entità che si devono autenticare, oppure (ii) condividendo la prima password $p_{0}$ e la funzione one-way $H$ per ricavare la seguente $p_{i+1}=H(p_{i})$.

Un \textit{Personal Number Identificator (PIN)} è una breve sequenza numerica che permette di autenticarsi con una \textit{master-key ad alta entropia} fornita da un qualcosa che si ha. Tipicamente, la procedura di verifica limita il numero di tentativi possibili.


\section{Challenge-Response}
Un protocollo \textit{Challenge-Response (CR)} è un protocollo di autenticazione in cui il claimer si identifica rispondendo correttamente (response) ad un quesito (challenge) posto dal verifier.

La sicurezza è garantita da un legame crittografico (chiave simmetrica/asimmetrica) e tempo-variante (timestamp, sequenze numeriche pseudo-random) tra la challenge e la response.

Questi protocolli realizzano l'autenticazione nei seguenti scenari:

\begin{description}
  \item[auntenticazione a chiave simmetrica] A si autentica con B dimostrando di saper rispondere correttamente ad una challenge che può essere basata su numeri casuali e/o timestamp. L'invio della challenge e del response sono cifrati secondo uno schema crittografico a chiave simmetrica.

  \item[autenticazione a chiave asimmetrica] A si autentica con B dimostrando di possedere la propria chiave privata. A decifra una challenge cifrata con la propria chiave pubblica, e fornisce una risposta firmata digitalmente.
  La coppia di chiavi utilizzata per l'autenticazione non dovrebbe essere utilizzata per altri scopi.

  \item[autenticazione X.509] standard di autenticazione bilaterale con chiave asimettrica mediante scambio di messaggi in cui ciascuna entità inserisce numeri random, timestamp e proprio certificato digitale.
\end{description}

\bigskip
\begin{algorithm}[H]
  \caption{Challenge-Response a chiave simmetrica (unilaterale)}
  \label{alg:authentication-challenge-response-unilateral-symmetric}
  \SetAlgoNoLine
  B $\rightarrow$ A: $E(k:r_{B})$\\
  A $\rightarrow$ B: $E(k:r_{A},r_{B})$\\
\end{algorithm}

\bigskip
\begin{algorithm}[H]
  \caption{Challenge-Response a chiave simmetrica (bilaterale)}
  \label{alg:authentication-challenge-response-bilateral-symmetric}
  \SetAlgoNoLine
  B $\rightarrow$ A: $E(k:r_{B})$\\
  A $\rightarrow$ B: $E(k:r_{A},r_{B})$\\
  B $\rightarrow$ A: $E(k:r_{A},r_{B})$\\
\end{algorithm}

\bigskip
\begin{algorithm}[H]
  \caption{Challenge-Response a chiave asimmetrica (unilaterale)}
  \label{alg:authentication-challenge-response-unilateral-asymmetric}
  \SetAlgoNoLine
  B $\rightarrow$ A: $E(k_{A}:r_{B},B)$\\
  A $\rightarrow$ B: $E(k_{A}:r_{B})$\\
\end{algorithm}
\bigskip

Questi protocolli sono vulnerabili ad \textit{attacchi replay}, \textit{parallel-session}, \textit{man-in-the-middle}, \textit{reflection} e \textit{known-text}.


\section{Zero-Knowledge}
Un protocollo \textit{Zero-Knowledge (ZK)} è un protocollo di dimostrazione interattivo che gode della proprietà di zero knowledge, ovvero in cui non si rivela null'altro che la veridicità della dimostrazione \footnote{vedi noto esempio di \textit{Peggy nela caverna}.}.

Il protocollo gode delle seguenti proprietà\footnote{correttezza e completezza sono proprietà dei protocolli di dimostrazione interrativi; mentre la zero-knowledge è specifica dei protocolli zero-knowledge.}:
\begin{description}
  \item[completezza] se l'affermazione è vera, un claimer onesto può dimostrarla vera ad un verifier.
  \item[correttezza] se l'affermazione è falsa, un claimer disonesto non può dimostrarla vera al verifier\footnote{la probabilità che ci riesca può essere resa bassa a piacere.}.
  \item[zero-knowledge] se l'affermazione è vera, un verificatore disonesto\footnote{anche detto \textit{simulatore}.} non potrà sapere altro che la sua veridicità.
\end{description}

La struttura generale di un protocollo ZK prevede $N$ round costituiti dalle seguenti fasi:
\begin{description}
  \item[commitment] il claimer fornisce al prover una prova di conoscenza;
  \item[challenge] il verifier fornisce al claimer una dimostrazione da verificare;
  \item[response] il claimer dimostra al verifier la veridicità della dimostrazione;
  \item[verification] il verifier verifica la veridicità fornita dal claimer.
\end{description}

I vantaggi principali dei protocolli ZK sono: \textit{resistenza ad attacchi chosen-text}, \textit{non degrado della sicurezza nel tempo}. Inoltre, rispetto a protocolli a chiave asimmetrica, sono più efficienti computazionalmente.
Tuttavia, non sono efficienti a livello di comunicazione, e sono basati su problemi la cui intrattabilità computazionale non è stata formalmente dimostrata.


\subsection{ZK con RSA}
Il seguente è un esempio di protocollo ZK basato su RSA. In questo esempio A dimostra a B di conoscere la propria chiave privata $d$, senza però comunicarla. Quindi basta dimostrare di saper decifrare un messaggio $c=m^{e} \mod n$, dove $(n,e)$ è la chiave pubblica di A.

\bigskip
\begin{algorithm}[H]
  \caption{ZK basato su RSA}
  \label{alg:authentication-zero-knowledge-rsa}
  \SetAlgoNoLine
  A sceglie un intero random $r \in [0,n)$\\
  A $\rightarrow$ B: $y=r^{e}$\\
  B $\rightarrow$ A: $c=m^{e} mod n$\\
  A decifra $m=c^{d} mod n$ \\
  A $\rightarrow$ B: $z=rm$\\
  B verifica che $z^{e}=yc$
\end{algorithm}

Questo protocollo è vulnerabile alla \textit{impersonificazione}: per autenticarsi come A, a Z(A) basterebbe infatti inviare a B $y=r^{e}/c$ e $z=r$.


\subsection{Cut\&Choose}
Il protocollo \textit{Cut\&Choose} è un protocollo di autenticazione zero-knowledge che risolve la vulnerabilità alla impersonificazione eseguendo round variabili di verifica.

Poichè l'autenticazione è soddisfatta se e solo se tutti i round sono soddisfatti, la proprietà di correttezza è soddisfatta con probabilità direttamente proporzionale al numero di round.

\bigskip
\begin{algorithm}[H]
  \caption{C\&C basato su RSA}
  \label{alg:authentication-zero-knowledge-cut-and-choose}
  \SetAlgoNoLine
  A sceglie un intero random $r \mod n$;\\
  A $\rightarrow$ B: $y=r^{e}$;\\
  B $\rightarrow$ A: $b \in {0,1}$;\\
  A $\rightarrow$ B: $z=rm^{b}$;\\
  B verifica che $z^{e}=yc^{b}$;
\end{algorithm}

L'intuizione del protocollo consiste nel fornire due challenge: una per dimostrare la conoscenza ed una per dimostrare l'onestà. Un claimer onesto risponde correttamente ad entrambe. Un claimer disonesto risponde correttamente con probabilità $\frac{1}{2^{N}}$, dove $N$ è il numero di round.


\subsection{ZK e isomorfismo tra grafi}
Un noto esempio di protocollo Cut\&Choose basa la sua sicurezza sull'intrattabilità computazionale del \textit{problema dell'isomorfismo tra grafi}. In questo esempio il segreto è l'isomorfismo $H$ tra $G_{1}$ e $G_{2}$, la cui conoscenza è provata fornendo una permutazione $G_{3}$ di $G_{1}$, e provandone l'isomorfismo con $G_{1}$ o $G_{2}$. Il protocollo è sicuro in quanto un claimer disonesto non può provare contemporaneamente l'isomorfismo di $G_{3}$ sia con $G_{1}$ che con $G_{2}$.


\subsection{Feige-Fiat-Shamir}
Il \textit{Feige-Fiat-Shamir (FFS)} è un protocollo di autenticazione zero-knowledge di tipo Cut\&Choose, la cui sicurezza si basa sull'intrattabilità computazionale del \textit{problema della radice quadrata in modulo di un intero a fattorizzazione incognita}.

\bigskip
\begin{algorithm}[H]
  \caption{Feige-Fiat-Shamir}
  \label{alg:authentication-zero-knowledge-feige-fiat-shamir}
  \SetAlgoNoLine
  È noto $n=pq$ dove $p,q$ sono grandi primi random\\
  A ha un segreto $s$ coprimo di $n$
  A sceglie un intero random $r \in [0,n)$\\
  A $\rightarrow$ B: $v=s^{2} \mod n$;\\
  A $\rightarrow$ B: $x=r^{2} \mod n$;\\
  B $\rightarrow$ A: $b \in \{0,1\}$;\\
  A $\rightarrow$ B: $y=rs^{b} \mod n$;\\
  B accetta A se $y^{2}=xv^{b} \mod n$;\\
\end{algorithm}


\section{Secret Sharing}
Un \textit{protocollo di secret sharing} partiziona una dato segreto $D$ in $n$ componenti $D_{1},...,D_{n}$, ognuna delle quali è destinata ad un'entità.
Ogni componente $D_{i}$ è necessaria alla determinazione di $D$.
Questi protocolli eliminano la presenza di single point of failure ed impediscono ad un singolo partecipante di accedere ad una risorsa senza il consenso degli altri.

Uno \textit{schema threshold $(k,n)$} è uno schema di secret sharing in cui il dato segreto è partizionato in $n$ componenti e sono necessarie almeno $k\leq n$ di esse per ricostruirlo.


\subsection{Shamir-SS}
Lo \textit{Shamir Secret Sharing (Shamir-SS)} è un protocollo di secrte sharing con schema threshold $(k,n)$ dove $k<n$.
Dato un segreto $S$, viene definita una funzione

\begin{equation}
  f(x)=S+\sum_{i=1}^{k}a_{i}x^{i}
\end{equation}

Ad ogni partecipante viene consegnato un punto $(i,f(i))$.
Ogni sottoinsieme di $k$ partecipanti può determinare $S$ mediante \textit{interpolazione lineare}\footnote{mediante il \textit{polinomio di Lagrange}.} dei $k$ punti.


\section{Autenticazione biometrica}
L'autenticazione biometrica è basata su \textit{pattern recognition di aspetti fisiologici o comportamentali} dell'individuo.
Permette di verificare l'identità di un individuo (\textit{autenticazione ono-to-one}), o di determinare l'identità di un individuo (\textit{autenticazione many-to-one}).

Gli schemi di autenticazione biometrica sono in continua evoluzione, ma al momento soffrono di problemi legati alla privacy, alla salute, ai falsi negativi.


\section{CAPTCHA}
Un CAPTCHA\footnote{\textit{Completely Automated Public Turing test to tell Computers and Humans Apart}} è una classe di metodi per distinguere un accesso umano da un accesso computerizzato.
Tipicamente un CAPTCHA chiede di riconoscere sequenze alfanumeriche e disegni durante la sottomissione di form onde evitare password-guessing e attacchi DOS.


\section{Kerberos}
\textit{Kerberos} è un protocollo di mutua autenticazione client/server con crittografia a chiave simmetrica (DES).
Permette di contrastare attacchi attivi di impersonificazione, contraffazione dell'identità e replay.

Il protocollo prevede: un \textit{client (C)} che richiede i servizi di un \textit{service server (SS)}, un \textit{Ticket-Granting Server (TGS)} che autentica C con SS, ed un \textit{Authentication Server (AS)} che autentica C con TGS.

\bigskip
\begin{algorithm}[H]
  \caption{Kerberos}
  \label{alg:kerberos}
  \SetAlgoNoLine
  C $\rightarrow$ AS: $(uname_{C})$\\
  AS ottiene dal proprio DB la password di C $pwd$\\
  AS ottiene la chiave $k_{C/TGS}$ per cifrare la comunicazione C/TGS\\
  AS ottiene la chiave $k_{AS/TGS}$ per cifrare la comunicazione AS/TGS\\
  AS produce un ticket $T_{TGS}$ di autenticazione per C con TGS\\
  AS $\rightarrow$ C: $E_{pwd}(k_{C/TGS})$\\
  AS $\rightarrow$ C: $c_{1}=E_{k_{AS/TGS}}(T_{TGS})$\\
  C $\rightarrow$ TGS: $(c_{1},id_{service})$\\
  C $\rightarrow$ TGS: $E_{k_{C/TGS}}(id_{C},timestamp_{C})$\\
  TGS ottiene la chiave $k_{C/SS}$ per cifrare la comunicazione C/SS\\
  TGS ottiene la chiave $k_{TGS/SS}$ per cifrare la comunicazione TGS/SS\\
  TGS produce un ticket $T_{SS}$ di autenticazione per C con SS\\
  TGS $\rightarrow$ C: $E_{k_{C/TGS}}(k_{C/SS})$\\
  TGS $\rightarrow$ C: $c_{2}=E_{k_{TGS/SS}}(T_{SS})$\\
  C $\rightarrow$ SS: $c_{2}$\\
  C $\rightarrow$ SS: $E_{k_{C/SS}}(id_{client},timestamp_{C})$\\
  SS si fida di C\\
  SS $\rightarrow$ C: $E_{k_{C/SS}}(timestamp_{SS})$\\
  Se $(timestamp_{C},timestamp_{SS})$ sono validi, allora $C$ si fida di $SS$\\
  C può richiedere i servizi di SS, ed SS può erogarli
\end{algorithm}

Le vulnerabilità di Kerberos sono:
(i) AS è un single point of failure;
(ii) attacchi a DES;
(iii) modifica del clock di TGS per reciclare i ticket;
(iv) server malevoli simulanti AS e/o TGS;


\section{Autenticazione dei messaggi}
L'autenticazione di un messaggio può essere realizzata mediante \textit{cifratura simmetrica/asimmetrica del messaggio} o mediante il rilascio di un \textit{Message Authentication Code (MAC)} $F(m,k)$.
Un MAC è dunque un \textit{digest} prodotto da una funzione one-way con alta resistenza alle collisioni. Un MAC può essere cifrato con un qualsivoglia crittoalgoritmo.

Un MAC realizza autenticazione del messaggio, autenticazione della sorgente, non repudiation. Se al messaggio è allegato un timestamp, il MAC certifica anche l'ordine di sequenza del messaggio.
Questa autenticazione protegge da impersonificazione, alternazione del messaggio, attacco replay e DOS.

Un \textit{Hash-MAC (HMAC)} è un MAC cifrato con una funzione hash crittografica (e.g. SHA1, MD5).
