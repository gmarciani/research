\chapter{Firma digitale}
\label{chp:digital-signature}

Una \textit{firma digitale} è una stringa che permette di associare un messaggio con l'entità che lo ha creato.
Uno schema di firma digitale è un meccanismo costituito da un algoritmo di \textit{generazione delle chiavi}, un algoritmo di \textit{generazione della firma} ed uno di \textit{verifica della firma}.

Uno schema di firma digitale può essere
\textit{con appendice} se è necessario il messaggio originale per verificare la firma;
\textit{con recovery del messaggio}, se il messaggio originale viene ricavato dalla firma, e non è pertanto necessario per verificare la firma.
Uno schema di firma digitale può inoltre essere
\textit{randomizzato} se la firma viene scelta casualmente da un set di firme;
\textit{deterministico} se vi è un'unica firma.

Uno schema di firma digitale prevede un algoritmo di \textit{generazione delle chiavi}, un algoritmo di \textit{generazione della firma} ed uno di \textit{verifica della firma}.

In uno schema di firma digitale con appendice, non è il messaggio originale ad essere firmato, bensì il suo hashing.

La firma digitale permette di realizzare autenticazione, integrità dei dati e non-repudiation. La sua applicazione più significativa è la certificazione delle chiavi pubbliche in un crittosistema a chiave asimmetrica.

I requisiti fondamentali di uno schema di firma digitale sono:
(i) $S_{A,k}$ deve essere efficiente da calcolare per ogni $k$;
(ii) $V_{A}$ deve essere efficiente da calcolare;
(iii) computazionalmente impossibile per un'entità $Z(A)$ determinare un messaggio $m \in M$ e una firma $s \in S$ tali che $V_{A}(h(m),s)=True$.

\bigskip
\begin{algorithm}[H]
  \caption{Firma digitale con appendice}
  \label{alg:digital-signature-appendix}
  \SetAlgoNoLine
  A seleziona una chiave $k \in \mathcal{R}$;\\
  A calcola $\tilde{m}=h(m)$ e $s*=S_{A,k}(\tilde{m})$;\\
  A invia $(m,s*)$ a B.\\

  B riceve $(m,s*)$ da A;\\
  B ottiene la chiave pubblica $V_{A}$;\\
  B calcola $\tilde{m}=h(m)$ e $u=V_{A}(\tilde{m},s*)$;\\
  B accetta la firma se $u=True$.
\end{algorithm}

\bigskip
\begin{algorithm}[H]
  \caption{Firma digitale con message recovery}
  \label{alg:digital-signature-message-recovery}
  \SetAlgoNoLine
  A seleziona una chiave $k \in \mathcal{R}$;\\
  A calcola $\tilde{m}=h(m)$ e $s*=S_{A,k}(\tilde{m})$;\\
  A invia $(m,s*)$ a B.\\

  B riceve $(m,s*)$ da A;\\
  B ottiene la chiave pubblica $V_{A}$;\\
  B calcola $\tilde{m}=h(m)$ e $u=V_{A}(\tilde{m},s*)$;\\
  B accetta la firma se $u=True$.
\end{algorithm}


\subsection{Vulnerabilità}
Gli attacchi a schemi di firma digitale sono volti a forgiare firme digitali, ovvero generare firme digitali autenticabili.
Uno schema di firma digitale può essere soggetto a:
(i) \textit{rottura totale}, se l'attaccante può generare la firma digitale della vittima;
(ii) \textit{forgiatura selettiva}, se l'attaccante può generare la firma digitale per una specifica classe di messaggi;
(iii) \textit{forgiatura esistenziale}, se l'attaccante può generare la firma digitale per almeno un messaggio, ma non di sua scelta.

Gli attacchi sono classificati in:
(i) \textit{key-only}: l'attaccante conosce solo la chiave pubblica della vittima;
(ii) \textit{known-message}: l'attaccante conosce la firma di un insieme di messaggi non di sua scelta;
(iii) \textit{chosen-message}: l'attaccante conosce la firma di un insieme di messaggi di sua scelta;
\textit{adapative chosen-message}: l'attaccante può far firmare messaggi di sua scelta, e richiedere firme che dipendano dalle firme precedentemente ottenute.

\section{DSA}
Il \textit{Digital Signature Algorithm (DSA)} è uno schema di firma digitale randomico con appendice, basato sull'intrattabilità computazionale del \textit{problema del logaritmo discreto}\footnote{il DSA può essere visto come una variante dello \textit{schema ElGamal}.}.
.

\bigskip
\begin{algorithm}[H]
  \caption{DSA: generazione delle chiavi}
  \label{alg:dsa-key-generation}
  \SetAlgoNoLine
  Seleziona un numero primo $p$ a 1024 bit\\
  Seleziona un numero primo $q$ a 160 bit che divida $(p-1)$\\
  Calcola $\alpha=g^{\frac{p-1}{q}} \mod p$ dove $g \in \mathcal{Z}_{p}*$\\
  Seleziona un intero $a \in [1,q-1]$\\
  Calcola $y=\alpha^{a} \mod p$\\
  La chiave pubblica è $(p,q,\alpha,y)$, la chiave privata è $a$
\end{algorithm}

\bigskip
\begin{algorithm}[H]
  \caption{DSA: generazione della firma}
  \label{alg:dsa-signature}
  \SetAlgoNoLine
  Seleziona un intero $k \in (0,q)$\\
  Calcola $r=(\alpha ^{k} \mod p) \mod q$\\
  Calcola $s=k^{-1}[H(m)+ar] \mod q$\\
  La firma è $(r,s)$
\end{algorithm}

\bigskip
\begin{algorithm}[H]
  \caption{DSA: verifica della firma}
  \label{alg:dsa-verification}
  \SetAlgoNoLine
  Ottieni la chiave pubblica $(p,q,\alpha,y)$, la firma $(r,s)$ ed il messaggio $m$\\
  Verifica $r\in (0,q)$\\
  Verifica $s\in (0,q)$\\
  Calcola $w=s^{-1} \mod q$\\
  Calcola $u_{1}=wH(m)\mod q$\\
  Calcola $u_{2}=rw\mod q$\\
  Calcola $v=(\alpha^{u_{1}}y^{u_{2}} \mod p) \mod q$\\
  La firma è verificata se $v=s$
\end{algorithm}


\section{RSA-DS}
\textit{RSA-Digital-Signature (RSA-DS)} è uno schema di firma digitale deterministico con message recovery basato su RSA.
La chiave privata viene utilizzata per firmare il messaggio, e la chiave pubblica per verificarne la firma.


\section{Blind signature}
Uno \textit{schema di blind signature} è uno schema di firma digitale in cui il firmatario firma il messaggio senza poterne conoscere il contenuto.
Questi schemi sono utilizzati nei \textit{protocolli per la privacy} (e.g. e-voting, e-cash).

Questi schemi richiedono:
(i) un meccanismo $S_{B}(m)$ di firma digitale dell'entità $B$ sul messaggio $m$;
(ii) una funzione $f$, detta \textit{blinding function}, ed una funzione $g$, detta \textit{unblinding function}, tali che $g(S_{B}(f(m)))=S_{B}(m)$.

Uno schema di blinding signature prevede le seguenti fasi:
(i) \textit{blinding} in cui A invia a B il messaggio $f(m)$;
(ii) \textit{signing} in cui B firma $f(m)$ ottenendo $S_{B}(f(m))$;
(iii) \textit{unblinding} in cui A ricava la firma di B $S_{B}(m)=g(S_{B}(f(m)))$;


\subsection{Chaum Blind Signature}
\textit{Chaum Blind Signature (Chaum-BS)} è uno schema di blind signature la cui blinding function è basata su RSA.

\bigskip
\begin{algorithm}[H]
  \caption{Chaum Blind Signature}
  \label{alg:authentication-blind-signature-chaum}
  \SetAlgoNoLine
  A ottiene la chiave pubblica di B $(n,e)$\\
  A sceglie un plaintext $m$ ed un numero intero $r \in (0,n)$\\
  A $\rightarrow$ B: $b=mr^{e} \mod n$\\
  B $\rightarrow$ A: $s=b^{d} \mod n$\\
  A calcola $u=sr^{-1} \mod n=m^{d}$
\end{algorithm}


\section{Certificati digitali}
I crittosistemi a chiave asimmetrica assumono l'autenticità delle chiavi pubbliche. Questa viene generalmente ottenuta mediante certificati digitali rilasciati da Certification Authority.

Un \textit{certificato digitale} è una struttura dati che consiste di una \textit{data part} contenente informazioni realtive all'entità da certificare, ed una \textit{signature part} contenente la firma digitale della Certification Authority che rilascia il certificato.

Lo \textit{standard X.509} prevede nel data part informazioni generali dell'entità, la sua chiave pubblica, la durata di validità della stessa

Un \textit{Certification Authority (CA)} è un TTP (Trusted Third Party) che assicura l'autenticità di una chiave pubblica mediante rilascio di un certificato.

Un CA rilascia un certificato digitale per A solo dopo aver adottato misure di autenticazione. Il CA può creare la coppia di chiavi e trasmettere la chiave privata ad A su un canale sicuro. Oppure l'entità A genera la coppia di chiavi e trasmette la chiave pubblica al CA, il quale verifica l'autenticità mediante challenge-response.
