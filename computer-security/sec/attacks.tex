\chapter{Attacchi}
\label{chp:attacks}

Un \textit{attaccante passivo} è un agente che ha un accesso illegittimo in lettura a informazioni protette.
Un \textit{attaccante attivo} è un agente che manipola in modo illegittimo informazioni protette.

\section{Crittanalisi}
Gli attacchi crittanalitici variano in funzione della conoscenza assunta sul crittosistema da attaccare:

\begin{description}
  \item[known ciphertext] conoscenza del solo ciphertext, sul quale è possibile condurre analisi statistiche.
  \item[known plaintext] conoscenza di porzioni corrispondenti  plaintext ciphertext.
  \item[chosen plaintext] conoscenza ciphertext corrispondenti a plaintext opportunamente scelti.
  \item[chosen ciphertext] conoscenza plaintext corrispondenti a ciphertext opportunamente scelti.
  \item[adapted text] conoscenza dell'operato di un cifrario in corrispondenza di chiavi diverse scelte in funzione dell'output osservato. È la tipologia di attacco più potente.
\end{description}

La \textit{crittanalisi lineare} consiste nella determinazione di sistemi di equazioni lineari che approssimano il comportamento di un cifrario. Questa analisi sfrutta le proprietà di linearità delle trasformazioni eseguite dal cifrario.

La \textit{crittanalisi differenziale} consiste nello studio di come le differenze nei dati forniti in ingresso ad una funzione crittografica possano incidere sulle differenze risultanti in uscita dalla stessa. Tale studio è volto a scoprire comportamenti non casuali della funzione, sfruttabili per recuperare la chiave segreta.

\section{Forward Search}
Un attacco \textit{Forward Search} consiste nel produrre un plaintext $m$, cifrarlo in $c=E_{e}(m)$ con chiave di crittazione nota $e$, poi provare a decifrarlo tentando chiavi di decrittazione $m'=D_{d_{i}}(c)$, finchè no si sia trovata quella chiave $d*$ tale che $D_{d*}(c)=m$.

\section{Brute-Force e Dizionario}
Un attacco \textit{brute-force} consiste nel determinare una chiave esplorando tutto lo spazio di ricerca sul quale la chiave è definita.

Un attacco \textit{dictionary} consiste nel determinare una chiave esplorando un sottospazio di ricerca sul quale si suppone che la chiave sia definita.

\section{Meet-In-The-Middle}
Un attacco \textit{Meet-In-The-Middle (MITM)} consiste nel determinare una chiave in un cifrario composto $c=G_{k}(F_{k}(m))$ ricercando le coincidenze tra il \textit{forward mapping} $F_{k}(m)$ ed il \textit{backward mapping}, $G_{k}^{-1}(c)$. Si determina dunque una chiave $k.F_{k}(m)=G_{k}^{-1}(c)$.

\section{Parallel session}
Un attacco \textit{parallel session} consiste nel sovrapporre opportunamente i messaggi scambiati durante esecuzioni concorrenti di un protocollo.

\bigskip
\begin{algorithm}[H]
  \caption{Parallel Session}
  \label{alg:attack-parallel-session}
  \SetAlgoNoLine
  --- Esecuzione regolare --- \\
  A $\rightarrow$ B: $E(k_{A,B}:N_{A})$;\\
  B $\rightarrow$ A: $E(k_{A,B}:N_{A}+1)$;\\
  --- Esecuzione attaccata --- \\
  A $\rightarrow$ B: $E(k_{A,B}:N_{A})$;\\
  Z(B) $\rightarrow$ A: $E(k_{A,B}:N_{A})$;\\
  A $\rightarrow$ Z(B): $E(k_{A,B}:N_{A}+1)$;\\
  Z(B) $\rightarrow$ A: $E(k_{A,B}:N_{A}+1)$;
\end{algorithm}

\section{Binding}
Un attacco \textit{binding} consiste nel convincere A che la chiave pubblica di B è in realtà la chiave pubblica di Z(B). Questo attacco può essere implementato come attacco parallel session.

\bigskip
\begin{algorithm}[H]
  \caption{Binding}
  \label{alg:attack-binding}
  \SetAlgoNoLine
  --- Esecuzione regolare --- \\
  C $\rightarrow$ S: $C,S,N_{C}$;\\
  S $\rightarrow$ C: $E(k_{S}^{-1}:S,C,N_{C},N_{S})$;\\
  --- Esecuzione attaccata --- \\
  C $\rightarrow$ S: $C,S,N_{C}$;\\
  Z(C) $\rightarrow$ S: $C,Z,N_{C}$;\\
  S $\rightarrow$ Z(C): $E(k_{S}^{-1}:S,C,N_{C},N_{Z})$;\\
  Z(S) $\rightarrow$ C: $E(k_{S}^{-1}:S,C,N_{C},N_{Z})$;\\
\end{algorithm}

\section{Freshness}
Un attacco \textit{freshness} consiste nell'inoltrare nell'esecuzione corrente del protocollo opportuni messaggi registrati da esecuzioni precedenti.

\section{Type Flaws}
Un attacco \textit{type flaws} consiste nell'alterare opportunamente  il valore semantico di una o più componenti di un messaggio, mantenendone la validità.

\section{Hardware timing}
Un attacco \textit{hardware timing} consiste nel determinare gli input a blocchi crittografici mediante lo studio delle performance del hardware che li implementa.

\section{Fluher-Martin-Shamir}
L'attacco \textit{Fluher-Martin-Shamir (FMS)} è un attacco passivo known-plaintext al cifrario RC4 che consente di determinare la chiave segreta.
Questo attacco sfrutta la \textit{correlazione tra i primi bytes del keystream e la chiave segreta} \footnote{Le debolezze evidenziate da questo attacco sono alla base di software di attacco al WEP, come \textit{AirSnort} ed \textit{AirCrack}.}.
Si esegue un attacco statistico sui pacchetti che reinizializzano il keystream. Esaminando un gran numero di questi pacchetti, si osserva un bias verso i byte della chiave.


\section{Attacco a WEP}
\textit{Wired Equivalent Privacy (WEP)}\footnote{il nome deriva dal fatto che il WEP è stato definito con l'obiettivo  di realizzare una protezione equivalente o migliore di quella fornita dalle reti cablate.} è un protocollo di sicurezza per reti wireless\footnote{IEEE 802.11}.
Il protocollo si propone di garantire confidenzialità, controllo degli accessi e integrità dei dati.

Ogni stazione\footnote{anche gli Access Point.} fornisce una chiave per accedere alla rete. Il protocollo di distribuzione della chiave è lasciato ai vendors.
Utilizza RC4 come keystream generator con chiave condivisa da ogni stazione.
Utilizza CRC-32 come algoritmo di checksum dei pacchetti.

Idealmente non dovrebbe usare la stessa chiave per messaggi diversi, pertanto ogni pacchetto è codificato utilizzando un keystream parametrizzato da un Initialization Vector (IV) distinto. Quindi IV non dovrebbe essere riutilizzato nel ciclo di vita di una chiave. Tuttavia IV viene resettato relativamente spesso\footnote{le modalità di gestione del IV sono lasciate ai vendors. Alcuni resettano l'IV ad ogni accensione della card, incrementando l'IV ad ogni pacchetto.}.

Debolezze: stessa chiave condivisa da tutte le stazioni; la chiave è ASCII quindi lo spazio delle chiavi è limitato; il keystream è reciclato; essendo un protocollo di collegamento cripta i dati dello strato di rete, ed è dunque nota la struttura dei primi byte dei pacchetti cifrati; il protocollo di autenticazione WEP fornisce all'attaccante una coppia $(m,c)$; linearità di CRC-32.


\subsection{Collisioni IV}
È possibile eseguire un \textit{attacco dizionario}, sfruttando le collisioni di IV.
L'attaccante ottiene una coppia plaintext/ciphertext $(p_{1},c_{1})$ potendo dunque ottenere il proprio IV, e, sfruttando la linearità dello XOR, ottiene il keystream $KS_{IV}=RC4(IV,k)=p_{1} \oplus c_{1}$.
L'attaccante popola un dizionario contenente i keystream $KS$ indicizzati per IV.
Per decifrare un messaggio $<\overline{c},\overline{IV}>$, calcola $\overline{m}=\overline{c} \oplus KS_{\overline{IV}}$.


\subsection{Integrità dei dati}
È possibile eseguire un \textit{attacco alla integrità dei dati}, sfruttando la linearità dello XOR e la linearità di CRC-32\footnote{$CRC(m \oplus d)=CRC(m) \oplus CRC(d)$.}.
Un attaccante può modificare bit arbitrari di un messaggio cifrato sapendo deterministicamente come modificare i bit del checksum.
L'attacante desidera inviare il messaggio manipolato $\tilde{m}=m \oplus \delta$, dove $\delta$ è il cambiamento da indurre su $m$ per ottenere $\tilde{m}$.
Genera dunque il messaggio cifrato manipolato $\tilde{c}=c \oplus <\delta,CRC(\delta)>$, e trasmette $<\tilde{c},IV>$.
Il ricevente legge il pacchetto $<m',CRC(m')>$, che soddisfa la seguente:

\begin{equation}
\begin{split}
<m',CRC(m')> & = RC4(IV,k) \oplus \tilde{c}\\
& = RC4(IV,k) \oplus c \oplus <\delta,CRC(\delta)>\\
& = <m,CRC(m)> \oplus <\delta,CRC(\delta)>\\
& = <m \oplus \delta,CRC(m) \oplus CRC(\delta)>\\
& = <\tilde{m},CRC(\tilde{m})>
\end{split}
\end{equation}


\subsection{Message injection}
È possibile eseguire un \textit{attacco di message injection}, sfruttando la linearità dello XOR ed il fatto che CRC-32 non dipende dalla chiave.
Un attaccante ottiene una coppia plaintext/ciphertext $<m,CRC(m)>,<c,IV>$.
Ricava il keystream $RC4(IV,k)=c \oplus <m,CRC(m)>$.
Avendo il keystream, può immetere un messaggio arbitrario $\tilde{m}$, cifrandolo come $\tilde{c}=<\tilde{m},CRC(\tilde{m})> \oplus RC4(IV,k)$, e rasmettendo infine il pacchetto $<\tilde{c},IV>$.

\section{Classi di attacco}

\begin{description}
  \item[sniffing]
  \item[spoofing] ARP spoofing, DNS spoofing
  \item[smurfing]
  \item[phishing]
  \item[poisoning] DNS poisoning
  \item[flooding] SYN flooding, SYN/ACK flooding, HTTP flooding, UDP flooding
  \item[stripping] SSL stripping
  \item[injection] message injection, SQL injection
  \item[redirection] DNS redirection
  \item[man-in-the-middle]
\end{description}

\section{Malwares}

\begin{description}
  \item[worm]
  \item[virus]
  \item[trojan]
  \item[adware]
  \item[spyware]
  \item[keylogger]
  \item[cryptolocker]
  \item[stripping]
  \item[injection]
  \item[redirection]
  \item[man-in-the-middle]
  \item[back-door]
  \item[logic bomb]
\end{description}

\section{Botnets}

\section{Attacchi DoS}
Un attacco \textit{Denial-of-Service (DoS)} ha lo scopo di mettere fuori uso o impedire l'erogazione di un servizio ai suoi utenti legittimi.

Questi attacchi possonmo essere implementati attraverso: \textit{consumo di risorse}, \textit{manipolazione (alterazione/distruzione) delle informazioni} o \textit{manipolazione hardware}
