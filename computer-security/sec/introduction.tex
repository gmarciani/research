\chapter{Introduzione}
\label{chp:introduction}

La \textit{crittologia} è la disciplina che si occupa delle scritture nascoste, nel suo duplice significato: da un lato la \textit{crittografia} si occupa di occultare il significato di un testo, dall'altro la \textit{crittanalisi} si occupa di decifrare un testo cifrato senza conoscere il metodo usato per cifrarli.
La \textit{steganografia} è la disciplina che si occupa di nascondere l'esistenza di un messaggio.
La stenografia è più debole della crittografia, in quanto la scoperta di un messaggio stenografato implica la conoscenza del suo significato; la crittografia invece nasconde il significato di un messaggio anche quando questo sia noto, qualora non sia nota la chiave per decifrarlo.

La crittografia si occupa di definire funzioni biettive tra un testo in chiaro (\textit{plaintext}) ed un testo cifrato (\textit{ciphertext}), parametrizzate da una chiave (\textit{secret key}). La \textit{crittografia a chiave simmetrica} prevede un'unica chiave sia per cifrare che per decifrare. La \textit{crittografia a chiave asimmetrica} prevede una \textit{chiave pubblica} per cifrare ed una \textit{chiave privata} per decifrare.

L'obiettivo della crittografia è tenere cifrati i dati per il tempo di vita del loro valore informativo. Una soluzione crittografica fragile permette la decrittazione durante il tempo di vita dell'informazione. Il trade-off della crittografia è dunque tra tempo e costo della soluzione adottata rispetto alla vita utile e valore dell'informazione.

\section{Cifrari}
Un \textit{cifrario} realizza la cifratura/decifratura secondo uno schema di cifratura definito come

\begin{equation}
  (M,C,K,E,D)
\end{equation}

dove
$M$ è lo \textit{spazio dei plaintext},
$C$ è lo \textit{spazio dei ciphertext},
$K$ è lo \textit{spazio delle chiavi},
$E=\{E_{k_{e}:k_{e}\in K}\}$ è l'insieme delle \textit{funzioni di cifratura},
$D=\{D_{k_d}:k_{d}\in K\}$ è l'insieme delle \textit{funzioni di decifratura}.
$K,M,C$ sono tali che 

\begin{equation}
\forall k_{e}\in K \exists! k_{d}\in K . D_{k_{d}}=E_{k_{e}}^{-1}
\end{equation}

Diciamo che $c=E_{k_{e}}(m)$ è la cifratura di $m$ sotto la chiave di cifratura $k_{e}$, mentre $m=D_{k_{d}}(c)$ è la decifratura di $c$ sotto la chiave di decifratura $k_{d}$.

Un cifrario è tale che
\begin{equation}
\begin{cases}
  G: \mathbf{N} \rightarrow \mathcal{K} \times \mathcal{K} \\
  E: \mathcal{M} \times \mathcal{K} \rightarrow \mathcal{C} \\
  D: \mathcal{C} \times \mathcal{K} \rightarrow \mathcal{M} \\
\end{cases}
\end{equation}
dove $G$ è il \textit{generatore della chiave}.

Un cifrario deve godere della \textit{proprietà di reversibilità}
\begin{equation}
m=D_{k_{d}}(E_{k_{e}}(m))\quad \forall m\in M, k_{e}\in K \exists! k_{d}\in K
\end{equation}
In generale, $K$ deve essere abbastanza grande affinchè il crittosistema possa essere robusto ad attacchi brute-force.

Un cifrario \textit{auto-key} utilizza come chiave lo stesso plaintext.
Un cifrario \textit{running-key} utilizza una chiave di lunghezza pari al plaintext.


\subsection{Proprietà}
Alcune proprietà fondamentali sono:
\begin{description}
	\item[confusione] avere rumore nel processo di cifratura. Questo si raggiunge facendo dipendere ogni bit plain da tutta la key ed aumentando la complessità del rapport plaintext/key.
	\item[diffusione] disperdere le proprietà statistiche del plaintext. Questa proprietà aumenta la robustezza ad attacchi di crittanalisi statistica.
	\item[effetto valanga] il cambiamento di un bit plaintext implica il cambiamento di almeno metà ciphertext.
\end{description}


\subsection{Cifrari e Codici}
Un cifrario realizza la cifratura basandosi sulla sostituzione di unità simboliche independentemente dal loro valore semantico, secondo una chiave segreta.

Un \textit{codice} realizza la cifratura basandosi sulla sostituzione di unità linguistiche (parole, gruppi di parole o frasi) secondo un dizionario segreto, detto \textit{codebook}.

Un codice è più difficile da rompere di un cifrario, in quanto
(i) lo spazio delle chiavi è maggiore
(ii) realizza una maggiore diffusione ed è pertanto più robusto ad attacchi di crittanalisi statistica.
Tuttavia è più vulnerabile ad attacchi known-plaintext, in quanto si può fare reverse engineering sul codebook.


\section{Algoritmi, protocolli e sistemi}
Un \textit{crittalgoritmo} è un algoritmo che realizza una primitiva crittografica.
Un \textit{protocollo crittografico} è un protocollo che impiega una qualche primitva crittografica. In prima battuta un protocollo crittografico è classificato in:
\begin{description}
  \item[self-enforcing] il protocollo è garantito senza intervento da parte di entità terze; le irregolarità sono individuate dal protocollo stesso.
  \item[arbitrated] il protocollo è garantito da un TTP\footnote{ovvero \textit{Trusted Third Party}.} real-time.
  \item[judged] il protocollo è garantito da un TTP coinvolto solo in caso di irregolarità. Il protocollo è dunque suddiviso in un sotto-protocollo semplice ed un sotto-protocollo arbitrated.
\end{description}

Un \textit{crittosistema} è un sistema che adotta crittoalgoritmi e protocolli crittografici. 
Un crittosistema deve soddisfare i seguenti requisiti:
(i) schema di cifratura pubblico
(ii) $E,D$ efficienti con $k$ nota
(iii) $D$ NP-hard con $K$ ignota e proporzionale alla sua lunghezza
(iv) alta diffusione
(v) alta confusione.


\section{Principi}
Si distinguono due tipi di sicurezza:

\begin{description}
	\item[sicurezza perfetta] un cifrario è perfettamente sicuro se $\prob{m \mid c}=\prob{m}$.
	\item[sicurezza computazionale] un cifrario è computazionalmente sicuro se il tempo di decrittazione con chiave ignota è superiore al tempi di vita dell'informazione cifrata.
\end{description}

\begin{description}
  \item[Kerchoff] la sicurezza di un crittosistema dipende solo dalla segretezza della chiave. In particolare essa non dipende dalla segretezza del crittalgoritmo.

  \item[Shannon] un buon cifratore deve prevedere sia una fase di sostituzione che una fase di permutazione. Il componente che realizza la sostituzione è detto \textit{S-box}. Il componente che realizza la permutazione è detto \textit{P-box}.

  Shannon ha dimostrato che
  \begin{equation}
    sicurezzaperfetta \Leftrightarrow H(\mathcal{K}) \geq H(\mathcal{M})
  \end{equation}
  dove $H(x)$ è l'entropia dello spazio $x$.
\end{description}
