\chapter{Introduzione}
\label{chp:introduction}

La crittologia è la disciplina che si occupa delle scritture nascoste, nel suo duplice significato: da un lato la crittografia si occupa di occultare il reale significato dei testi, dall'altro la crittanalisi riguarda la decifrazione di testi occultati senza conoscerne a priori il metodo usato per cifrarli.
La steganografia è la disciplina che si occupa di nascondere l'esistenza di un messaggio. La stenografia è più debole della crittografia, in quanto la scoperta di un messaggio stenografato implica la conoscenza del suo significato; mentre la crittografia nasconde il significato di un messaggio noto, qualora non sia nota la chiave per decifrarlo.

La crittografia si occupa di definire funzioni di trasformazione reversibile tra un testo in chiaro ed un testo cifrato, parametrizzate da una chiave segreta.



Un crittosistema realizza la cifratura di dati. L'obiettivo di un crittosistema è tenere cifrati i dati per il tempo di vita del loro valore informativo. Un crittosistema fragile permette la decrittazione durante il tempo di vita dell'informazione. Il trade-off di un crittosistema è tra tempo e costo del crittosistema rispetto alla vita utile e valore dell'informazione.
Un crittosistema a chiave privata prevede $k_{e}=k_{d}$, mentre un crittosistema a chiave pubblic prevede $k_{e}\neq k{d}$.

\begin{description}
  \item[confusione] avere rumore nel processo di cifratura. Questo si raggiunge facendo dipendere ogni bit plain da tutta la key ed aumentando la complessità del rapport plaintext/key.
  \item[dispersione] disperdere le proprietà statistiche del plaintext.
  \item[distribuzione della chiave] la key deve essere scambiata tra gli interlocutori. Questo è un problema in quanto può essere intercettata durante lo scambio, è un processo che non scala con il numero degli utenti.
\end{description}

\section{Principi generali}
\begin{description}
  \item[Kerchoff] la sicurezza di un crittosistema non dipende dalla segretezza del crittoalgoritmo, ma solo dalla segretezza della chiave.
  \item[Shannon] un buon cifratore deve prevedere una fase di sostituzione (S-box) ed una fase di permutazione (P-box). Shannon ha provato che H$(K)\geq H(M)$ è condizione necessaria alla incondizionata sicurezza di un crittosistema, ovvero se l'entropia della chiave non è minore dell'entropia del plaintext. Questa condizione implica che $H(M\mid C)=H(M)$, ovvero che il ciphertext non fornisce alcuna informazione sul plaintext.
\end{description}

Un crittosistema deve dunque soddisfare i seguenti requisiti: schema di cifratura pubblicamente noto, $E$ proietta uniformamente $M$ su $C$, $E,D$ efficienti con $k$ nota, $D$ NP-hard con chiave sconosciuta, con complessità proporzionale alla lunghezza della chiave.

La distribuzione uniforme aumenta la robustezza ad attacchi di crittanalisi statistica. La maggiore dimensione di $K$ aumenta la robustezza ad attacchi brute-force.

Un crittosistema è perfettamente sicuro se ...
Un crittosistema è computazionalmente sicuro se ...

\section{Cifrario}
Un crittosistema realizza la cifratura/decifratura secondo uno schema di cifratura definito come $(M,C,K,E,D)$, dove
$M$ è lo spazio dei plaintext,
$C$ è lo spazio dei ciphertext,
$K$ è lo spazio delle chiavi,
$E=\{E_{k_{e}:k_{e}\in K}\}$ è l'insieme delle funzioni di cifratura,
$D=\{D_{k_d}:k_{d}\in K\}$ è l'insieme delle funzioni di decifratura
tali che $\forall k_{e}\in K \exists! k_{d}\in K . D_{k_{d}}=E_{k_{e}}^{-1}$.

Un crittosistema è tale che
$G:N\rightarrow K \times K$
$E: M \times K \rightarrow C$
$D: C \times K \rightarrow M$

Diciamo che $c=E_{k_{e}}(m)$ è la cifratura di $m$ sotto la chiave di cifratura $k_{e}$, mentre $m=D_{k_{d}}(c)$ è la decifratura di $c$ sotto la chiave di decifratura $k_{d}$.

Un crittosistema deve soddisfare il requisito di reversibilità, ovvero
$m=D_{k_{d}}(E_{k_{e}}(m))\quad \forall m\in M, k_{e}\in K \exists! k_{d}\in K$.
$K$ deve essere abbastanza grande affinchè il crittosistema posso essere robusto ad attacchi brute-force.

L'\textit{effetto a valanga} è una proprietà di un crittoalgoritmo che consiste nel cambiamento di akmeno metà ciphertext in corrispondenza del cambiamento di un singolo bit plaintext.

\section{Cifrari e Codici}
Un cifrario realizza la cifratura basandosi sulla sostituzione di unità simboliche independentemente dal loro vlore semantico, secondo una chiave segreta. La distribuzione della chiave è efficiente in quanto la chiave è tipicamente corta. È debole ad attacchi di crittologia statistica.

Un codice, o codebook crittografico, realizza la cifratura basandosi sulla sostituzione di unità linguistiche (parole, gruppi di parole o frasi) secondo un dizionario segreto, detto codebook. La distribuzione della chiave è inefficiente, in quanto il dizionario è tipicamente grande. È robusto ad attacchi di crittologia statistica.

Un codice è più difficile da rompere di un cifrario, in quanto lo spazio delle chiavi è maggiore, la distribuzione statistica viene alterata pertanto è robusto ad attacchi di crittanalisi statistica, ma è debole ad attacchi known-plaintext oin quanto si può fare reverse engineering sul codebook.


\section{Protocolli}
Un protocollo crittografico è un protocollo che impiega un crittoalgoritmo.
Un protocollo crittografico è in prima battuta classificato in:
\begin{description}
  \item[self-enforcing] il protocollo è garantito senza intervento da parte di entità terze; le irregolarità sono individuate dal protocollo stesso.
  \item[arbitrated] il protocollo è garantito da un TTP real-time.
  \item[judged] il protocollo è garantito da un TTP coinvolto solo in caso di irregolarità. Il protocollo è dunque suddiviso in un sotto-protocollo semplice ed un sotto-protocollo arbitrated.
\end{description}
