\chapter{Crittografia a chiave pubblica}
\label{chp:public-key-cryptography}

Un \textit{crittosistema a chiave asimmetrica}\footnote{anche detto \textit{crittosistema a chiave pubblica}.} realizza servizi crittografici in cui ogni entità dispone di una chiave \textit{chiave pubblica} $e$, che realizza la trasformazione di cifratura $E_{e}$, ed una corrispondente \textit{chiave privata} $d$, che realizza la trasformazione di decifratura $D_{d}$.

Un crittosistema a chiave asimmetrica è (computazionalmente) sicuro se è (computazionalmente) impossibile determinare $d$ a partire da $e$. La sicurezza di questi crittosistemi è garantita dall'intrattabilità di problemi matematici alla base della generazione della coppia di chiavi.

Un crittosistema a chiave pubblica è \textit{(parzialmente) rotto} se un attaccante ha decifrato uno specifico ciphertext.
Un crittosistema a chiave pubblica è \textit{completamente rotto} se un attaccante ha ricavato la chiave privata della vittima, in quanto questo gli permetterà di decifrare ogni ciphertext inviato alla vittima.

In generale, un crittosistema a chiave asimmetrica realizza la confidenzialità dei dati, senza garantire autenticità ed integrità,  Inoltre, esso è più lento di un crittosistema a chiave simmetrica.

Tipicamente, questi crittosistemi vengono impiegati per distribuire chiavi privati di protocolli crittografici a chiave simmetrica, o per cifrare piccole quantità di dati.

I crittosistemi a chiave asimmetrica sono vulnerabili ad \textit{attacchi chosen ciphertext}, in quanto la chiave di decifratura è pubblica.

I requisiti generali di un crittosistema a chiave asimmetrica sono:
(i) efficienza nella generazione della coppia di chiavi;
(ii) efficienza di cifratura con chiave pubblica;
(iii) efficienza di decifratura con chiave privata;
(iv) ogni chiave può decifrare un messagio cifrato con l'altra;
(v) computazionalmente impossibile calcolare la chiave privata a partire da quella pubblica;
(vi) computazionalmente impossibile determinare il plaintext a partire dalla chiave pubblica e dal ciphertext;
(vii) computazionalmente impossibile decifrare un ciphertext cifrato conla stessa chiave usata per cifrarlo.


\section{Protocollo Diffie-Hellman}
Il Diffie-Hellman è un protocollo di key agreement per crittosistemi a chiave pubblica, la cui sicurezza è basata sull'intrattabilità del problema del logaritmo discreto.
Il protocollo fornisce una soluzione pratica al problema della distribuzione della chiave su canale aperto.
Non garantendo l'autenticazione, può garantire segretezza contro attacchi passivi, ma no contro attacchi attivi.

La versione più semplice del protocollo è mostrata nell'Algoritmo~\ref{alg:diffie-hellman}, ed è la base di molti protocolli di key establishment, alcuni dei quali garantiscono anche l'autenticazione.

\bigskip
\begin{algorithm}[H]
  \caption{Diffie-Hellman (basic)}
  \label{alg:diffie-hellman}
  \SetAlgoNoLine
  A e B scelgono pubblicamente un numero primo $p$ ed un generatore $\alpha\in\mathcal{Z}_{p}*$ tali che $\alpha\in[2,p-2]$;
  A sceglie un numero casuale $x\in[1,p-2]$ e invia $X=\alpha^{x}\mod p$ a B;
  B sceglie un numero casuale $y\in[1,p-2]$ e invia $Y=\alpha^{y}\mod p$ ad A;
  A calcola la chiave condivisa $k=Y^{x} \mod p$;
  B calcola la chiave condivisa $k=X^{y} \mod p$;
  La chiave condivisa da A e B è $K$.
\end{algorithm}


\section{RSA}
RSA (Rivest-Shamir-Adleman) è uno schema di cifratura a chiave pubblica la cui sicurezza è basata sull'intrattabilità del problema di fattorizzazione degli interi.

Ogni entità genera una coppia di chiavi: una chiave pubblica $(n,e)$ ed una chiave privata $d$, dove $n$ è detto modulo, $e$ è detto esponente di crittazione e $d$ è detto esponente di decrittazione.

L'Algoritmo~\ref{alg:rsa-key-generation} mostra la generazione delle chiavi, mentre l'Algoritmo~\ref{alg:rsa-protocol} mostra il protocollo di comunicazione.

\bigskip
\begin{algorithm}[H]
  \caption{Generazione chiave RSA}
  \label{alg:rsa-key-generation}
  \SetAlgoNoLine
  Genera $p$ e $q$ grandi interi coprimi distinti.\\
  Calcola $n=pq$ e $\phi=(p-1)(q-1)$.\\
  Seleziona un intero random $e \in (1,\phi)$ tale che $gdc(e,\phi)=1$\\
  Calcola l'intero $d \in (1,\phi)$ tale che $ed\equiv 1 (\mod \phi)$.\\
  La chiave pubblica è $(n,e)$ e la chiave privata è $d$.\\
\end{algorithm}

\bigskip
\begin{algorithm}[H]
  \SetAlgoNoLine
  \caption{Protocollo comunicazione RSA}
  \label{alg:rsa-protocol}
  B ottiene la chiave pubblica $(n.e)$ di A. \\
  B rappresenta il messagio $m$ come un intero in $[0,n-1]$. \\
  B calcola il ciphertext $c=m^{e}(\mod n)$. \\
  B invia $c$ ad A. \\
  A riceve $c$ da B. \\
  A calcola il plaintext $m=c^{d}(\mod n)$.
\end{algorithm}

RSA funziona perchè:

$c^{d} =^{(1)} m^{ed} \equiv^{(2)} m \mod n =^{(3)} m$

\begin{description}
  \item[1] verificato per definizione $c=m^{e}$;

  \item[2] verificato in quanto:
    per definizione $ed\equiv 1 \mod \phi$, da cui $\exists k \in \mathcal{Z^{+}}.ed=1+k\phi$;
    per il teorema di Fermat sappiamo che $m^{(p-1)} \equiv 1 \mod p$, da cui $m^{k(p-1)(q-1)+1} \equiv m \mod p$ e $m^{k(p-1)(q-1)+1} \equiv m \mod q$;
    poichè per definizione $gcd(p,q)=1$, allora $m^{k(p-1)(q-1)+1} \equiv m \mod pq$, ovvero $m^{\phi +1} = m^{ed} \equiv m \mod n$.

  \item[3] verificato per definizione $n \in [0,n-1]$;
\end{description}

RSA è uno schema di cifratura lento, se paragonato a DES, considerato la sua controparte in chiave simmetrica. Per questo motivo viene principlamente utilizzato per la comunicazione di chiavi in schemi a chiave simmetrica su canali insicuri, crittagio di piccoli quantitativi di dati (e.g. PINs) e autenticazione con firma digitale.


Il problema RSA (RSAP) consiste nel determinare il plaintext $m$ dati il ciphertext $c$ e la chiave pubblica $(n,e)$. Attualmente non esiste un algoritmo efficiente per risolvere RSAP.

\subsection{Vulnerabilità}
Un crittosistema RSA può essere completamente rotto se:
si conosce la fattorizzazione del modulo $n=pq$;
si conosce $\phi$, anche senza fattorizzare $n$;

Un crittosistema RSA può essere parzialmente rotto se:
\begin{itemize}
  \item se si utilizza uno stesso esponente di crittazione $e$ piccolo per inviare uno stesso plaintext $m$ a più entità aventi moduli differenti a due a due primi $n_{i}$, il plaintext può essere decifrato risolvendo un sistema di equazioni. Questa vulnerabilità può essere risolta utilizzano $e$ grandi e preferibilmente distinti, e/o mediante il salting del plaintext\footnote{eseguire un salting di un plaintext vuol dire appendervi una stringa random prima di ogni sua cifratura cifratura.}.

  \item se si utilizza un esponente di decrittazione $d$ piccolo (approssimativamente $\frac{1}{4}$ dei bit del modulo $n$), questo può essere determinato a partire dalla chiave pubblica mediante un \textit{attacco di Wiener}. Questa vulnerabilità può essere risolta adottando $d$ lungo quanto $n$.

  \item se lo spazio dei messaggi è piccolo o prevedibile, un attaccante può criptare un messaggio $m$ mediante un \textit{attacco Forward Search}. Questa vulnerabilità può essere risolta mediante il salting del plaintext.

  \item la proprietà di omomorfismo di RSA, ovvero $m_{1}^{e}m_{2}^{e} \equiv c_{1}c_{2} \mod n$, lo rende vulnerabile ad un attacco \textit{adaptive chosen ciphertext}.
  Supponiamo che (i) l'attaccante voglia decifrare uno specifico ciphertext $c=m^{e} \mod n$ indirizzato ad A, e che (2) A decifri per l'attaccante qualunque ciphertext eccetto $c$. L'attaccante produce il ciphertext $c'=cx^{e} \mod n$, dove $x\in\mathcal{Z}_{n}$, e lo fa decifrare da A, ottenendo $m'=c'^{d}=c^{d}(x^{e})^{d}\equiv mx \mod n$, da cui $m=m'x^{-1} \mod n$.
  Questa vulnerabilità può essere risolta considerando fraudolenti tutti quei ciphertext il cui plaintext non rispetti opportuni vincoli strutturali.
\end{itemize}

Una vulnerabilità teorica, sono i cosidetti \textit{messaggi rivelati}, ovvero plaintext per cui vale $m^{e} \equiv m \mod n$. Di fatto sono talmente pochi che non rappresentano una vulnerabilità pratica.

\section{Key management}
I crittosistemi a chiave asimmetrica assumono l'autenticità delle chiavi pubbliche. Questa viene generalmente ottenuta mediante certificati digitali rilasciati da Certification Authority.

Un \textit{certificato digitale} è una struttura dati che consiste di una \textit{data part} contenente informazioni realtive all'entità da certificare, ed una \textit{signature part} contenente la firma digitale della Certification Authority che rilascia il certificato.

Lo \textit{standard X.509} prevede nel data part informazioni generali dell'entità, la sua chiave pubblica, la durata di validità della stessa

Un \textit{Certification Authority (CA)} è un TTP (Trusted Third Party) che assicura l'autenticità di una chiave pubblica mediante rilascio di un certificato.

Un CA rilascia un certificato digitale per A solo dopo aver adottato misure di autenticazione. Il CA può creare la coppia di chiavi e trasmettere la chiave privata ad A su un canale sicuro. Oppure l'entità A genera la coppia di chiavi e trasmette la chiave pubblica al CA, il quale verifica l'autenticità mediante challenge-response.
