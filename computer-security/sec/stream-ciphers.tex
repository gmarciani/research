\chapter{Cifrario a flusso}
\label{chp:stream-ciphers}

Un cifrario a flusso è un crittoalgoritmo che cifra il testo in chiaro un bit alla volta utilizzando funzioni di trasformazione con memoria.
Questi cifrari possono essere a chiave privata o chiave pubblica, ma ci concdentreremo solo sui primi.

Questi cifrari sono adatti per applicazioni con buffering limitato e in cui è necessario minimizzare la propagazione dell'errore.

Esistono pochi algoritmi pienamente definiti in letteratura, in quanto vengono tipicamente impiegati in applicazioni proprietarie e confidenziali.

Un cifrario a flusso sincrono ha un keystream dipendente dalla sola chiave. Questo cifrario necessita la sincronizzazione degli interlocutori e blocca immediatamente la propagazione dell'errore.
Un attaccante attivo può dunque far fallire facilmente la comunicazione compromettendo la sincronizzazione ed inoltre può prevedere gli effetti di manipolazione dei bit sul resto del testo.
L'esempio più comune è il cifrario a flusso binario-additivo, dove p,c,k sono stringhe binarie e $c_{i}=m_{i} \oplus z_{i}$.

Un cifrario a flusso asincrono ha un keystream dipendente dalla chiave e da una sottostringa del ciphertext.
Questo cifrario può ristabilire la sincronizzazione qualora si perda (quindi difficile individuare intrusioni sul canale), diffondere brevemente la propagazione dell'errore (quindi rendere possibile individuare corruzione dei bit), inoltre disperde la distribuzione statistica del plaintext. L'esempio più comune è il cifrario a blocco con CFB.

\section{Vernam}
Il cifrario di Vernam è un cifrario a flusso sincrono binario-additivo.
Se il keystream è random viene detto one-time pad ed è dimostrato perfettamente sicuro.

\section{One Time Pad}
Un \textit{cifrario One Time Pad (OTP)} è un cifrario di Vernam con keystream random.
Questo cifrario è \textit{perfettamente sicuro} a prescindere dalla distribuzione statistica del plaintext e ottimo nel senso che ha la chiave più piccola di tutti gli schemi di cifratura a chiave privata.

Dimostriamo che OTP è perfettamente sicuro.
Dobbiamo provare che $\mathcal{P}(m \mid c)=\mathcal{P}(m)$.
Abbiamo che $\mathcal{P}(m \mid c)=\frac{\mathcal{P}(c \mid m)\mathcal{P}(m)}{\mathcal{P}(c)}$.

Valgono le seguenti:

(i) $\mathcal{P}(c \mid m)=\mathcal{P}(k=m \oplus c)=\frac{1}{2^{n}}$;

(ii) $\mathcal{P}(c)=\sum_{m \in M}\mathcal{P}(c \mid m)\mathcal{P}(m)=\frac{1}{2^{n}}\sum_{m\in M}\mathcal{P}(m)=\frac{1}{2^{n}}$;

Pertanto $\mathcal{P}(m \mid c)=\frac{\mathcal{P}(c \mid m)\mathcal{P}(m)}{\mathcal{P}(c)}=\frac{\frac{1}{2^{n}}\mathcal{P}(m)}{\frac{1}{2^{n}}}=\mathcal{P}(m)$.

\section{Two Time Pad}
Un cifrario Two Time Pad (TTP) è un cifrario di Vernam con keystream costante.
Questo cifrario è perfettamente insicuro, in quanto il riutilizzo di una stessa chiave per plaintext diversi permette di effettuare semplicmenete attacchi known-ciphertext.

Dimostriamo che TTP non è perfettamente sicuro.
Inviaando due messaggi cifrati con la stessa chiave, ottengo
$c_{1}=m_{1} \oplus k$ e $c_{2}=m_{2} \oplus k$. Ponendo in XOR i due ciphertext ottengo
$c_{1} \oplus c_{2} =
m_{1} \oplus k \oplus m_{2} \oplus k =
m_{1} \oplus m_{2} \oplus k \oplus k =
m_{1} \oplus m_{2} 0 =
m_{1} \oplus m_{2}$
In particolare, se un attaccante conosce $c_{1}$ e $c_{2}$ perchè sniffati e $m_{1}$ perchè da lui stesso inviato, potrà decifrare $m_{2}$.

\section{RC4}
RC4\footnote{RC4 è un cifrario proprietario di cui si è diffusa in rete un'implementazione, ed è pertanto spesso chiamato Alleged-RC4 (ARC4).} è un cifrario a flusso semplice ed altamente performante utilizzato per produrre un keystream pseudo-random.

RC4 è stato adottato come generatore di keystream in molti protocolli (WEP, WPA, SSL, TLS) fino a quando è stato provato insicuro dall'\textit{attacco Fluhrer-Mantin-Shamir (FMS)} su WEP.

Il cifrario è costituito da una fase di \textit{key-scheduling (KS)} seguita da una fase di \textit{pseudo-random generation (PRG)}. Lo stato del cifrario è costituito da una S-box $S$ a 256 bytes parametrizzata dalla chiave segreta $K$, e da due indici $i,j$.

\bigskip
\begin{algorithm}[H]
  \caption{RC4 - Key Scheduling}
  \label{alg:rc4-key-scheduling}
  \SetAlgoNoLine
  \For{i from 0 to 255}{
    S[i] = i
  }
  j = 0\\
  \For{i from 0 to 255} {
    j = (j + S[i] + K[i mod len(K)]) mod 256\\
    swap(S[i],S[j])\\
  }
\end{algorithm}

\bigskip
\begin{algorithm}[H]
  \caption{RC4 - Pseudo-Random Generation}
  \label{alg:rc4-pseudorandom-generation}
  \SetAlgoNoLine
  i = 0\\
  j = 0\\
  \While{generatingOutput} {
    i = (i+1) mod 256\\
    j = (j + S[i]) mod 256\\
    swap(S[i],S[j])\\
    k = S[(S[i]+S[j])mod 256]\\
    \Return{k}
  }
\end{algorithm}

Il cifrario è altamente performante (10x rispetto a DES), efficiente da un punto di vista di memoria, garantisce un buon one-time-padding, ed è robusto ad attacchi di crittanalisi lineare.
Purtroppo, i primi bytes del keystream presentano una forte correlazione la chiave chiave, il keystream risulta inoltre leggermente polarizzato su specifiche sequenze. Infine lo spazio del keystream è limitato superiormente a $2^{1700}$ permutazioni, a prescindere dalla chiave segreta. Questo vuol dire che aumentare la lunghezza della chiave non migliora l'entropia.
