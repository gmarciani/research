\chapter{Cifrari di flusso}
\label{chp:stream-ciphers}

Un cifrario di flusso è un crittalgoritmo che esegue \textit{bit-a-bit} una \textit{trasformazioni con memoria reversibili}. La trasformazione non utilizza direttamente la chiave, bensì un \textit{keystream} da essa dipendente.

Questi cifrari sono adatti per applicazioni con buffering limitato e in cui è necessario minimizzare la propagazione dell'errore.

Un cifrario di flusso può essere:

\begin{description}
  \item[sincrono] il keystream dipende solo dalla chiave. La perdita di sincronizzazione sul keystream compromette definitivamente il cifrario. L'errore non si propaga. Un esempio è il cifrario di Vernam.
  \item[asincrono] il keystream dipende dalla chiave e da una sottostringa del ciphertext. La perdita di sincronizzazione sul keystream può essere recuperata. L'errore si propaga brevemente. Produce alta dispersione. Un esempio è il cifrario di blocco con block mode CFB.
\end{description}

Un cifrario a flusso asincrono ha un keystream dipendente dalla chiave e da una sottostringa del ciphertext.
Questo cifrario può ristabilire la sincronizzazione qualora si perda (quindi difficile individuare intrusioni sul canale), diffondere brevemente la propagazione dell'errore (quindi rendere possibile individuare corruzione dei bit), inoltre disperde la distribuzione statistica del plaintext. L'esempio più comune è il cifrario a blocco con CFB.


\section{Vernam}
Il \textit{cifrario di Vernam}\footnote{anche detto \textit{cifrario di flusso binario-addittivo}.} è un cifrario di flusso sincrono con trasformazione

\begin{equation}
  \begin{cases}
    c_{i}=m_{i} \oplus z_{i} \\
    m_{i}=c_{i} \oplus z_{i}
  \end{cases}
\end{equation}
Se il keystream è random viene detto one-time pad ed è dimostrato perfettamente sicuro.


\section{One Time Pad}
Il \textit{One Time Pad (OTP)} è un cifrario di Vernam con \textit{keystream random}. Questo cifrario è \textit{perfettamente sicuro}.

Dimostriamo che OTP è perfettamente sicuro.
Dobbiamo provare che 

\begin{equation*}
\prob{m \mid c}=\prob{m}
\end{equation*}

Poichè abbiamo:

\begin{equation*}
\prob{m \mid c}=\frac{\prob{c \mid m}\prob{m}}{\prob{c}}
\end{equation*}

e valgono le seguenti:

\begin{itemize}
	\item $\prob{c \mid m}=\prob{k=m \oplus c}=\frac{1}{2^{n}}$;
	
	\item $\prob{c}=\sum_{m \in M}\prob{c \mid m}\prob{m}=\frac{1}{2^{n}}\sum_{m\in M}\prob{m}=\frac{1}{2^{n}}$;
\end{itemize}

allora abbiamo che:

\begin{equation*}
\prob{m \mid c}=\frac{\prob{c \mid m}\prob{m}}{\prob{c}}=\frac{\frac{1}{2^{n}}\prob{m}}{\frac{1}{2^{n}}}=\prob{m}
\end{equation*}


\section{Two Time Pad}
Il \textit{Two Time Pad (TTP)} è un cifrario di Vernam con \textit{keystream costante}. Questo cifrario è \textit{completamente insicuro}, in quanto il riutilizzo di uno stesso keystream per plainbit diversi permette di rompere il cifrario con un semplice attacco known-text.

Dimostriamo che TTP non è completamente insicuro.
Inviando due messaggi cifrati con la stessa chiave, ottengo:

\begin{equation*}
\begin{cases}
c_{1}=m_{1} \oplus k \\
c_{2}=m_{2} \oplus k
\end{cases}
\end{equation*}

Da cui

\begin{equation}
\begin{split}
c_{1} \oplus c_{2} & = m_{1} \oplus k \oplus m_{2} \oplus k \\
& = m_{1} \oplus m_{2} \oplus k \oplus k \\
& = m_{1} \oplus m_{2} \oplus 0 \\
& = m_{1} \oplus m_{2}
\end{split}
\end{equation}

Ora, se un attaccante conosce $c_{1}$ e $c_{2}$ perchè sniffati e $m_{1}$ perchè da lui stesso inviato, potrà decifrare $m_{2}$.

\section{RC4}
RC4\footnote{RC4 è un cifrario proprietario di cui si è diffusa in rete un'implementazione, ed è pertanto spesso chiamato Alleged-RC4 (ARC4).} è un cifrario di flusso sincrono, semplice ed altamente performante utilizzato per produrre un keystream pseudo-random.

RC4 è stato adottato come generatore di keystream in molti protocolli (WEP, WPA, SSL, TLS) fino a quando è stato provato insicuro dall'\textit{attacco Fluhrer-Mantin-Shamir (FMS)} su WEP.

Il cifrario è costituito da una fase di \textit{key-scheduling (KS)} seguita da una fase di \textit{pseudo-random generation (PRG)}.
Lo stato del cifrario è costituito da una S-box $S$ a 256 bytes parametrizzata dalla chiave segreta $K$, e da una coppia di indici $(i,j)$.

\bigskip
\begin{algorithm}[H]
  \caption{RC4 - Key Scheduling}
  \label{alg:rc4-key-scheduling}
  \SetAlgoNoLine
  \For{i from 0 to 255}{
    S[i] = i
  }
  j = 0\\
  \For{i from 0 to 255} {
    j = (j + S[i] + K[i mod len(K)]) mod 256\\
    swap(S[i],S[j])\\
  }
\end{algorithm}

\bigskip
\begin{algorithm}[H]
  \caption{RC4 - Pseudo-Random Generation}
  \label{alg:rc4-pseudorandom-generation}
  \SetAlgoNoLine
  i = 0\\
  j = 0\\
  \While{generatingOutput} {
    i = (i+1) mod 256\\
    j = (j + S[i]) mod 256\\
    swap(S[i],S[j])\\
    k = S[(S[i]+S[j])mod 256]\\
    \Return{k}
  }
\end{algorithm}

Il cifrario è altamente performante (10x rispetto a DES), efficiente da un punto di vista di memoria, garantisce un buon one-time-padding, ed è robusto ad attacchi di crittanalisi lineare.

Purtroppo, presenta i seguenti difetti:
(i) la dimensione spazio del keystream è limitata a $2^{1700}$ (a prescindere dalla chiave utilizzata)
(ii) i primi bytes del keystream sono fortemente correlati alla chiave e
(iii) il keystream è leggermente polarizzato su alcune sequenze.
