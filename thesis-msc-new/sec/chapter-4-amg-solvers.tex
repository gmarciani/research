% From mitthesis package
% Version: 1.04, 2023/10/19
% Documentation: https://ctan.org/pkg/mitthesis


\chapter{Algebraic Multi-Grid Solvers}

Algebraic Multigrid (AMG) solvers are powerful iterative methods used for solving large, sparse linear systems of equations, typically arising from the discretization of partial differential equations (PDEs).

By leveraging the algebraic properties of the problem, AMG solvers can handle complex geometries and unstructured grids, making them applicable to a wide range of scientific and engineering challenges.

This chapter provides a comprehensive overview of AMG solvers, exploring their theoretical foundation, algorithmic components, implementations, and practical applications in computational science and engineering. 
We will also show how to run an AMG Solver in a HPC environment deployed with AWS ParallelCluster.

\section{Basic Concepts}

TODO: describe AMG Solvers at high level, also explaining the difference, pros and cons w.r.t geometric multi-grid solvers.

\subsection{Coarsening}

TODO: describe the coarsening techniques.

\subsection{Interpolation}

TODO: describe the interpolation techniques.

\subsection{Smoothing}

TODO: describe the smoothing techniques.

\section{Algorithms}

TODO: describe most relevant algorithms with pseudo code and high level description of the approach with pros and cons. Examples: V-Cycle, W-Cycle, F-Cycle.

\section{Implementations}

TODO: describe the most relevant data structures, parallelization approaches and libraries providing AMG solvers. For every library provide a representative usage example with code/commands.

\section{Performance}

TODO: describe convergence rate, computational complexity, scalability.

\section{Applications}

TODO: describe applications in Compitational Fluid Dynamics, Structural Analysis, Electromagnetic Simulations

\section{Running AMG Solvers on a Cluster}

TODO: describe how to run an AMG solver on a pcluster cluster (provide config files and code).