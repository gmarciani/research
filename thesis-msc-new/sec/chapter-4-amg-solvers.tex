% From mitthesis package
% Version: 1.04, 2023/10/19
% Documentation: https://ctan.org/pkg/mitthesis


\chapter{Algebraic Multi-Grid Solvers}

Algebraic Multigrid (AMG) solvers are powerful iterative methods used for solving large, sparse linear systems of equations, typically arising from the discretization of partial differential equations (PDEs).

By leveraging the algebraic properties of the problem, AMG solvers can handle complex geometries and unstructured grids, making them applicable to a wide range of scientific and engineering challenges.

This chapter provides a comprehensive overview of AMG solvers, exploring their theoretical foundation, algorithmic components, implementations, and practical applications in computational science and engineering. 

We will also show how to run an AMG Solver in a HPC environment deployed with AWS ParallelCluster.

\section{Basic Concepts}

TODO

\subsection{Coarsening}

TODO

\subsection{Interpolation}

TODO

\subsection{Smoothing}

TODO

\section{Algorithms}

TODO

\subsection{V-Cycle Algorithm}

TODO

\subsection{W-Cycle Algorithm}

TODO

\subsection{F-Cycle Algorithm}

TODO

\section{Implementations}

TODO

\subsection{Data Structures}

TODO

\subsection{Parallelization Strategies}

TODO

\section{Performance}

TODO: describe convergence rate, computational complexity, scalability.


\section{Applications}

TODO: describe applications in Compitational Fluid Dynamics, Structural Analysis, Electromagnetic Simulations

\subsection{Implementations}

TODO: describe the most famouse libraries èrpviodiong AMG solvers and for every library provide a representative example.

\section{Running AMG Solvers on a Cluster}

TODO: describe how to run an AMG solver on a pcluster cluster (provide config files and code).