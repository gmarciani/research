% From mitthesis package
% Version: 1.04, 2023/10/19
% Documentation: https://ctan.org/pkg/mitthesis


\chapter{High Performance Computing}

High Performance Computing (HPC) refers to the use of supercomputers, tightly coupled networks and parallel processing techniques to solve complex computational problems. 

From its beginning in the 70s, HPC has evolved to embrace a wide variety of applications across scientific research, engineering, finance, genomics, artificial intelligence and more. 
It represents a significant advancement in computational science and plays more and more a strategic role to enable compute intensive tasks, driving innovation across various sectors. 

The field of HPC has traditionally been dominated by on-premises (on-prem) infrastructures with specialized hardware and software stacks; however, Cloud computing is now acting as a transformative force in making HPC more flexible, scalable, accessible and easier to maintain.

This chapter delves into the fundamental concepts of HPC, exploring its architecture, methodologies, applications. We will also dive deep into how Cloud infrastructures, opposed to traditional on-prem, represent a disruptive opportunity for HPC.

\section{Definition, Scope and History}

TODO

\section{Architectures}

TODO

\section{Methodologies}

TODO

\subsection{Parallel Computing}

TODO

\subsection{Performance Optimization}

TODO

\section{Applications}

TODO

\section{New Trends}

TODO

\section{On-Premises Infrastructures}

TODO

\subsection{Basic Components}

TODO

\subsection{Cluster Topologies}

TODO

\subsection{Deployment Strategies}

TODO

\subsection{Advantages and Challeges}

TODO

\section{Cloud Infrastructures}

TODO

\subsection{Deployment Strategies}

TODO

\subsection{Advantages and Challeges}

TODO

\subsection{Services and Tools}

TODO
