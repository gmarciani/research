% From mitthesis package
% Version: 1.04, 2023/10/19
% Documentation: https://ctan.org/pkg/mitthesis


\chapter{AWS ParallelCluster}

AWS ParallelCluster is an open-source cluster management tool that makes it easy to deploy, scale and administer HPC clusters on the Amazon Web Services (AWS) Cloud. 

ParallelCluster abstracts much of the complexity involved in creating HPC environments, providing a user-friendly CLI interface to create, update, describe and delete them, out of a simple YAML configuration file. 
With ParallelCluster users do not need to deal with the low level technicalities of a HPC infrastructure and the operational burden to maintain it. 
Users are immediately up to speed and can quickly tailor their clusters to meet evolving requirements with close to zero downtime, focusing on what really matters for them: executing workloads at scale at high speed within a controllable cost-effective environment.
ParallelCluster has a proven track record in empowering scientists and engineers from leading organizations across various sectors to harness the full potential of the AWS infrastructure for their HPC workloads.

This chapter delves into the user experience, architecture and key features of AWS ParallelCluster, providing a comprehensive overview of its capabilities, applications and alternative solutions. 
We will also show how to deploy and customize a cluster from the ground up with practical step-by-step examples.

\section{Definition, Scope and History}

TODO

\section{User Experience}

TODO

\section{Key Features}

TODO

\section{Architecture}

TODO

\subsection{Computing}

TODO

\subsection{Software Stack}

TODO

\subsection{Storage}

TODO

\subsection{Networking}

TODO

\subsection{Access Management}

TODO

\subsection{Scheduler}

TODO

\subsection{Monitoring}

TODO

\subsection{Security}

TODO

\subsection{Customizations}

TODO

\section{Setting Up a Cluster}

TODO

\subsection{Prerequisites}

TODO

\subsection{Configuration File}

TODO

\subsection{Cluster Deployment}

TODO

\subsection{Cluster Management}

TODO

\subsection{Jobs Submission}

TODO

\section{Setting Up a custom AMI}

TODO

\subsection{Prerequisites}

TODO

\subsection{Configuration File}

TODO

\subsection{AMI Deployment}

TODO

\section{Applications}

TODO

\section{Alternative Solutions}

TODO

