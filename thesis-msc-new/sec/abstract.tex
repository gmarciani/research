% From mitthesis package
% Version: 1.01, 2023/06/19
% Documentation: https://ctan.org/pkg/mitthesis
%
% The abstract environment creates all the required headers and footnote. 
% You only need to add the text of the abstract itself.
%
% Approximately 500 words or less; try not to use formulas or special characters
% If you don't want an initial indentation, do \noindent at the start of the abstract

High Performance Computing (HPC) refers to the use of supercomputers and parallel processing techniques to solve complex computational problems. 

HPC has traditionally relied on specialized on-premise infrastructures characterized by dedicated supercomputers and tightly coupled networks to meet the demands of intensive computational tasks. However, the advent of Cloud technologies and its rapid evolution is now offering a flexible, scalable, and cost-effective alternative for deploying HPC applications.
As such, leading organizations from the academia and various industrial sectors are increasingly replacing or supplementing traditional HPC systems with Cloud-based solutions.

This thesis explores the use of HPC in the Cloud, specifically leveraging AWS ParallelCluster to optimize the execution of HPC workloads. The study primarily uses the OSU Micro-Benchmarks and Algebraic Multi-Grid (AMG) Solvers as representative workloads, conducting experiments to evaluate performance across different cluster configurations.

This research contributes to the broader understanding of Cloud computing's role in the future of HPC and provides a comprehensive set of best practices for practitioners to take the most out of AWS ParallelCluster when running HPC workloads.