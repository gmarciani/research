% From mitthesis package
% Version: 1.01, 2023/06/19
% Documentation: https://ctan.org/pkg/mitthesis
%
% The abstract environment creates all the required headers and footnote. 
% You only need to add the text of the abstract itself.
%
% Approximately 500 words or less; try not to use formulas or special characters
% If you don't want an initial indentation, do \noindent at the start of the abstract

High Performance Computing (HPC) has traditionally relied on specialized, on-premise infrastructure to meet the demands of intensive computational tasks. However, the rapid evolution of cloud technologies offers a flexible, scalable, and cost-effective alternative for deploying HPC workloads. This thesis explores the integration of HPC with cloud computing, focusing on the benefits, challenges, and optimization strategies involved in leveraging cloud environments for high-performance applications. 

Through a comprehensive review of existing literature and practical experiments, this research investigates the performance trade-offs between traditional HPC setups and cloud-based solutions. Key performance metrics such as computation speed, scalability, cost-efficiency, and resource management are analyzed using various cloud service models (IaaS, PaaS, SaaS) and providers (AWS, Azure, Google Cloud). Additionally, the study evaluates the impact of virtualization, containerization, and orchestration technologies on HPC performance in the cloud.

The findings demonstrate that while cloud-based HPC can achieve comparable performance to traditional systems, certain considerations such as network latency, data transfer costs, and workload optimization are critical for maximizing efficiency. The thesis concludes with a set of best practices and recommendations for organizations considering the transition to cloud-based HPC, emphasizing the importance of workload profiling, hybrid cloud strategies, and continuous performance monitoring.

This research contributes to the broader understanding of cloud computing's role in the future of HPC, providing a framework for further studies and practical guidelines for industry adoption.
