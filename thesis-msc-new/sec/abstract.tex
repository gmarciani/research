% From mitthesis package
% Version: 1.01, 2023/06/19
% Documentation: https://ctan.org/pkg/mitthesis
%
% The abstract environment creates all the required headers and footnote. 
% You only need to add the text of the abstract itself.
%
% Approximately 500 words or less; try not to use formulas or special characters
% If you don't want an initial indentation, do \noindent at the start of the abstract

% Version 1
High Performance Computing (HPC) has traditionally relied on specialized, on-premise infrastructure to meet the demands of intensive computational tasks. However, the rapid evolution of cloud technologies offers a flexible, scalable, and cost-effective alternative for deploying HPC workloads. This thesis explores the integration of HPC with cloud computing, focusing on the benefits, challenges, and optimization strategies involved in leveraging cloud environments for high-performance applications. 

Through a comprehensive review of existing literature and practical experiments, this research investigates the performance trade-offs between traditional HPC setups and cloud-based solutions. Key performance metrics such as computation speed, scalability, cost-efficiency, and resource management are analyzed using various cloud service models (IaaS, PaaS, SaaS) and providers (AWS, Azure, Google Cloud). Additionally, the study evaluates the impact of virtualization, containerization, and orchestration technologies on HPC performance in the cloud.

The findings demonstrate that while cloud-based HPC can achieve comparable performance to traditional systems, certain considerations such as network latency, data transfer costs, and workload optimization are critical for maximizing efficiency. The thesis concludes with a set of best practices and recommendations for organizations considering the transition to cloud-based HPC, emphasizing the importance of workload profiling, hybrid cloud strategies, and continuous performance monitoring.

This research contributes to the broader understanding of cloud computing's role in the future of HPC, providing a framework for further studies and practical guidelines for industry adoption.

% Version 2
The advent of cloud computing has revolutionized various fields, offering scalable, on-demand, and cost-effective computational resources. This thesis explores the integration and optimization of High Performance Computing (HPC) within cloud environments, focusing on the challenges and opportunities this paradigm presents. Traditional HPC systems, characterized by dedicated supercomputers and tightly coupled networks, are increasingly being supplemented or replaced by cloud-based solutions that promise enhanced flexibility and accessibility.

The research investigates key aspects of deploying HPC workloads in the cloud, including resource allocation, performance optimization, and cost management. Through a series of experimental studies and performance benchmarks, this work evaluates the efficacy of various cloud service providers in handling intensive computational tasks. Additionally, it examines the impact of virtualization, containerization, and orchestration tools on the efficiency and scalability of HPC applications in the cloud.

Key findings highlight the trade-offs between performance and cost, the benefits of hybrid cloud architectures, and the role of emerging technologies such as serverless computing and AI-driven resource management. The thesis also provides a comprehensive set of best practices and recommendations for practitioners aiming to leverage cloud HPC for scientific research, big data analytics, and enterprise applications.

Ultimately, this work contributes to the field by demonstrating that while cloud computing presents viable solutions for HPC, careful consideration of workload characteristics, cloud infrastructure capabilities, and cost implications is essential for maximizing performance and achieving optimal outcomes.

% Version 3
The advent of cloud computing has revolutionized the landscape of high performance computing (HPC), offering unprecedented scalability, flexibility, and cost-efficiency. This thesis explores the integration of HPC within cloud environments, addressing both the potential and challenges associated with this convergence. The research delves into architectural frameworks and deployment strategies that optimize performance, resource utilization, and cost management in cloud-based HPC systems. Through a combination of theoretical analysis and practical experimentation, this work evaluates the performance trade-offs between traditional on-premises HPC infrastructures and cloud-based solutions.

Key contributions include the development of a novel scheduling algorithm that dynamically allocates resources based on real-time workload characteristics, significantly enhancing computational efficiency and reducing latency. Additionally, the thesis presents a comprehensive benchmarking suite tailored for cloud HPC, enabling precise performance assessments across diverse cloud platforms. Case studies spanning scientific simulations, big data analytics, and machine learning workloads are employed to validate the proposed methodologies.

The findings indicate that, while cloud-based HPC can achieve performance parity with traditional systems for many applications, optimal results require careful consideration of factors such as network bandwidth, virtualization overhead, and instance heterogeneity. This research provides a roadmap for leveraging cloud resources to meet the demanding computational requirements of modern scientific and engineering applications, highlighting best practices and identifying future research directions in the field of high performance computing in the cloud.

**Keywords:** high performance computing, cloud computing, resource allocation, scheduling algorithm, benchmarking, computational efficiency, virtualization.

% Version 4
This thesis explores the use of High Performance Computing (HPC) in the cloud, specifically leveraging AWS ParallelCluster to optimize various HPC workloads. 

The research focuses on developing a comprehensive set of best practices to maximize performance, efficiency, and cost-effectiveness of AWS ParallelCluster when running computationally intensive applications. 

The study primarily uses the OSU Micro-Benchmarks and Algebraic Multi-Grid (AMG) Solvers as representative workloads, conducting a series of experiments to evaluate performance across different configurations. 

The outcome is a detailed guide that can serve as a valuable resource for researchers and engineers looking to deploy HPC workloads on AWS.