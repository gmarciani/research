% !TeX document-id = {f8e6f84b-6dd6-4ae8-a59e-b9f398dae83e}
% !TEX encoding = UTF-8 Unicode
% !BIB TS-program = biber 
% !BIB program = biber    

% This file is MIT-Thesis.tex, a LaTeX template for formatting an MIT thesis with the mitthesis class.
%
% Version: 1.11, 2023/11/02
%
% Author: John H. Lienhard, copyright 2023. Reuse under the MIT license: https://ctan.org/license/mit 

% Documentation is here: https://ctan.org/pkg/mitthesis

%% Don't modify the \DocumentMetadata command unless you know what it does. 
%% If this command throws an "undefined" error, your latex system is out of date: try commenting this command out.
\DocumentMetadata{ 
	pdfstandard = a-2b,
	pdfversion  = 1.7,
	lang		= en-US,
%	debug		= {xmp-export}, % uncomment to output a separate xmpi file showing the metadata
}
%%%%%%%%%%%%%%%%%%%%%%%%%%%%%%%%%%%%%%%

\documentclass[twoside,mydesign]{mitthesis} %,fontset=libertine, fontset=newtx-sans-text, fontset=heros-stix2, fontset=stix2
%
% option [twoside]		gives facing-page behavior for printing; omitting twoside will eliminate even-numbered blank pages.
% option [lineno]	 	provides line numbers, as for editing
% option [mydesign] 	loads packages for color, title and list formats, margins, or captions: edit mydesign.tex to change defaults.
% option [fontset] is a keyvalue which can be:
%					 	pdftex or unicode engines:  defaultfonts, libertine, lucida
%					 	pdftex only: 				fira-newtxsf, newtx, newtx-sans-text
%						unicode engines (luatex):	heros-stix2, stix2, termes, termes-stix2
%					 	if no key value is given, fonts default to CMR (pdftex) or LMR (unicode), i.e., "the LaTeX font".
%					 	You can edit the fontset files or you can write your own, myfonts.tex, and do [fontset=myfonts].
%						If you are using multiple languages, load the babel package in your fontset file, before the fonts.

%%%%%%%%% Packages used in sample chapters (not otherwise required) %%%%%%%

%% Package for code listing in Appendix A.
\usepackage{listings}%   documentation is here https://ctan.org/pkg/listings
\usepackage{xcolor}
%\usepackage{minted}
%\usepackage{color}

%% Set chemical formulas nicely
\usepackage[version=4]{mhchem}%   documentation is here https://ctan.org/pkg/mhchem

%% Latin filler used in Chapter 1, with a test for package version date. https://ctan.org/pkg/lipsum
\usepackage{lipsum}
\IfPackageAtLeastTF{lipsum}{2021/09/20}{\setlipsum{auto-lang=false}}{}


%%%%%%%%%  Graphics path (to figure files)  %%%%%%%%%%%%%%%%%%%%%%%%%%%%%%%%

%% Can set graphicspath to point to specific directories containing figures (the current directory is searched automatically)
%% For instance, to search a subdirectory of the current directory called "figures" and a parallel directory called "art", set:

% \graphicspath{ {figures/} {../art/} }% For details see: https://latexref.xyz/dev/latex2e.html#g_t_005cgraphicspath


%%%%%%%%%  Representative set-up for biblatex  %%%%%%%%%%%%%%%%%%%%%%%%%%%%%

\usepackage[style=ieee,maxbibnames=10,sorting=none]{biblatex}% style=ext-numeric-comp,articlein=false,giveninits=true
	\DefineBibliographyStrings{english}{url= \textsc{url} ,  }% replaces default "[Online]. Available" by "URL"


\addbibresource{bib/main.bib}%% <== change to YOUR bib file <= CHANGE

%% to avoid split urls and stretched white space, you can set the bibliography ragged-right:
%\appto{\bibsetup}{\raggedright}

% biblatex is very powerful, and you can customize most aspects the reference list and citations to suit your needs.
% documentation is here: https://ctan.org/pkg/biblatex


%%%%%%%%%%  Option to use natbib   %%%%%%%%%%%%%%%%%%%%%%%%%%%%%%%%%%%%%%%%%

%\RequirePackage[numbers,sort&compress]{natbib}
 
%%% add bibliography to table of contents
%\apptocmd{\bibliography}{\addcontentsline{toc}{chapter}{\protect\textbf{\bibname}}}{}{}

%%% You can use this to rename the bibliography section
%\renewcommand{\bibname}{References}

%%% Can adjust space between bibliography items (change 4pt to something else; don't drop last two lengths, they are stretchable "glue")
%\setlength\bibsep{4pt plus 1pt minus 1pt}


%%%%%%%%%%  Table related packages  %%%%%%%%%%%%%%%%%%%%%%%%%%%%%%%%%%%%%%%%

\usepackage{booktabs}% better quality tables, https://ctan.org/pkg/booktabs
\usepackage{array}%    additional options for table columns, https://ctan.org/pkg/array

%\usepackage{tabularx}%   https://ctan.org/pkg/tabularx

%\usepackage{dcolumn}%    alignment on decimal place, https://ctan.org/pkg/dcolumn
%\newcolumntype{d}[1]{D{.}{.}{#1}}


%%%%%%%%%%  Option for "double spacing" %%%%%%%%%%%%%%%%%%%%%%%%%%%%%%%%%%%%

%% Back in the typewriter era, double spaced lines were convenient for editing with a pencil. 
%% In typography, the separation between lines is called "leading", and it is usually set in 
%% proportion to the font size (i.e., when the font is loaded).  If you really feel the need 
%% to change the line separation, the most attractive results will be obtained by changing the
%% leading in proportion to the the current font size, rather than just doubling the space.

%% The setspace package provides a tool for changing line separation. Use these two commands here:
%
% \usepackage{setspace}%  documentation at https://ctan.org/pkg/setspace
% \setstretch{1.1}% you can choose some other value for the stretch of space between lines
%
%% Use one or more of the these commands AFTER the frontmatter
%
% \onehalfspacing
% \doublespacing
% \singlespacing  % will turn these effects off (you can use these anywhere in the document)

%% The best result may be to stay with leading selected by the typographer who set up the font.


%%%%%%%%%%%  Metadata  %%%%%%%%%%%%%%%%%%%%%%%%%%%%%%%%%%%%%%%%%%%%%%%%%%%%%%%

% Most of the document metadata is created automatically. 
% The following items should be adjusted to match your work. <================= !!!!!!!!!!

\hypersetup{%
	pdfsubject={Template for writing MIT theses with the mitthesis class},
	% Change this to briefly state topic of your thesis 
% 
	pdfkeywords={Massachusetts Institute of Technology, MIT},
	% Add keywords that will help search engines and libraries to find your work.
	% Includes the name[s] of the author[s] 
	% (If you have used \DocumentMetadata, at line 15, you can just put "\CopyrightAuthor," for the names.)
%
	pdfurl={},
	% If you have a url for the thesis, put it here. Otherwise delete this.
	% (MIT Libraries will put your thesis in DSPACE with a persistent url after you submit it.)
%	
	pdfcontactemail={},
	% You can put a [permanent] email address into the metadata, if you like.
	% Otherwise delete this.
%
	pdfauthortitle={},
	% If you have a title, you can include it here.
}

\Institution{University of Rome Tor Vergata}

%%%%%%%%%%%%%%  End preamble %%%%%%%%%%%%%%%%%%%%%%%%%%%%%%%%%%%%%%%%%%%%%%%%%%%%%%%%%%%%%%%%%%%%%
%%%%%%%%%%%%%%%%%%%%%%%%%%%%%%%%%%%%%%%%%%%%%%%%%%%%%%%%%%%%%%%%%%%%%%%%%%%%%%%%%%%%%%%%%%%%%%%%%%

\begin{document}

%%% edit the following commands to match your thesis %%%%%%%%%%

\title{High Performance Computing on AWS:\\Lifecycle and Best Practices}

% \Author{Author full name}{Author department}[Author's first PREVIOUS degree][Author's second PREVIOUS degree][...
% Note that third, fourth, fifth, and sixth arguments are optional [] and may be omitted

% note on names: most of the following names are made up; Silas Holman was a physics professor at MIT in the 19th century.

\Author{Giacomo Marciani}{B.S. Computer Science Engineering}
% \Author{Luisa Hernández}{Department of Research}[B.S. Mechanical Engineering, UCLA, 2018][M.S. Stellar Interiors, Vulcan Science Academy, 2020]
% \Author{Thurston Howell III}{Department of Economics}[MBA, Ferengi School of Management, 2022]

% Use once for each degree fulfilled by thesis
% For two degrees from one department, leave the department argument blank for the second degree {}.
% \Degree{Bachelor of Science in Physics}{Department of Physics}
% \Degree{Master of Science in Physics}{}
\Degree{Master of Science in Computer Science Engineering and Data Science}{Department of Civil and Computer Science Engineering}

% If there is more than one supervisor, use the \Supervisor command for each.
\Supervisor{Salvatore Filippone}{Professor of High Performance Computing}
% \Supervisor{Secunda Castor}{Professor of Research}
% \Supervisor{Quintus Castor}{Professor of Log Dams}

% Professor who formally accepts theses for your department (e.g., the Graduate Officer, Professor Sméagol,...)
% If more than one department, use more than once
% **If you need to reduce vertical space, put the acceptor title in the second argument and leave the third blank {}.**
%\Acceptor{Primus Castor}{Professor of Wetlands Engineering}{Undergraduate Officer, Department of Physics}
\Acceptor{Renato Baciocchi}{Director}{Department of Civil and Computer Science Engineering}
% \Acceptor{Quarta Castor}{Professor of Lodge Building}{Graduate Officer, Department of Mechanical Engineering}

% Usage: \DegreeDate{Month}{year}
% Valid degree months are September, February, or June
\DegreeDate{February}{2025}

% Date that final thesis is submitted to department
\ThesisDate{October 27, 2024}

%%%%%%  Choose whether to have a CREATIVE COMMONS License  %%%%%%%%%%%%%%%%%%%%%%%%%%%%%%%%%%%%%%
%
% If you are using a cc license, put details of your cc license here. 
% Omit this command if you are not using a cc license.
%
\CClicense{CC BY-NC-ND 4.0}{https://creativecommons.org/licenses/by-nc-nd/4.0/}
%

%%%%%%%  Solutions for overflowing titlepage  %%%%%%%%%%%%%%%%%%%%%%%%%%%%%%%%%%%%%%%%%%%%%%%%%%%

% If your title page is overflowing (from too many names, degrees, etc.):
%
% (a) you can reduce the 12pt and 18pt skips between various blocks to 6pt with this command:
%
% \Tighten
%
% (b)  you can scale down the Signature block at the bottom with this command:
%
% \SignatureBlockSize{\small}  %or this one \SignatureBlockSize{\footnotesize}
%
% (c) you can put the acceptor name and title onto two lines, rather than three like this:
%
% \Acceptor{Tertius Castor}{Professor and Graduate Officer, Department of Research}{}
% \Acceptor{Quarta Castor}{Professor and Graduate Officer, Department of Mechanical Engineering}{}
%
% (d) you can change the font size of the the author name[s] with
%
%	\AuthorNameSize{\normalsize}
%
% (e) and you can omit any previous degrees from the title page, instead mentioning them in the Biosketch

% Also, if you prefer to keep the text toward the top of the page with most white space at the bottom, you
% can you this command to squash all of the vertical glue (stretchy space) with this command:
%
% \Squash 
%
% This command is useful when the text has not already reach the bottom of the page, since the glue gets squashed automatically
% when the page is too full.

%%%%%%%%%%%%%%%%%%%%%%%%%%%%%%%%%%%%%%%%%%%%%%%%%%%%%%%%%%%%%%%%%%%%%%%%%%%%%%%%%%%%%%%%%%%%%%%%%

%%% Make titlepage
\maketitle*

%%%%%%%%% Contents that you need to write follows %%%%%%%%%%%%%%%%%%%%%%%%%%%%%%%%%%%%%%%%%%%%%%%%

% \includeonly{acknowledgments,biography,chapter1,chapter2,...,appendixa,...} 
%   for usage, see https://latexref.xyz/_005cinclude-_0026-_005cincludeonly.html

%%% Frontmatter (write this material in the mentioned files)  %%%%%%%%%%%%%%%%%%%%%%%%%%%%%%%%%%%%

% The abstract environment creates all the required headings and footers. 
% You only need to the text of the abstract in the file abstract.tex
\begin{abstract}
	% From mitthesis package
% Version: 1.01, 2023/06/19
% Documentation: https://ctan.org/pkg/mitthesis
%
% The abstract environment creates all the required headers and footnote. 
% You only need to add the text of the abstract itself.
%
% Approximately 500 words or less; try not to use formulas or special characters
% If you don't want an initial indentation, do \noindent at the start of the abstract

The developments of the ``kinetic theory'' of gases made within the last ten years have enabled it to account satisfactorily for many of the laws of gases. The mathematical deductions of Clausius, Maxwell and others, based upon the hypothesis of a gas composed of molecules acting upon each other at impact like perfectly elastic spheres, have furnished expressions for the laws of its elasticity, viscosity, conductivity for heat, diffusive power and other properties. For some of these laws we have experimental data of value in testing the validity of these deductions and assumptions. Next to the elasticity, perhaps the phenomena of the viscosity of gases are best adapted to investigation.\footnote{Text from Holman (1876): \doi{10.2307/25138434}.}  
% use \input rather than \include because we're inside an environment
\end{abstract}

%% acknowledgments.tex

% From mitthesis package
% Version: 1.01, 2023/10/16
% Documentation: https://ctan.org/pkg/mitthesis


\chapter*{Acknowledgments}
\addcontentsline{toc}{chapter}{Acknowledgments}

I would like to express my deepest gratitude to all those who have supported me throughout this research and my overall course of study.

First and foremost, I would like to thank with all my heart my beloved wife and my parents for always motivating me to conclude my thesis and believing in me in good and bad times.

I thank my sons Edoardo, Arianna and Vittorio because they fill my life with love and their birth has been the strongest motivation to complete my studies.

I thank my grandfather for passing on my passion for computer science to me since I was a child.

A special thanks goes to my friends and colleagues, Antonella, Debora, Gianluca, Giorgio and Michele: we shared joys and sorrows during my university journey and they motivated me to complete my thesis every time we met.

I'd like to thank my advisor, Prof. Salvatore Filippone, whose expertise, encouragement, and insightful feedback were invaluable in shaping this thesis.

I thank Prof. Valeria Cardellini for passing on to me the passion for distributed systems, that shaped my entire career.

Last but not least, I thank my amazing company Amazon Web Services for betting on me even before I finished my studies.
% .tex extension is presumed by \include 

%%% biography.tex
%% This section is optional

% From mitthesis package
% Version: 1.01, 2023/10/16
% Documentation: https://ctan.org/pkg/mitthesis

\chapter*{Biographical Sketch}
\addcontentsline{toc}{chapter}{Biographical Sketch}

Silas Whitcomb Holman was born in Harvard, Massachusetts on January 20, 1856. He received his S.B. degree in Physics from MIT in 1876, and then joined the MIT Department of Physics as an Assistant. He became Instructor in Physics in 1880, Assistant Professor in 1882, Associate Professor in 1885, and Full Professor in 1893. Throughout this period, he struggled with increasingly severe rheumatoid arthritis. At length, he was defeated, becoming Professor Emeritus in 1897 and dying on April 1, 1900.

Holman's light burned brilliantly before his tragic and untimely death. He published extensively in thermal physics, and authored textbooks on precision measurement, fundamental mechanics, and other subjects. He established the original Heat Measurements Laboratory. Holman was a much admired teacher among both his students and his colleagues. The reports of his department and of the Institute itself refer to him frequently in the 1880's and 1890's, in tones that gradually shift from the greatest respect to the deepest sympathy.

Holman was a student of Professor Edward C. Pickering, then head of the Physics department. Holman himself became second in command of Physics, under Professor Charles R. Cross, some years later. Among Holman's students, several went on to distinguish themselves, including: the astronomer George E. Hale ('90) who organized the Yerkes and Mt. Wilson observatories and who designed the 200 inch telescope on Mt. Palomar; Charles G. Abbot ('94), also an astrophysicist and later Secretary of the Smithsonian Institution; and George K. Burgess ('96), later Director of the Bureau of Standards. % optional, see MIT Libraries https://libraries.mit.edu/distinctive-collections/thesis-specs/#format


%%% Table of contents and lists of stuff (delete lists you don't need, e.g., if no tables) %%%%%%%%

\tableofcontents
%\listoffigures
%\listoftables


%%% Chapters of thesis  %%%%%%%%%%%%%%%%%%%%%%%%%%%%%%%%%%%%%%%%%%%%%%%%%%%%%%%%%%%%%%%%%%%%%%%%%%%

%% If you want to use "double spacing", you should start here...

 % From mitthesis package
% Version: 1.04, 2023/10/19
% Documentation: https://ctan.org/pkg/mitthesis


\chapter{High Performance Computing}

High Performance Computing (HPC) refers to the use of supercomputers, tightly coupled networks and parallel processing techniques to solve complex computational problems. 

From its beginning in the 70s, HPC has evolved to embrace a wide variety of applications across scientific research, engineering, finance, genomics, artificial intelligence and more. 
It represents a significant advancement in computational science and plays more and more a strategic role to enable compute intensive tasks, driving innovation across various sectors. 

The field of HPC has traditionally been dominated by on-premises (on-prem) infrastructures with specialized hardware and software stacks; however, Cloud computing is now acting as a transformative force in making HPC more flexible, scalable, accessible and easier to maintain.

This chapter delves into the fundamental concepts of HPC, exploring its architecture, methodologies, applications. We will also dive deep into how Cloud infrastructures, opposed to traditional on-prem, represent a disruptive opportunity for HPC.

\section{Definition, Scope and History}

TODO

\section{Architectures}

TODO

\section{Methodologies}

TODO

\subsection{Parallel Computing}

TODO

\subsection{Performance Optimization}

TODO

\section{Applications}

TODO

\section{New Trends}

TODO

\section{On-Premises Infrastructures}

TODO

\subsection{Basic Components}

TODO

\subsection{Cluster Topologies}

TODO

\subsection{Deployment Strategies}

TODO

\subsection{Advantages and Challeges}

TODO

\section{Cloud Infrastructures}

TODO

\subsection{Deployment Strategies}

TODO

\subsection{Advantages and Challeges}

TODO

\subsection{Services and Tools}

TODO

 % From mitthesis package
% Version: 1.04, 2023/10/19
% Documentation: https://ctan.org/pkg/mitthesis


\chapter{AWS ParallelCluster}

AWS ParallelCluster is an open-source cluster management tool that makes it easy to deploy, scale and administer HPC clusters on the Amazon Web Services (AWS) Cloud. 

ParallelCluster abstracts much of the complexity involved in creating HPC environments, providing a user-friendly CLI interface to create, update, describe and delete them, out of a simple YAML configuration file. 
With ParallelCluster users do not need to deal with the low level technicalities of a HPC infrastructure and the operational burden to maintain it. 
Users are immediately up to speed and can quickly tailor their clusters to meet evolving requirements with close to zero downtime, focusing on what really matters for them: executing workloads at scale at high speed within a controllable cost-effective environment.
ParallelCluster has a proven track record in empowering scientists and engineers from leading organizations across various sectors to harness the full potential of the AWS infrastructure for their HPC workloads.

This chapter delves into the user experience, architecture and key features of AWS ParallelCluster, providing a comprehensive overview of its capabilities, applications and alternative solutions. 
We will also show how to deploy and customize a cluster from the ground up with practical step-by-step examples.

\section{Definition, Scope and History}

TODO

\section{User Experience}

TODO

\section{Key Features}

TODO

\section{Architecture}

TODO

\subsection{Computing}

TODO

\subsection{Software Stack}

TODO

\subsection{Storage}

TODO

\subsection{Networking}

TODO

\subsection{Access Management}

TODO

\subsection{Scheduler}

TODO

\subsection{Monitoring}

TODO

\subsection{Security}

TODO

\subsection{Customizations}

TODO

\section{Setting Up a Cluster}

TODO

\subsection{Prerequisites}

TODO

\subsection{Configuration File}

TODO

\subsection{Cluster Deployment}

TODO

\subsection{Cluster Management}

TODO

\subsection{Jobs Submission}

TODO

\section{Setting Up a custom AMI}

TODO

\subsection{Prerequisites}

TODO

\subsection{Configuration File}

TODO

\subsection{AMI Deployment}

TODO

\section{Applications}

TODO

\section{Alternative Solutions}

TODO


 % From mitthesis package
% Version: 1.04, 2023/10/19
% Documentation: https://ctan.org/pkg/mitthesis


\chapter{OSU Benchmarks}

\section{Introduction}

The OSU Micro-Benchmarks (OMB) suite is a set of benchmarks developed by the Ohio State University to evaluate the performance of MPI (Message Passing Interface) libraries. 

These benchmarks are widely used in the HPC community to measure point-to-point and collective communication performance, as well as MPI synchronization mechanisms.

In this chapter, we provide an overview of the OSU Benchmarks, their relevance in the context of HPC, and how they can be used to assess the performance of different configurations within AWS ParallelCluster. 

We will also discuss the methodology used for running these benchmarks and interpreting the results.

\section{Alternative Benchmarks for HPC}
While the OSU Micro-Benchmarks are widely used for evaluating MPI performance, there are several other benchmark suites that can be used to assess various aspects of HPC systems. This section will provide a list of alternative benchmarks, including:

1. HPCG (High Performance Conjugate Gradients): A benchmark that complements the High Performance Linpack (HPL) benchmark by focusing on memory bandwidth and communication rather than raw computational power.
2. SPEC MPI: A benchmark suite designed to evaluate the performance of parallel systems running MPI applications, offering a diverse set of workloads.
3. STREAM: A simple yet powerful benchmark that measures memory bandwidth, crucial for understanding the performance of memory-intensive HPC applications.
4. LINPACK: The classic benchmark used to rank supercomputers in the Top500 list, focusing on solving dense linear systems.

This section will provide a brief overview of each benchmark, discussing their relevance and applicability to different HPC scenarios.
 % From mitthesis package
% Version: 1.04, 2023/10/19
% Documentation: https://ctan.org/pkg/mitthesis


\chapter{Algebraic Multi-Grid Solvers}

% Version 1
Algebraic Multigrid (AMG) solvers are powerful iterative methods used for solving large, sparse linear systems of equations, typically arising from the discretization of partial differential equations (PDEs). Unlike geometric multigrid methods, which rely on the geometric structure of the problem, AMG solvers operate purely on the algebraic properties of the matrix, making them highly versatile for a wide range of applications. This chapter provides an in-depth exploration of AMG solvers, including their theoretical foundation, algorithmic components, implementation strategies, and practical applications in computational science and engineering.

Algebraic Multigrid solvers offer a versatile and powerful approach for solving large, sparse linear systems of equations. By leveraging the algebraic properties of the problem, AMG solvers can handle complex geometries and unstructured grids, making them applicable to a wide range of scientific and engineering problems. Understanding the theoretical foundation, algorithmic components, and implementation strategies of AMG solvers is essential for their effective use in high-performance computing environments. This chapter has provided a comprehensive overview of AMG solvers, highlighting their significance and practical applications in computational science and engineering.

% Version 2
Algebraic Multi-Grid (AMG) solvers are iterative methods used to solve large sparse linear systems, which arise frequently in scientific computing and engineering applications. 

AMG solvers are known for their efficiency and scalability, making them ideal candidates for HPC environments.

This chapter focuses on the principles and applications of AMG solvers, exploring their implementation within HPC frameworks. 

We will also examine the specific challenges and considerations when running AMG solvers in a cloud-based HPC environment using AWS ParallelCluster.

\section{Basic Concepts}

Multigrid methods are designed to accelerate the convergence of iterative solvers by addressing the problem across multiple scales. 

They decompose the problem into coarser and coarser grids, solving on these grids iteratively and correcting errors on finer grids. 

The basic components of a multigrid cycle include:

- **Smoothing**: Reducing high-frequency errors on a given grid.
- **Restriction**: Transferring the residual to a coarser grid.
- **Coarse Grid Correction**: Solving the coarser grid problem approximately.
- **Prolongation**: Interpolating the correction back to the finer grid.

\subsection{Algebraic Multigrid Concept}

AMG extends the multigrid concept by constructing the multigrid hierarchy based on the matrix's algebraic properties rather than the underlying physical geometry.

This is particularly useful for unstructured grids or problems with complex geometries.

1. **Coarsening**: AMG creates a sequence of increasingly coarser matrices that approximate the original fine-grid matrix.
2. **Interpolation (Prolongation)**: Defines how solutions on the coarser grids are interpolated back to the finer grids.
3. **Smoothing**: Uses relaxation methods to damp high-frequency errors on each level.

\section{Components of AMG Solvers}

\subsection{Coarsening Strategies}

Coarsening is the process of selecting a subset of variables to represent the problem at a coarser level. Common coarsening strategies include:

- **Classical AMG Coarsening**: Nodes are divided into coarse and fine sets based on their connectivity and influence.
- **Aggregation-Based Coarsening**: Groups of nodes (aggregates) are treated as single units in the coarser grid.

\subsection{Interpolation Operators}

Interpolation operators define how the coarse grid corrections are interpolated to the finer grid. Key types include:

- **Direct Interpolation**: Uses direct connections between fine and coarse nodes.
- **Smoothed Aggregation**: Combines aggregation with a smoothing step to improve interpolation quality.

\subsection{Smoothing Techniques}

Smoothing techniques reduce high-frequency errors on each grid level. Common smoothers include:

- **Jacobi Smoothing**: Simple and easy to implement but may require many iterations.
- **Gauss-Seidel Smoothing**: More efficient than Jacobi but can be harder to parallelize.
- **Symmetric Gauss-Seidel**: A combination of forward and backward Gauss-Seidel sweeps.

\section{AMG Algorithms}

\subsection{The V-Cycle Algorithm}

The V-cycle is the most commonly used multigrid cycle. It involves:

1. **Pre-Smoothing**: Applying a few iterations of a smoother on the fine grid.
2. **Restriction**: Transferring the residual to the coarse grid.
3. **Coarse Grid Correction**: Solving the coarse grid problem approximately.
4. **Prolongation**: Interpolating the coarse grid correction back to the fine grid.
5. **Post-Smoothing**: Applying additional smoothing iterations on the fine grid.

\subsection{The W-Cycle and F-Cycle Algorithms}

- **W-Cycle**: Similar to the V-cycle but performs additional coarse grid corrections, leading to more robust but computationally expensive solutions.
- **F-Cycle**: A hybrid approach combining features of both V-cycle and W-cycle, providing a balance between robustness and efficiency.

\section{Implementation of AMG Solvers}

\subsection{Data Structures}

Efficient implementation of AMG solvers requires appropriate data structures to store sparse matrices, such as:

- **Compressed Sparse Row (CSR)**: Efficient for matrix-vector multiplication.
- **Coordinate List (COO)**: Flexible and simple, often used for constructing matrices.

\subsection{Parallelization Strategies}

AMG solvers can benefit from parallelization to handle large-scale problems. Strategies include:

- **Domain Decomposition**: Dividing the problem into subdomains and solving them in parallel.
- **Parallel Smoothing**: Implementing parallel versions of smoothing algorithms like Jacobi or Gauss-Seidel.
- **Load Balancing**: Ensuring even distribution of computational workload across processors.

\section{Performance Considerations}

\subsection{Convergence Rate}

The convergence rate of AMG solvers depends on factors such as the quality of the coarsening strategy, the effectiveness of interpolation, and the choice of smoother. Balancing these components is crucial for optimal performance.

\subsection{Computational Complexity}

AMG solvers aim to achieve linear computational complexity with respect to the number of unknowns. Efficient implementation and parallelization are key to maintaining this complexity in practice.

\subsection{Scalability}

Scalability is a critical consideration for AMG solvers, especially for large-scale problems. Implementing parallel AMG solvers and optimizing communication overhead are essential for achieving good scalability.

\section{Applications of AMG Solvers}

Algebraic Multi-Grid (AMG) solvers are used in a variety of applications that require the solution of large sparse linear systems. 

This section will explore the primary use cases for AMG solvers, including fluid dynamics, structural mechanics, electromagnetics, and reservoir simulation. 

These use cases often involve complex simulations where AMG solvers can significantly reduce computation time, making them a critical component in many scientific and engineering workflows.

\subsection{Computational Fluid Dynamics (CFD)}

AMG solvers are widely used in CFD for solving discretized Navier-Stokes equations, enabling efficient simulations of fluid flow and heat transfer.

\subsection{Structural Analysis}

In structural analysis, AMG solvers facilitate the solution of large, sparse systems arising from finite element discretizations, aiding in the analysis of stress, strain, and deformation in structures.

\subsection{Electromagnetic Simulations}

AMG solvers are employed in the simulation of electromagnetic fields, particularly in solving Maxwell's equations for complex geometries and materials.

\subsection{Implementations of AMG Solvers}
Several implementations of AMG solvers exist, each tailored to different types of problems and computational environments. This section will provide a list of popular AMG solver implementations, including:

1. Hypre: A library of high-performance preconditioners and solvers, widely used in parallel computing environments.
2. ML (Multi-level Preconditioner Package): Part of the Trilinos project, designed for solving large linear systems on parallel architectures.
3. SAMG (Algebraic Multigrid Solver by Fraunhofer SCAI): A commercial solver known for its efficiency in solving large-scale industrial problems.
4. AMGCL: A header-only C++ library for parallel algebraic multigrid methods, optimized for modern hardware.

We will also discuss the pros and cons of each implementation, as well as their suitability for different types of HPC workloads.

\section{Running AMG Solvers on a Cluster}
This section will provide a practical guide on how to run AMG solvers on an HPC cluster, focusing on the integration with AWS ParallelCluster. 

We will cover the necessary steps for setting up the environment, compiling the solver, and running it across multiple nodes. 

Additionally, we will discuss performance tuning strategies, such as optimizing communication patterns and load balancing, to ensure that the AMG solvers run efficiently in a distributed computing environment.
 % From mitthesis package
% Version: 1.04, 2023/10/19
% Documentation: https://ctan.org/pkg/mitthesis


\chapter{Experiments}

In this chapter, we detail the experiments conducted to evaluate the performance of AWS ParallelCluster for different HPC workloads, specifically focusing on the OSU Benchmarks and AMG Solvers.

For very experiment, we outline the environment setup, including cluster configurations, benchmarks setup and of course, the interpretation of the results.

The results of these experiments are presented, analyzed, and compared across various scenarios.

\section{OSU Benchmarks}

TODO: Add the results and interpretation of experiments running different OSU benchmarks on different clusters.

\section{AMG Solvers}

TODO: Add the results and interpretation of experiments running different AMG solvers on different clusters.

 % From mitthesis package
% Version: 1.04, 2023/10/19
% Documentation: https://ctan.org/pkg/mitthesis


\chapter{Best Practices}

\section{Introduction}

Based on the findings from the experiments, this chapter presents a set of best practices for optimizing AWS ParallelCluster for different HPC workloads. 

These best practices cover areas such as instance selection, network configuration, storage optimization, and cost management. 

The recommendations are designed to help users achieve the best possible performance and efficiency when deploying HPC workloads on AWS.
 % From mitthesis package
% Version: 1.04, 2023/10/19
% Documentation: https://ctan.org/pkg/mitthesis


\chapter{Conclusions}

\section{Introduction}

The concluding chapter summarizes the key findings of the thesis, reflecting on the implications of cloud-based HPC for the future of scientific computing. 

We also discuss potential areas for further research and development, particularly in the context of emerging technologies and evolving HPC workloads.



%%% Appendicies of thesis  %%%%%%%%%%%%%%%%%%%%%%%%%%%%%%%%%%%%%%%%%%%%%%%%%%%%%%%%%%%%%%%%%%%%%%%%

\appendix
% From mitthesis package
% Version: 1.01, 2023/07/04
% Documentation: https://ctan.org/pkg/mitthesis


\chapter{Code listing}

\lstdefinestyle{mystyle}{
    backgroundcolor=\color{CadetBlue!15!white},   
    commentstyle=\color{Red3},
    numberstyle=\tiny\color{gray},
    stringstyle=\color{Blue3},
    basicstyle=\small\ttfamily,
    breakatwhitespace=false,         
    breaklines=true,                 
    numbers=left,                    
    numbersep=5pt,                  
    showspaces=false,                
    showstringspaces=false,
    showtabs=false,                  
    tabsize=2
}%
\lstset{language=[5.3]Lua,style={mystyle}}%

\begin{lstlisting}
function print_rate(kappa,xMin,xMax,npoints,option)
     local c = 1-kappa*kappa
     local croot = (1-kappa*kappa)^(1/2)
     local logx = math.log(xMin)
     local psi = 0
     
     local xstep = (math.log(xMax)-math.log(xMin))/(npoints-1)
     
     arg0 = math.sqrt(xMin/c)
     psi0 = (1/c)*math.exp((kappa*arg0)^2)*(erfc(kappa*arg0)-erfc(arg0))
     
     if option~=[[]] then
  		 tex.sprint("\\addplot+["..option.."] coordinates{") 
  		 -- addplot+ for color cycle to work
     else
  		 tex.sprint("\\addplot+ coordinates{")
     end
     tex.sprint("("..xMin..","..psi0..")")
     
     for i=1, (npoints-1) do
  		 x = math.exp(logx + xstep)
  		 arg = math.sqrt(x/c)
  		 karg = kappa*arg
  		 if karg<5 then 
		 -- this break compensates for exp(karg^2), which multiplies the error in the erf approximation...
  		    logpsi = -math.log(croot) + karg^2 + math.log(erfc(karg)-erfc(arg))
  		    psi = math.exp(logpsi)
  		 else
  		    psi = (1/(karg) - 1/(2*(karg^3)) + 3/(4*(arg^5)) )/(1.77245385*croot)
  		    -- this is the large x asymptote of the reaction rate
  		 end
  		 logx = math.log(x)
  		 tex.sprint("("..x..","..psi..")")
     end
     tex.sprint("}")
end
\end{luacode*}
\end{lstlisting}



%%% Bibliography  %%%%%%%%%%%%%%%%%%%%%%%%%%%%%%%%%%%%%%%%%%%%%%%%%%%%%%%%%%%%%%%%%%%%%%%%%%%%%%%%%

\printbibliography[title={References},heading=bibintoc]

% biblatex also supports chapter-by-chapter bibliography, https://tex.stackexchange.com/a/296502/119566
% see the biblatex manual, section 3.14.3


%%%% Option for natbib %%%%%%%%%%%%%

%%   use an appropriate style (.bst) and your own .bib file[s]

%\bibliographystyle{plainnat}
%\bibliography{bib/main.bib}

\end{document} 
 