\section{Usage}
\label{sec:usage}

The program can be run as a java project inside an IDE, as a Maven project or as a self-contained Jar.
In a real scenario, the program would be delivered via an infected file as a self-contained Jar with minimized and obfuscated code \cite{anderson2008security}. Therefore, here, we will show only this compilation and execution method.
The reader may refer to the \texttt{README} file for details on other compilation and execution methods.

In the following, we assume that the execution environment is provided with JDK SE 8 (downloadable from \cite{jdk}) and Maven 3 (downloadable from \cite{maven}). We also assume that both the Jar execution and the path of the files referenced in the configuration do not require any root rights.

\subsection{Compilation}
\label{sec:compilation}

The bot is compiled into a self-contained Jar using the Maven packaging procedure, running the command:

\begin{verbatim}
  $[bot-dir]> mvn package -P optimize,skip-tests
\end{verbatim}

The previous command executes the packaging using both the profile \texttt{optimize} and \texttt{skip-tests}.

Since the bot code should always be optimized (i.e. minimized and obfuscated), we recommend providing this stage, activating the \texttt{optimize} profile. The code optimization is realized by running Proguard in the background (downloadable from \cite{proguard}) and may slow down the compilation. If the execution environment doesn't provide it, or you don't require code optimization, or you want to speed up the compilation, simply omit the corresponding profile.

Since the project involves more than one hundred JUnit tests, we recommend skipping them to speed up compilation, activating the \texttt{skip-tests} profile. If you considered necessary to perform tests, simply omit the corresponding profile.

The compilation produces a jar file \texttt{botnet-1.0-optimized.jar} in the \texttt{target} folder. Since it is self contained, it can be run in any system directory. In the following, we refer to this jar with the shorter name \texttt{botnet.jar}.


\subsection{Execution}
\label{sec:execution}

The program provides the user with three execution mode: \texttt{default}, \texttt{trace} and \texttt{silent}, which are distinguished by the adopted logging discipline. The \texttt{default} mode prints in console the strictly relevant output and produces log files. The \texttt{trace} mode prints in console a detailed tracing output and produces log files. The \texttt{silent} mode does neither print anything in console nor produce any log file.
To run the program in one of these modes, specify the corresponding option, as indicated in \Cref{sec:helper}.
For more details about logging, please refer to \Cref{sec:bot-implementation}.

The program behaviour can be customized providing a custom YAML configuration file. If no configuration is specified, the program runs with default configuration. To run the program with custom configuration, specify the corresponding configuration, as indicated in \Cref{sec:helper}. For more details about configuration, please refer to \Cref{sec:bot-configuration}.

To run the program with default configuration and default execution mode, run the command:

\begin{verbatim}
  $> java -jar botnet.jar
\end{verbatim}

To run the program in \texttt{trace} execution mode, run the command:

\begin{verbatim}
  $> java -jar botnet.jar --trace
\end{verbatim}

To run the program in \texttt{silent} execution mode, run the command:

\begin{verbatim}
  $> java -jar botnet.jar --silent
\end{verbatim}

To run the program with custom configuration, run the command:

\begin{verbatim}
  $> java -jar botnet.jar --config <CONFIG-FILE>
\end{verbatim}

\subsection{The program helper}
\label{sec:helper}

The reader may refer to the program CLI helper for the options usage. We show here the helper output for reader's convenience.

\begin{verbatim}
  BOTNET version 1.0-SNAPSHOT
  Team: ACM Rome Tor Vergata (http://acm.uniroma2.it)

  A botnet showcase.
  Coursework in Computer Security 2015/2016

  usage: BOTNET [-c <CONFIG-FILE>] [-h] [-s] [-t] [-v]
   -c,--config <CONFIG-FILE>   Custom configuration.
   -h,--help                 Show app helper.
   -s,--silent               Activate silent mode.
   -t,--trace                Activate trace mode.
   -v,--version              Show app version.
\end{verbatim}

\subsection{Sample executions}
\label{sec:sample-executions}

\lipsum[1]

\begin{verbatim}
$> java -jar botnet -config config-custom.yaml -trace
\end{verbatim}

\lipsum[1]
