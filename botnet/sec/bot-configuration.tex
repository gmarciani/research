\section{Bot configuration}
\label{sec:bot-configuration}

The bot behaviour is ruled by a fully customizable configuration. With no custom configuration specified, the bot looks up for a YAML file named \texttt{config.yaml} in the present working directory. If not there, the bot loads the default one, which is hardly wired in code. For details about how to submit a custom configuration, please refer to \Cref{sec:usage}.

We now show the configuration components (i.e. what can be configured) and the YAML configuration schema to express them (i.e. how can be configured).

\begin{description}
  \setlength\itemsep{1em}

  \item[cnfInfo] if true, the bot report includes the current bot configuration.

  \item[tgtInfo] if true, the bot report includes details about the targets to attack.

  \item[sysInfo] if true, the bot report includes details about the localhost system.

  \item[netInfo] if true, the bot report includes details about the network the localhost is attached to.

  \item[polling] period between pollings to controller for the next command to execute. The polling period is a random amount of time within a certain time interval. The period randomness guarantees a greater variance in bots behaviour.

  \item[reconnections] number of times the bot tries to connect to an unreacheable controller.

  \item[reconnectionWait] period between reconnections. The reconnection period is a random amount of time within a certain time interval. The period randomness guarantees a greater variance in bots behaviour.

  \item[proxy] HTTP proxy through which the bot contacts remote controllers and targets. The default configuration has no proxy.

  \item[sleep] calendar for the sleeping mode. The default configuration has no sleep calendar.

  \item[controllers] list of controllers to contact. The default configuration has an empty list of controller and a code wired fallback controller \texttt{data/samples/controllers/1/bot{init,cmd,log}.json}.
\end{description}

The configuration is given in YAML format. The schema shows some non primitive data types that deserves further attention.

\begin{verbatim}
  cnfInfo: True|False
  tgtInfo: True|False
  sysInfo: True|False
  netInfo: True|False
  polling: time-expression
  reconnections: integer in [0,65535]
  reconnectionWait: time-expression
  proxy: proxy-expression
  sleep: cron-expression
  controllers:
    - controller-object
\end{verbatim}

A \texttt{time-expression} represents a temporal interval (e.g. a value in seconds between 3 and 5 seconds). This expression is a string in the form \textit{min-max:unit}, where \textit{min} is a positive long, \textit{max} is a positive long greater than or equal to \textit{min}, and \textit{unit} is the string representation of a standard Java TimeUnit\footnote{i.e. NANOSECONDS, MICROSECONDS, MILLISECONDS, SECONDS, MINUTES, HOURS, DAYS.}. If \textit{min} and \textit{max} are both equal to a positive long \textit{amount}, the time interval could be representaed both by the redundant expression \textit{amount-amount:unit} and by the more compact expression \textit{amount:unit}.

A \texttt{proxy-expression} represents a HTTP proxy (e.g. the proxy 123.123.123.123 with port 3000). This expression is a string in the form \textit{address:port}, where \textit{addres} is an IPv4 address and \textit{port} is a port number. The expression can also be a string \textit{none}, meaning that no proxy should be used, and \textit{null}, meaning that any eventual default proxy should be used.

A \texttt{cron-expression} represents a calendar (e.g. every Wednesday between 10 PM and 11PM). This expression is a standard Unix CRON expression. For details about the standard, please refer to \cite{cron-expression}.

A \texttt{controller-object} represents a bot controller (e.g. a controller with init interface X command interface Y and log interface Z). A controller is expressed in the following form:

\begin{verbatim}
  init: resource expression
  cmd:  resource expression
  log:  resource expression
\end{verbatim}

where a \texttt{resource expression} represents a readable local or remote resource (e.g. the file my-folder/data.json, the web resource http://www.mysite.com/data.json), that is a standard file pathname or a remote Web URL.

Here we show a sample configuration for reader's convenience. Other sample configurations can be found in \texttt{data/samples/configurations}.

\begin{verbatim}
  cnfInfo: True
  tgtInfo: False
  sysInfo: True
  netInfo: False
  polling: 10-15:SECONDS
  reconnections: 3
  reconnectionWait: 3-5:SECONDS
  proxy: 123.123.123.123:3000
  sleep: * * * SAT-SUN * ?
  controllers:
    - init: data/samples/controllers/1/botinit.json
      command: data/samples/controllers/1/botcmd.json
      log: data/samples/controllers/1/botlog.json
    - init: data/samples/controllers/2/botinit.json
      command: data/samples/controllers/2/botcmd.json
      log: data/samples/controllers/2/botlog.json
\end{verbatim}
