\section{Bot commands}
\label{sec:bot-commands}

In the following we list all commands with the corresponding JSON schema.
For sake of synthesis, we indicate in a pretty custom way both the optional fields and the non primitive data types.

For example, in the following JSON: \texttt{f1} is a mandatory field with value \texttt{v1}, \texttt{f2} is a mandatory field with domain in ad-hoc data type \texttt{type}, \texttt{f3} is an optional field with domain in ad-hoc data type \texttt{type} and default value \texttt{default}, \texttt{f4} and \texttt{f5} are optional fields going together with domain in ad-hoc data type \texttt{type} and default value \texttt{default}.

\begin{verbatim}
  {
    "f1": "v1",
    "f2": type,
    ("f3": type (default)),
    (
      "f4": type (default),
      "f5": type (default)
    )
  }
\end{verbatim}

In the command listed below there are the following non primitive data types:

\begin{description}
  \setlength\itemsep{1em}

  \item[\texttt{time-expression}] represents a temporal interval (e.g. a value in seconds between 3 and 5 seconds). This expression is a string in the form \textit{min-max:unit}, where \textit{min} is a positive long, \textit{max} is a positive long greater than or equal to \textit{min}, and \textit{unit} is the string representation of a standard Java TimeUnit\footnote{i.e. NANOSECONDS, MICROSECONDS, MILLISECONDS, SECONDS, MINUTES, HOURS, DAYS.}. If \textit{min} and \textit{max} are both equal to a positive long \textit{amount}, the time interval could be representaed both by the redundant expression \textit{amount-amount:unit} and by the more compact expression \textit{amount:unit}.
  For example, the following are valid expressions:

  \begin{verbatim}
    3-5:SECONDS
  \end{verbatim}

  and

  \begin{verbatim}
    10:SECONDS
  \end{verbatim}

  \item[\texttt{proxy-expression}] represents a HTTP proxy (e.g. the proxy 123.123.123.123 with port 3000). This expression is a string in the form \textit{address:port}, where \textit{addres} is an IPv4 address and \textit{port} is a port number. The expression can also be a string \textit{none}, meaning that no proxy should be used, and \textit{null}, meaning that any eventual default proxy should be used.

  For example, the following are valid expressions:

  \begin{verbatim}
    123.123.123:3000
  \end{verbatim}

  and

  \begin{verbatim}
    none
  \end{verbatim}

  \item[\texttt{attack-object}] represents a target of a HTTP attack (e.g. )
  \begin{verbatim}
  {
    "method": value in [GET,POST],
    "target": url-expression,
    ("proxy": proxy-expression (null)),
    ("properties": map-expression ({})),
    (
      "executions": integer in [1,65535] (1),
      "period": time-expression (null)
    )
  }
  \end{verbatim}

  For example, the following are valid expressions:

  \begin{verbatim}
  {
    "method": "GET",
    "target": "http://www.google.com"
  }
  \end{verbatim}

  and

  \begin{verbatim}
  {
    "method": "GET",
    "target": "http://www.google.com",
    "proxy": "123.123.123:3000",
    "properties": map-expression {
      "User-Agent": "CustomUserAgent"
    },
    "executions": 3,
    "period": "3-5:SECONDS"
  }
  \end{verbatim}

\end{description}

\begin{description}
  \setlength\itemsep{1em}

  \item[ATTACK-HTTP] \lipsum[1]
  \begin{verbatim}
  {
    "command": "ATTACK_HTTP",
    "attacks": [ attack-object ]
    ],
    ("delay": time-expr (null))
  }
  \end{verbatim}

  \item[CALMDOWN] all attacks are unscheduled.
  If \texttt{wait} is true, it waits for for termination of currently executing attacks; if false, it kills them immediately.
  If a \texttt{delay} is specified, the commands is executed after a random amount of time within the specified interval.

  \begin{verbatim}
  {
    "command": "CALMDOWN",
    ("wait": boolean (false)),
    ("delay": time-expr (null))
  }
  \end{verbatim}

  \item[KILL] the bot is killed, that is alla attacks are unscheduled and it transit to state \texttt{DEAD} for resource releasing.
  If \texttt{wait} is true, it waits for for termination of currently executing attacks; if false, it kills them immediately.
  If a \texttt{delay} is specified, the commands is executed after a random amount of time within the specified interval.

  \begin{verbatim}
  {
    "command": "KILL",
    ("wait": boolean (false)),
    ("delay": time-expr (null))
  }
  \end{verbatim}

  \item[NONE] instructs the bot to do nothing, that is to neither change its executions flow nor send any report. In particular, both the null (empty file) and the empty command (empty JSON, i.e. {}) are equivalent to \texttt{NONE}.

  \begin{verbatim}
  {
    "command": "NONE"
  }
  \end{verbatim}

  \item[REPORT] the bot sends the report to the controller.
  If a \texttt{delay} is specified, the commands is executed after a random amount of time within the specified interval.

  \begin{verbatim}
  {
    "command": "REPORT",
    ("delay": time-expr (null))
  }
  \end{verbatim}

  \item[RESTART] all attacks are unscheduled, the bot transits to state \texttt{INIT} trying to contact the controller with \texttt{resource} as its init-interface.
  If \texttt{wait} is true, it waits for for termination of currently executing attacks; if false, it kills them immediately.
  If \texttt{delay} is specified, the commands is executed after a random amount of time within the specified interval.

  \begin{verbatim}
  {
    "command": "RESTART",
    "resource": resource-expr,
    ("wait": boolean (false)),
    ("delay": time-expr (null))
  }
  \end{verbatim}

  \item[SAVE-CONFIG] the currently loaded bot configuration is locally saved as default configuration.
  If a \texttt{delay} is specified, the commands is executed after a random amount of time within the specified interval.

  \begin{verbatim}
  {
    "command": "SAVE_CONFIG",
    ("delay": time-expr (null))
  }
  \end{verbatim}

  \item[SLEEP] all attacks are suspended and the bot transits to state \texttt{SLEEP}. If a \texttt{timeout} si specified the command \texttt{WAKEUP} is internally invoked after a random amount of time within the specified interval.
  If a \texttt{delay} is specified, the commands is executed after a random amount of time within the specified interval.

  \begin{verbatim}
  {
    "command": "SLEEP",
    ("timeout": time-expr (null)),
    ("delay": time-expr (null))
  }
  \end{verbatim}

  \item[UPDATE] the bot configuration is updated with the properties specified in \texttt{settings}. This command is tipically used to update the property \texttt{sleep} that sets the sleep mode.
  If a \texttt{delay} is specified, the commands is executed after a random amount of time within the specified interval.

  \begin{verbatim}
  {
    "command": "UPDATE",
    "settings": map-expr,
    ("delay": time-expr (null))
  }
  \end{verbatim}

  \item[WAKEUP] if the bot is in state \texttt{SLEEP}, it lets scheduled attacks be able to fire again and transits to state \texttt{EXECUTION}.
  If a \texttt{delay} is specified, the commands is executed after a random amount of time within the specified interval.

  \begin{verbatim}
  {
    "command": "WAKEUP",
    ("delay": time-expr (null))
  }
  \end{verbatim}

\end{description}
