\section{Validation}
\label{sec:performance-modeling-validation}
A well conducted performance modeling workflow should include a final validation step to assess the consistency of the model with the real system. 
%
As the simulation main purpose is insight, a widely adopted Turing-like technique is to place the computational model alongside with the real system and assess the consistency of performance metrics.
%
Clearly, we cannot adopt this technique because, since we do not have any real system at hand, we cannot compare the model with its real counterpart.
%
For this reason, we totally rely on the validation with respect to the analytical model.

In Tables \ref{tbl:evaluation-performance-metrics-1}, \ref{tbl:evaluation-performance-metrics-2-5} and \ref{tbl:evaluation-performance-metrics-2-20} we show the comparison between theoretical performance results, taken from the analytical model, and their experimental counterpart, taken from the simulator.
%
The results demonstrate that our simulator is a pretty reliable tool to conduct the performance analysis for the target system. 