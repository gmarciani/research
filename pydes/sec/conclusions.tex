\section{Conclusions}
\label{sec:conclusions}

%%
% SUMMARY
%%
In this work we propose a next-event simulator for a two-layer Cloud system with off-loading policies on class-partitioned workload, whose random components leverage a multi-stream Lehmer pseudo-random number generator.

%%
% CONCLUSIONS
%%
We may conclude that 
(i) our simulator returns experimental results that approximates the theoretical ones, 
(ii) the adopted pseudo-random number generator has a satisfactory randomness degree,
(iii) the system can achieve the steady-state,
(iv) the adoption of an off-loading policy and the fine tuning of its parameters are crucial for system performances.

%%
% IMPROVEMENTS
%%
Although our results are pretty satisfactory, the proposed solutions could certainly be improved and be subjected to a more in-depth analysis.
%
From an analytical point of view, the proposed analytical model should be refined adjusting some of the very strong assumption we made to simplify it.
%
From an implementation point of view, the proposed solution should be ported from Python to C and leverage multi-threading to achieve better performances, e.g. to speed-up the algorithms to find suitable multipliers for modulus in 64-bit architectures and make faster simulations.
%
From a randomness point of view, the proposed random number generator should be tested more extensively. For example, we may (i) take into account more tests of randomness (ii) use a pseudo-random number generator with a 64-bit modulus and less number of streams.
%
Finally, the simulation model should be extended in order to 
(i) study the impact of different thresholds,
(ii) study the influence of different server selection policies for the $2^{nd}$ class task interruption, e.g. equity-selection, and 
(iii) achieve more performance evaluation goals, such as what-if analysis with respect to the variation of the arrival and service processes.
