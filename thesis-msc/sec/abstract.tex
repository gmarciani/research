% %
% MOTIVATION
% %
The containers orchestration in Cloud environment is a disruptive technology that opens new research fields and industrial applications.
%
First, Cloud computing allows applications to scale over a theoretically infinite pool of resources in a pay-per-use fashion.
%
Then, the deployment of applications as containerized micro-services provides software engineers with a fine-grained control over components placement and scaling.
%
Since placement and scaling decisions have both monetary and performance impact, the elastic container orchestration, that is the self-adaptive scaling and placing of containers, is a strategic asset both for Cloud-oriented and On-Premises industries.
%
% %
% PROBLEM STATEMENT
% %
Even if the controllability over placement and scaling allows companies to satisfy both budget and QoS requirements, the high variability of workloads makes manual interventions ineffective and requires to put in place efficient and effective automated solutions.
%


% %
% APPROACH
% %
In this work, we propose and compare some heuristics solutions for the elastic containers orchestration.
%
In particular, we focus on the adoption of Hill Climbing techniques to place and scale containers in the Kubernetes environment, that is an industrial de-facto standard for containers orchestration.
%
We propose an elasticity service that is deployable as a Kubernetes controller.
%
Hence, our main contribution is the design, implementation and experimental analysis of solutions to drive elastic containers orchestration as an extension of the traditional reactive threshold-based auto-scaling, natively provided by Kubernetes.

%
%
% %
% RESULTS
% %
%The experimental results show that our solution converges to an auto-scaling policy that meets QoS requirements while reducing the cost of the infrastructure. 
%
%
% %
% CONCLUSIONS
% %
%Although there is still a long way to go to speed-up the learning convergence, our work show that the application of Reinforcement Learning for elastic containers auto-scaling should be a promising research field for Cloud-driven industries.
