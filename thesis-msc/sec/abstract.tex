% %
% MOTIVATION
% %
The containers orchestration in Cloud environment is a disruptive technology that opens new research fields and industrial applications.
%
First, Cloud computing allows applications to scale over a theoretically infinite pool of resources in a pay-per-use fashion.
%
Next, the deployment of applications as containerized micro-services provides software engineers with a fine-grained control over components scaling.
%
% %
% PROBLEM STATEMENT
% %
Even if the controllability over scaling allows companies to satisfy both budget and QoS requirements, the high variability of workloads makes ineffective the widely adopted techniques of manual scaling.
%
Since scaling decisions have both monetary and performance costs, the elastic container orchestration, that is the adaptive auto-scaling of containers, is a strategic asset for Cloud-centric industries.


% %
% APPROACH
% %
In this work, we propose a smart elasticity solution for containers orchestration, that is the proactive auto-scaling of containers though the use of Artificial Intelligence.
%
In particular, we focus on the adoption of Reinforcement Learning techniques to auto-scale containers in the Kubernetes environment, that is an industrial de-facto standard for containers orchestration.
%
We propose a smart elasticity service that is deployable as a Kubernetes controller implemented as a highly scalable micro-service architecture.
%
Hence, our main contribution is the design, implementation and experimental analysis of an Artificial Intelligence solution to realize proactive elasticity as an extension of the traditional reactive auto-scaling, natively provided by Kubernetes.

%
%
% %
% RESULTS
% %
%The experimental results show that our solution converges to an auto-scaling policy that meets QoS requirements while reducing the cost of the infrastructure. 
%
%
% %
% CONCLUSIONS
% %
%Although there is still a long way to go to speed-up the learning convergence, our work show that the application of Reinforcement Learning for elastic containers auto-scaling should be a promising research field for Cloud-driven industries.
