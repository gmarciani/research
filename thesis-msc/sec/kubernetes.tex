\chapter{Kubernetes}
\label{chp:kubernetes}

Kubernetes is a production-grade, open-source platform that orchestrates the placement (scheduling) and execution of application containers within and across computer clusters.


With modern web services, users expect applications to be available 24/7, and developers expect to deploy new versions of those applications several times a day. Containerization helps package software to serve these goals, enabling applications to be released and updated in an easy and fast way without downtime. Kubernetes helps you make sure those containerized applications run where and when you want, and helps them find the resources and tools they need to work. Kubernetes is a production-ready, open source platform designed with Google's accumulated experience in container orchestration, combined with best-of-breed ideas from the community.

Kubernetes is an open-source platform designed to automate deploying, scaling, and operating application containers.

With Kubernetes, you are able to quickly and efficiently respond to customer demand:

Deploy your applications quickly and predictably.
Scale your applications on the fly.
Roll out new features seamlessly.
Limit hardware usage to required resources only.

Our goal is to foster an ecosystem of components and tools that relieve the burden of running applications in public and private clouds.
Kubernetes is

Portable: public, private, hybrid, multi-cloud
Extensible: modular, pluggable, hookable, composable
Self-healing: auto-placement, auto-restart, auto-replication, auto-scaling

Google started the Kubernetes project in 2014. Kubernetes builds upon a decade and a half of experience that Google has with running production workloads at scale, combined with best-of-breed ideas and practices from the community.

At a minimum, Kubernetes can schedule and run application containers on clusters of physical or virtual machines. However, Kubernetes also allows developers to ‘cut the cord’ to physical and virtual machines, moving from a host-centric infrastructure to a container-centric infrastructure, which provides the full advantages and benefits inherent to containers. Kubernetes provides the infrastructure to build a truly container-centric development environment.


% %
% ARCHITECTURE
% %
\section{Architecture}
\label{sec:kubernetes-architecture}

KUBERNETES CLUSTER

Kubernetes coordinates a highly available cluster of computers that are connected to work as a single unit. The abstractions in Kubernetes allow you to deploy containerized applications to a cluster without tying them specifically to individual machines. To make use of this new model of deployment, applications need to be packaged in a way that decouples them from individual hosts: they need to be containerized. Containerized applications are more flexible and available than in past deployment models, where applications were installed directly onto specific machines as packages deeply integrated into the host. Kubernetes automates the distribution and scheduling of application containers across a cluster in a more efficient way. Kubernetes is an open-source platform and is production-ready.

The Master is responsible for managing the cluster. The master coordinates all activities in your cluster, such as scheduling applications, maintaining applications' desired state, scaling applications, and rolling out new updates.

A node is a VM or a physical computer that serves as a worker machine in a Kubernetes cluster. Each node has a Kubelet, which is an agent for managing the node and communicating with the Kubernetes master. The node should also have tools for handling container operations, such as Docker or rkt. A Kubernetes cluster that handles production traffic should have a minimum of three nodes.

When you deploy applications on Kubernetes, you tell the master to start the application containers. The master schedules the containers to run on the cluster's nodes. The nodes communicate with the master using the Kubernetes API, which the master exposes. End users can also use the Kubernetes API directly to interact with the cluster.




% %
% RESOURCE MANAGEMENT
% %
\section{Resource Management}
\label{sec:kubernetes-resource-management}


% %
% POD
% %
\section{Pod}
\label{sec:kubernetes-pod}

Pods are the smallest deployable units of computing that can be created and managed in Kubernetes.

A Pod models an application-specific "logical host" and can contain different application containers which are relatively tightly coupled. For example, a Pod might include both the container with your Node.js app as well as a different container that feeds the data to be published by the Node.js webserver. The containers in a Pod share an IP Address and port space, are always co-located and co-scheduled, and run in a shared context on the same Node.

A Pod always runs on a Node. A Node is a worker machine in Kubernetes and may be either a virtual or a physical machine, depending on the cluster. Each Node is managed by the Master. A Node can have multiple pods, and the Kubernetes master automatically handles scheduling the pods across the Nodes in the cluster. The Master's automatic scheduling takes into account the available resources on each Node.

Every Kubernetes Node runs at least:

Kubelet, a process responsible for communication between the Kubernetes Master and the Nodes; it manages the Pods and the containers running on a machine.
A container runtime (like Docker, rkt) responsible for pulling the container image from a registry, unpacking the container, and running the application.

A node is a worker machine in Kubernetes and may be a VM or physical machine, depending on the cluster. Multiple Pods can run on one Node. 

A pod (as in a pod of whales or pea pod) is a group of one or more containers (such as Docker containers), with shared storage/network, and a specification for how to run the containers. A pod’s contents are always co-located and co-scheduled, and run in a shared context. A pod models an application-specific “logical host” - it contains one or more application containers which are relatively tightly coupled — in a pre-container world, they would have executed on the same physical or virtual machine.

The shared context of a pod is a set of Linux namespaces, cgroups, and potentially other facets of isolation - the same things that isolate a Docker container. Within a pod’s context, the individual applications may have further sub-isolations applied.

Containers within a pod share an IP address and port space, and can find each other via localhost. They can also communicate with each other using standard inter-process communications like SystemV semaphores or POSIX shared memory. Containers in different pods have distinct IP addresses and can not communicate by IPC.

In terms of Docker constructs, a pod is modelled as a group of Docker containers with shared namespaces and shared volumes. PID namespace sharing is not yet implemented in Docker.

Like individual application containers, pods are considered to be relatively ephemeral (rather than durable) entities. As discussed in life of a pod, pods are created, assigned a unique ID (UID), and scheduled to nodes where they remain until termination (according to restart policy) or deletion. If a node dies, the pods scheduled to that node are scheduled for deletion, after a timeout period. A given pod (as defined by a UID) is not “rescheduled” to a new node; instead, it can be replaced by an identical pod, with even the same name if desired, but with a new UID (see replication controller for more details). (In the future, a higher-level API may support pod migration.)

When something is said to have the same lifetime as a pod, such as a volume, that means that it exists as long as that pod (with that UID) exists. If that pod is deleted for any reason, even if an identical replacement is created, the related thing (e.g. volume) is also destroyed and created anew.

MOTIVATIONS

MANAGEMENT

Pods are a model of the pattern of multiple cooperating processes which form a cohesive unit of service. They simplify application deployment and management by providing a higher-level abstraction than the set of their constituent applications. Pods serve as unit of deployment, horizontal scaling, and replication. Colocation (co-scheduling), shared fate (e.g. termination), coordinated replication, resource sharing, and dependency management are handled automatically for containers in a pod.
Resource sharing and communication

RESOURCE SHARING AND COMMUNICATION

Pods enable data sharing and communication among their constituents.

The applications in a pod all use the same network namespace (same IP and port space), and can thus “find” each other and communicate using localhost. Because of this, applications in a pod must coordinate their usage of ports. Each pod has an IP address in a flat shared networking space that has full communication with other physical computers and pods across the network.

The hostname is set to the pod’s Name for the application containers within the pod. More details on networking.

In addition to defining the application containers that run in the pod, the pod specifies a set of shared storage volumes. Volumes enable data to survive container restarts and to be shared among the applications within the pod.

USES OF PODS

Pods can be used to host vertically integrated application stacks (e.g. LAMP), but their primary motivation is to support co-located, co-managed helper programs

ALTERNATIVES

Why not just run multiple programs in a single (Docker) container?

Transparency. Making the containers within the pod visible to the infrastructure enables the infrastructure to provide services to those containers, such as process management and resource monitoring. This facilitates a number of conveniences for users.
Decoupling software dependencies. The individual containers may be versioned, rebuilt and redeployed independently. Kubernetes may even support live updates of individual containers someday.
Ease of use. Users don’t need to run their own process managers, worry about signal and exit-code propagation, etc.
Efficiency. Because the infrastructure takes on more responsibility, containers can be lighter weight.

Why not support affinity-based co-scheduling of containers?

That approach would provide co-location, but would not provide most of the benefits of pods, such as resource sharing, IPC, guaranteed fate sharing, and simplified management

TERMINATION OF PODS

Because pods represent running processes on nodes in the cluster, it is important to allow those processes to gracefully terminate when they are no longer needed (vs being violently killed with a KILL signal and having no chance to clean up). Users should be able to request deletion and know when processes terminate, but also be able to ensure that deletes eventually complete. When a user requests deletion of a pod the system records the intended grace period before the pod is allowed to be forcefully killed, and a TERM signal is sent to the main process in each container. Once the grace period has expired the KILL signal is sent to those processes and the pod is then deleted from the API server. If the Kubelet or the container manager is restarted while waiting for processes to terminate, the termination will be retried with the full grace period.

By default, all deletes are graceful within 30 seconds. The kubectl delete command supports the --grace-period=<seconds> option which allows a user to override the default and specify their own value. The value 0 force deletes the pod. In kubectl version >= 1.5, you must specify an additional flag --force along with --grace-period=0 in order to perform force deletions.

SERVICES

A Kubernetes Service is an abstraction layer which defines a logical set of Pods and enables external traffic exposure, load balancing and service discovery for those Pods.

Kubernetes Pods are mortal. Pods in fact have a lifecycle. When a worker node dies, the Pods running on the Node are also lost. A ReplicationController might then dynamically drive the cluster back to desired state via creation of new Pods to keep your application running. As another example, consider an image-processing backend with 3 replicas. Those replicas are fungible; the front-end system should not care about backend replicas or even if a Pod is lost and recreated. That said, each Pod in a Kubernetes cluster has a unique IP address, even Pods on the same Node, so there needs to be a way of automatically reconciling changes among Pods so that your applications continue to function.

A Service in Kubernetes is an abstraction which defines a logical set of Pods and a policy by which to access them. Services enable a loose coupling between dependent Pods. A Service is defined using YAML (preferred) or JSON, like all Kubernetes objects. The set of Pods targeted by a Service is usually determined by a LabelSelector (see below for why you might want a Service without including selector in the spec).

Although each Pod has a unique IP address, those IPs are not exposed outside the cluster without a Service. Services allow your applications to receive traffic. Services can be exposed in different ways by specifying a type in the ServiceSpec:

ClusterIP (default) - Exposes the Service on an internal IP in the cluster. This type makes the Service only reachable from within the cluster.
NodePort - Exposes the Service on the same port of each selected Node in the cluster using NAT. Makes a Service accessible from outside the cluster using <NodeIP>:<NodePort>. Superset of ClusterIP.
LoadBalancer - Creates an external load balancer in the current cloud (if supported) and assigns a fixed, external IP to the Service. Superset of NodePort.
ExternalName - Exposes the Service using an arbitrary name (specified by externalName in the spec) by returning a CNAME record with the name. No proxy is used. This type requires v1.7 or higher of kube-dns.

More information about the different types of Services can be found in the Using Source IP tutorial. Also see Connecting Applications with Services.

Additionally, note that there are some use cases with Services that involve not defining selector in the spec. A Service created without selector will also not create the corresponding Endpoints object. This allows users to manually map a Service to specific endpoints. Another possibility why there may be no selector is you are strictly using type: ExternalName

A Service routes traffic across a set of Pods. Services are the abstraction that allow pods to die and replicate in Kubernetes without impacting your application. Discovery and routing among dependent Pods (such as the frontend and backend components in an application) is handled by Kubernetes Services.

Services match a set of Pods using labels and selectors, a grouping primitive that allows logical operation on objects in Kubernetes. Labels are key/value pairs attached to objects and can be used in any number of ways:

Designate objects for development, test, and production
Embed version tags
Classify an object using tags

SCALING

we created a Deployment, and then exposed it publicly via a Service. The Deployment created only one Pod for running our application. When traffic increases, we will need to scale the application to keep up with user demand.

Scaling is accomplished by changing the number of replicas in a Deployment.

Scaling out a Deployment will ensure new Pods are created and scheduled to Nodes with available resources. Scaling in will reduce the number of Pods to the new desired state. Kubernetes also supports autoscaling of Pods, but it is outside of the scope of this tutorial. Scaling to zero is also possible, and it will terminate all Pods of the specified Deployment.

Running multiple instances of an application will require a way to distribute the traffic to all of them. Services have an integrated load-balancer that will distribute network traffic to all Pods of an exposed Deployment. Services will monitor continuously the running Pods using endpoints, to ensure the traffic is sent only to available Pods.

Once you have multiple instances of an Application running, you would be able to do Rolling updates without downtime.


ROLLING UPDATES

Users expect applications to be available all the time and developers are expected to deploy new versions of them several times a day. In Kubernetes this is done with rolling updates. Rolling updates allow Deployments' update to take place with zero downtime by incrementally updating Pods instances with new ones. The new Pods will be scheduled on Nodes with available resources.
Rolling updates allow Deployments' update to take place with zero downtime by incrementally updating Pods instances with new ones.

If a Deployment is exposed publicly, the Service will load-balance the traffic only to available Pods during the update.

Rolling updates allow the following actions:

Promote an application from one environment to another (via container image updates)
Rollback to previous versions
Continuous Integration and Continuous Delivery of applications with zero downtime


% %
% ELASTICITY
% %
\section{Elasticity}
\label{sec:kubernetes-elasticity}

\lipsum[1]


% %
% PIPELINE
% %
\section{Pipeline}
\label{sec:openshift-pipeline}

\lipsum[1]