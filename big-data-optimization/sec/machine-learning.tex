\chapter{Machine Learning}
\label{chp:machine-learning}
Il machine learning si occupa di definire tecniche di apprendimento automatico.
Un task di machine learning è classificato in base al processo di apprendimento e al modello prodotto.

Il processo di apprendimento può essere
\begin{itemize}
  \item supervisionato: in tal caso parliamo di machine learning supervisionato. La classificazione è guidata da un dataset classificato. Sono noti a priori pattern rappresentativi della classifiazione, Le tecniche più rilevanti sono: Support Vector Machine, Neural Netorks, Decision Trees con boosting o bagging e Bayesian Classifier.

  \item non supervisionato: in tal caso parliamo di machine learning non supervisionato. La classificazione avviene senza avere a disposizione un dataset classificato. Non sono noti a priori pattern rappresentativi della classificazione. Le tecniche più rilevanti sono: Gerarchic Clustering, Markov Hidden Chains, Self-Organized Maps e Neural Networks.
\end{itemize}

Il modello prodotto può essere:
\begin{itemize}
  \item modello predittivo: predizione dell'evoluzione di un fenomeno identificato dai dati analizzati.
  \item modello descrittivo: caratterizzazione generale dei dati analizzati.
\end{itemize}

I più importanti task di machine learning sono: \textit{classificazione}, \textit{regressione} e \textit{clustering}.

Nei capitoli successivi applicheremo le reti neurali e le support vector machines in questi task.


\section{Classificazione}
\label{sec:machine-learning.classification}
La classificazione è un task di machine learning predittivo con output categorico. Lo scopo è individuare l'appartenza di un pattern ad una specifica classe. Il pattern è un insieme di coppie $(attributi,classe)$ rappresentativi delle diverse classi, sulla base dei quali si costruisce un modello matematico che, dato un generico elemento appartenente allo spazio delle caratteristiche, definisce la corrispondente classe di appartenenza.

Per valutare la bontà di un modello di apprendimento bisogno tenere separati dati di apprendimento da dati di test.
Il 70\% delle istanze viene utilizzato come training ed il 30\% come test.

Un problema di classificazione può essere:

\begin{itemize}
  \item classificazione binaria: il target può appartenere o non appartenere ad una certa classe.
  \item classificazione multiclasse: il target può appartenere a più classi secondo valori specifici.
\end{itemize}

L'input di un problema di classificazione è il \textit{training set}:

\begin{equation}
  \label{eqn:machine-learning.training-set}
  T:={(x^p,y^p): x\in\Re^{n},y^{p}\in S\subset N, p=1,...,P}
\end{equation}

dove $P$ è il numero di osservazioni (o pattern) ognuna delle quali è descritta da $n$ attributi.

Per ogni osservazione $p\in[1,P]$ si ha un vettore di osservazione $x^{p}=(x_{i}^{p})$ di dimensione $n$ in cui la generica componente $x_{i}^{p}$ è l'attributo $i$-esimo dell'osservazione $p$-esima, con $i\in[1,n]$.

Per ogni osservazione $p\in[1,P]$ su ha un vettore di classificazione $y^{p}$ che indica la classe di appartenenza dell'osservazione.

L'output di un problema di classificazione è il \textit{classification model}:

\begin{equation}
  \label{eqn:machine-learning.classification-model}
  f:\Re^{n}\rightarrow S\subset N
\end{equation}

è una funzione in grado di approssimare la relazione $x^{i}\rightarrow y^{i}$, e quindi, dato un nuovo elemento $x$, di predire la corrispondente cassificazione $y$.

\section{Regressione}
\label{Sec:machine-learning.regression}
La regressione è un task di machine learning supervisionato predittivo con output continuo. Lo scopo è determinare il legame funzionale tra coppie di valori \footnote{L'obiettivo non è interpolare i dati ma approssimare la funzione che li ha generati.}.
Il pattern è costituito da coppie di valori rappresentativi di una funzione incognita a valori reali, sulla base dei quali si costruisce l'approssimazione della funzione reale.

\section{Clustering}
\label{sec:machine-learning.clustering}
Il clustering è un task di machine learning non supervisionato descrittivo con output categorico. Lo scopo è determinare le categorie (cluster) in cui classificare il dataset ed il modello matematico per stabilirne l'appartenenza. I cluster devono essere più distinti possibile, e le osservazioni ad essi appartenenti devono essere più simili possibile.
