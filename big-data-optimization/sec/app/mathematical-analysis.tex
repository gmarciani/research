\chapter{Richiami di analisi matematica}
\label{chp:foundations-mathematical-analysis}
In questa sezione richiamiamo alcuni concetti di analisi matematica.


\section{Limite ed accumulazione di successione}

\begin{definition}[Limite di successione]
	\label{dfn:succession-limit}
	Sia data una successione $\{x_{k}\}$ ad elementi in $\Re^{n}$. La successione $\{x_{k}\}$ converge ad $x\in\Re^{n}$ se
	\begin{equation}
	\label{eqn:succession-limit}
	\lim_{k\rightarrow\infty}\norm{x_{k}-x}=0
	\end{equation}
	ovvero se per ogni $\epsilon>0$ esiste un $k_{\epsilon}$ tale che $\norm{x_{k}-x}<\epsilon$ per ogni $k\geq k_{\epsilon}$.
\end{definition}

\begin{definition}[Punto di accumulazione per successione]
	\label{dfn:succession-accumulation}
	Sia data una successione $\{x_{k}\}_{N}$ ad elementi in $\Re^{n}$. Il punto $\tilde{x}\in\Re^{n}$ è un punto di accumulazione per $\{x_{k}\}_{N}$ se esiste un insieme inifito $K\subseteq N$ di indici, tale che la sottosuccessione $\{x_{k}\}_{K}$ converge a $\tilde{x}$
\end{definition}

Un punto di accumulazione può essere definito anche in modo alternativo.
Dato l'insieme $A\subset\Re$, $\tilde{x}\in\Re$ è un punto di accumulazione se

\begin{equation}
	\label{eqn:accumulation-point}
	\forall\varrho>0\exists x\in A\cap B(\tilde{x},\varrho).x\neq\tilde{x}
\end{equation}

Intuitivamente questo vuol dire che zoommando a qualsiasi livello nell'intorno di $\tilde{x}$ è sempre possibile vedere punti di $A$ diversi da $\tilde{x}$.

La differenza tra punto limite e punto di accumulazione è che al punto limite converge tutta la successione, mentre per una stessa successione posso individuare punti di accumulazione (eventualmente diversi) a cui convergono sottosuccessioni distinte della stessa successione. Una successione ammette al più un punto limite, mentre può ammettere zero, uno, molti o infiniti punti di accumulazione, eventualmente distinti.

Una successione in un insieme compatto, ammette punti di accumulazione nell'insieme.


\section{Norma di matrice}
La norma di una matrice soddisfa le seguenti proprietà

\begin{eqnarray}
\label{eqn:matrix-norm-properties}
\begin{split}
\norm{AB}       & \leq \norm{A}\norm{B} \\
\norm{A}        & \geq 0 \\
\norm{A}        & =    0 \Leftrightarrow A=0 \\
\norm{A+B}      & \leq \norm{A}+\norm{B} \\
\norm{\alpha A} & =    \abs{\alpha}\norm{A}
\norm{A}        & \leq \lambda_{max}(A)
\end{split}
\end{eqnarray}


\section{Coercività}
Il concetto di cercività è molto utile a determinare la compattezza degli insiemi di livello e l'esistenza di soluzioni locali.

\begin{definition}[Funzione coerciva]
	\label{dfn:coercive-function}
	Una funzione $f:\Re^{n}\rightarrow\Re$ è coerciva se esiste una successione $\{x_{k}\}$ tale che
	\begin{equation}
	\label{eqn:coercive-function}
	\lim_{k\rightarrow\infty}\norm{x_{k}}=\infty
	\Rightarrow
	\lim_{k\rightarrow\infty}f(x_{k})=\infty
	\end{equation}
\end{definition}

Se dunque l'insieme del problema è compatto, applico Weierstrass; se non lo è (e.g. $\Re^{n}$) verifico la coercività della funzione.

Il quadrato conserva la coercività.

Una forma quadratica tale che $\nabla f(x)\succ0$ è coerciva.

\begin{definition}[Lipschitz continuità]
	\label{dfn:continuity-lipschitz}
	Una funzione $f$ è Lipschitz continua se e solo se esiste un $L>0$ tale che
	\begin{equation}
		\label{eqn:lipschitz.continuity}
		\norm{f(x_{2})-f(x_{1})}\leq L\norm{x_{2}-x{1}}
	\end{equation}
	dove $L$ è detta \textit{costante di Lipschitz}.
	La Lipschitz continuità implica la continuità.
\end{definition}
