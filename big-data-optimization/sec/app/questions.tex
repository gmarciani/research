\chapter{Domande}
\label{chp:questions}

In riferimento agli appunti:

\begin{enumerate}

	\item È vero che un sistema di disequazioni lineari è sempre un insieme convesso?

	\item quali sono i vantaggi di avere una funzione quadratica (oltre al fatto che l'essiano non dipende da x)?

	\item la coercività implica sempre l'esistenza di un ottimo?

	\item quale è l'utilità di una funzione di forzamento?

	\item perchè in un insieme limitato una successione ha sempre punti di accumulazione? Deve essere infinita perchè questo sia verificato?

	\item in Armijo, $\gamma$ rappresenta la nostra tolleranza ad incrementi locali?

	\item in Armijo, $\delta$ cosa rappresenta?

	\item che relazione c'è tra punti di accumulazione della successione generata da un algoritmo e punti stazionari della funzione?

	\item la line search esatta su funzione quadratica strettamente convessa è il metodo migliore per determinare il passo. Perchè la funzione deve essere quadratica? Non basterebbe la stretta convergenza?

	\item in un algoritmo con espansione del passo, espando il passo fin tanto che soddisfa Armijo, o lo espando fino a quando lo soddisfa?

	\item per quale motivo non voglio direzioni quasi ortogonali al gradiente?

	\item la line search esatta è un buon metodo per f.o. convessa?

\end{enumerate}

In riferimento al libro di testo \cite{grippo2011metodi}:

\begin{enumerate}

	\item contorno e curva di livello sono sinonimi? (pag. 13)

	\item cosa si intende per ottimizzazione deterministica/probabilistica? (pag . 56)

	\item cosa si intende per convergenza deterministica? cosa si intenderebbe per convergenza probabilistica? (pag. 64)

	\item che differenza c'è tra la convergenza (b) e (c)? (pag. 65)

	\item cosa si intende per "modello locale"? (pag. 73)

	\item cosa si intende per "globalizzazione dell'algoritmo"? (pag. 73)

	\item perchè il passo della line search è una strategia di globalizzazione? (pag. 73)

	\item i metodi trust-region sono applicabili solo a forme quadratiche? Una qualunque funzione viene convertita localmente in una forma quadratica analizzata in un intorno sferico? (pag. 73)

	\item cosa vuol dire utilizzare un modello locale per il calcolo di $\delta_{k}$? (pag. 75)

	\item in quali casi non si hanno informazioni complete su $f,\nabla f$ per ogni iterazione dell'algoritmo? (pag. 91)

	\item come può valere la convergenza del gradiente se non esistono punti di accumulazione? (pag. 65)

	\item (pag 99) perchè il passo in un range con interpolazione e approssimazione quadratica cubica della funzione obiettivo dovrebbe ridurre il numero di iterazioni ?

	\item (pag 153) migliori rispetto a chi? per quale motivo il metodo del gradiente non ha una buona convergenza locale?

	\item (pag 154) la Proposizione 4.2 vuol dire, in parole povere, che i passi diventano sempre più piccoli e/o che se non termina vuol dire che possiamo accontentarci perchè siamo in prossimità di un punto di minimo?

	\item (pag 154) in che senso "usare periodicamente" l'algoritmo del gradiente?

	\item (pag 159) esistono più tecniche di precondizionamento?

	\item (pag 159) perchè la matrice $H_{K}$ di precondizionamento della direzione del gradiente deve approssimare $Q^{-1}$?

  \item (pag 170) che vuol dire che la direzione di Newton è invariante rispetto alla scala?

\end{enumerate}

In riferimento ad AMPL:

\begin{enumerate}

	\item il comando "data;" all'inizio del file .dat è ridondante? A cosa serve?

	\item (CPLEX) cosa sono i branch-and-bound nodes?

	\item (CPLEX) cosa significa "No basis"?

	\item (KNITRO) cosa significano i suffissi feaserror, opterror, numfcevals, numiters?

\end{enumerate}

\clearpage
