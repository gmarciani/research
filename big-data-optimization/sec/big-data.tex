\chapter{Big Data}
\label{chp:big-data}

Si parla di Big Data quando si ha un dataset talmente grande da richiedere
strumenti non convenzionali per estrapolare, gestire e processare informazioni
entro un tempo ragionevole. Non esiste una dimensione di riferimento, ma questa
cambia sempre, poichè le macchine sono sempre più veloci e i dataset sono
sempre più grandi. Secondo uno studio del 2001, l’analista Doug Laney aveva
definito il modello di crescita come tridimensionale (modello 3V): volume, velocità, varietà. Il modello è stato poi esteso con l'aggiunta di altre due dimensioni (modello 5V): veridicità e valore.

Altra dimensione importante è la complessità: maggiore la dimensione del dataset, maggiore la difficoltà di estrazione di valore informativo.

L'interesse per i Big Data è motivato da: aumento di sorgenti dati, maggiori capacità di archiviazione.

Questi volumi di dati superano di molto la capacità di analisi dei metodi tradizionali.

\section{Data Mining}
\label{sec:data-mining}
Il data mining è la disciplina di estrazione di informazione implicita in dati strutturati, e di inferenza di pattern dai dati in modo (semi)automatico.
Il concetto di informazione e di pattern rilevante dipende dal dominio applicativo in cui si analizzano i dati.
