\chapter{Metodo di Armijo}
\label{chp:methods.armijo}
Il metodo di Armijo è un metodo di line search inesatta che calcola il passo
$\alpha_{k}$ mediante \textit{backtracking} \cite{armijo1966minimization}.
Supposta l'esistenza di una direzione di discesa $d_{k}$ per ogni punto non
stazionario e dato un passo iniziale $\Delta_{k}>0$, il passo $\alpha_{k}$ viene
ridotto di un fattore $\delta\in(0,1)$, detto \textit{fattore di backtracking},
fino a soddisfare la \textit{condizione di Armijo}, espressa da:

\begin{equation}
\label{eqn:methods.armijo.condition}
f(x_{k}+\alpha_{k}d_{k})\leq f(x_{k})+\gamma\alpha_{k}\nabla f(x_{k})^{T}d_{k}
\end{equation}

dove $\gamma\in(0,1)$ è il \textit{fattore di riduzione} della funzione.

La \textit{sufficiente riduzione} della funzione è garantita dalla condizione di
Armijo. Il \textit{sufficiente spostamento} è garantito scegliendo il passo
iniziale abbastanza grande.

Esiste una versione dell'algoritmo in cui viene rilassata la monotonicità della
condizione di Armijo. Tale versione prende il nome di \textit{metodo di Armijo
non monotono} ed è stata proposta in \cite{grippo1986nonmonotone}.

L'algoritmo generale per il calcolo del passo $\alpha_{k}$ con il metodo di Armijo
è mostrato in \Cref{alg:methods.armijo}.

\begin{algorithm}
	\label{alg:methods.armijo}
	\caption{Armijo}
	$\Delta_{k}>0, \gamma\in(0,1),\delta\in(0,1)$ \\
	$\alpha=\Delta,j=0$ \\
	\textbf{while} \\
	$f(x_{k}+\alpha d_{k}) > f(x_{k})+\gamma\alpha\nabla f(x_{k})^{T}d_{k}$: \\
	$\alpha=\delta\alpha$ \\
	$j=j+1$ \\
	\textbf{end while} \\
	$\alpha_{k}=\alpha$
\end{algorithm}


\subsection{Terminazione e convergenza}
\label{sec:methods.armijo.termination.convergence}
Dimostriamo ora la terminazione e la convergenza dell'algoritmo di Armijo.

\begin{theorem}[Terminazione di Armijo]
	\label{thm:methods.armijo.termination}
	Sia $f:\Re^{n}\rightarrow\Re$ una funzione continuamente differenziabile su
	$\Re^{n}$. Sia $x_{k}\in\Re^{n}$ un punto tale che $\nabla f(x_{k})^{T}d_{k}<0$.
	Il metodo di Armijo determina, in un numero finito di passi, un valore
	$\alpha_k>0$ che soddisfa la condizione di Armijo e una delle seguenti:

	\begin{enumerate}
		\item $\alpha_{k}=\Delta_{k}$;

		\item $\alpha_{k}\leq\delta\Delta_{k}$ e tale che
		$f(x_{k}+\frac{\alpha_{k}}{\delta}d_{k})
		>
		f(x_{k})+\gamma\frac{\alpha_{k}}{\delta}\nabla f(x_{k})^{T}d_{k}$;
	\end{enumerate}

	\begin{proof}
		INSERT HERE THE THEOREM PROOF.
	\end{proof}
\end{theorem}

La convergenza è assicurata scegliendo il passo iniziale abbastanza grande.

\begin{theorem}[Convergenza di Armijo]
	\label{thm:methods.armijo.convergence}
	Sia $f:\Re^{n}\rightarrow\Re$ una funzione continuamente differenziabile su
	$\Re^{n}$. Sia l'insieme di livello $\mathcal{L}_{0}$ compatto. 	Sia $\nabla f(x_{k})d_{k}<0$ per ogni punto $x_{k}$ non stazionario. Se il passo iniziale $\Delta_{k}$ soddisfa la seguente:

	\begin{equation}
	\label{eqn:methods.armijo.convergence}
	\Delta_{k}
	\geq
	\frac{1}{\norm{d_{k}}}\sigma\Bigg(\frac{\abs{\nabla f(x_{k})^{T}d_{k}}}{\norm{d_{k}}}\Bigg)
\end{equation}

dove $\sigma$ è una funzione di forzamento, allora l'algoritmo di Armijo
determina in un numero finito di passi un valore $\alpha_{k}>0$ tale che la
successione dei punti $\{x_{k}:x_{k+1}=x_{k}+\alpha_{k}d_{k}\}$ soddisfa le
seguenti:

\begin{enumerate}
	\item $f(x_{k+1})<f(x_{k})$;

	\item $\lim_{k\rightarrow\infty}\frac{\nabla f(x_{k})^{T}d_{k}}{\norm{d_{k}}}=0$.
\end{enumerate}

\begin{proof}
	INSERT HERE THE THEOREM PROOF.
\end{proof}
\end{theorem}

La dimostrazione del \Cref{thm:methods.armijo.convergence} è molto importante perchè
la maggior parte delle dimostrazioni di convergenza degli algoritmi vi
assomiglia.


\subsection{I parametri di Armijo}
\label{sec:methods.armijo.parameters}
Nell'algoritmo di Armijo intervengono i seguenti parametri:

\begin{enumerate}
	\item passo iniziale $\Delta_{k}$: è il primo passo da considerare.
	Deve essere sufficientemente grande per garantire la convergenza.
	Teoricamente $\Delta_{k}>0$, ma alcuni algoritmi richiedono $\Delta_{k}=1$
	per garantire una sufficiente rapidità di convergenza.

	\item fattore di riduzione $\gamma$: è la pendenza della retta che definisce
	la condizione di sufficiente riduzione della funzione obiettivo.
	Teoricamente $\gamma\in(0,1)$; praticamente $\gamma\in(0,0.5)$ così da
	rendere l'algoritmo applicabile al caso quadratico e garantire una
	sufficiente rapidità di convergenza per alcuni algoritmi; tipicamente
	$\gamma\in(10^{-3},10^{-4})$.

	\item fattore di backtracking $\delta$: è il fattore di riduzione del passo.
	Teoricamente $\delta\in(0,1)$; tipicamente $\delta=0.5$.
	Può essere anche scelto in un opportuno range $[\delta_{-},\delta_{+}]$
	(vedi \Cref{sec:methods.armijo.backtracking.bounded}).
\end{enumerate}


\subsubsection{Backtracking vincolato}
\label{sec:methods.armijo.backtracking.bounded}
Il fattore di backtracking $\delta$ può essere scelto in un opportuno range
$[\delta_{-},\delta_{+}]$. Applicando interpolazione sull'approssimazione
quadratica o cubica della funzione obiettivo si può ridurre sensibilmente il
numero di iterazioni del'algoritmo.

Una formulazione alterativa dell'algoritmo di Armijo, che tenga conto del
nacktracking vincolato è la seguente:

\begin{algorithm}
	\label{alg:methods.armijo.backtracking.bounded}
	\caption{Armijo (backtracking vincolato)}
	$\Delta_{k}>0, \gamma\in(0,0.5),0<\delta_{-}<\delta_{+}<1$ \\
	$\alpha=\Delta_{k}, j=0$ \\
	\textbf{while} \\
	$f(x_{k}+\alpha d_{k}) > f(x_{k})+\gamma\alpha\nabla f(x_{k})^{T}d_{k}$: \\
	scelgo $\delta\in[\delta_{-},\delta_{+}]$ \\
	$\alpha=\delta\alpha$ \\
	$j=j+1$ \\
	\textbf{end while} \\
	$\alpha_{k}=\alpha$
\end{algorithm}


\subsubsection{Armijo nel caso quadratico}
\label{sec:methods.armijo.quadratic}

\begin{theorem}[Armijo, caso quadratico]
	\label{thm:methods.armijo.quadratic}
	Sia $f(x)=\frac{1}{2}x^{T}Qx+c^{T}x$ con $Q$ simmetrica definita positiva.
	Siano $x_{k}$ e $d_{k}$ tali che $d_{k}$ è una direzione di discesa in $x_{k}$.
	Sia $\alpha_{k}^{*}$ il passo ottimo lungo $d_{k}$, calcolato come segue:
	\begin{equation}
	\alpha_{k}^{*}=-\frac{\nabla f(x_{k})^{T}d_{k}}{d_{k}^{T}Qd_{k}}
	\end{equation}
	Allora $\alpha_{k}^{*}$ soddisfa la condizione di Armijo se e solo se
	$\gamma\in(0,0.5]$.

	\begin{proof}
		INSERT HERE THE THEOREM PROOF.
	\end{proof}
\end{theorem}
