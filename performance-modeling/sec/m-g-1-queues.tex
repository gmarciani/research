\section{M/G/1 Queues}
\label{sec:M-G-1-queues}

A $M/G/1$ is a queue where 
(i) the arrival process is Poissonian with rate $\lambda$,
(ii) the service process is generically distributed,
(iii) there is one servers,
(iv) the buffer has infinite capacity,
(v) the scheduling policy is FCFS.

%\begin{figure}[tp]
%\label{fig:M-G-1-queue}	
%	\centering
%	\includegraphics{fig/M-G-1-Queue}
%	\caption{M/G/1 queue}
%\end{figure}

In general, we have that

\begin{equation}
\label{sec:M-G-1-Mean-Queue-Excess-Time}
\expected{T_{Q}}=\frac{\varrho}{1-\varrho}\expected{S_{e}}
\end{equation}

where $\expected{T_{e}}$ is the \textit{excess of service time}, that is the remaining service time of the job in service, given that there is some job in service.



\subsection{P-K Formula}
\label{sec:PK-Formula}
The Pollaczek-Khinchin (P-K Formula) \cite{pollaczek1930aufgabe,khinchin1967mathematical} is written in several equivalent forms:

\begin{equation}
\expected{T_{Q}}=\frac{\varrho}{1-\varrho}\cdot\frac{\expected{S^{2}}}{2\expected{S}}
\end{equation}

\begin{equation}
\expected{T_{Q}}=\frac{\varrho}{1-\varrho}\cdot\frac{\expected{S}}{2}\cdot(C_{S}^{2}+1)
\end{equation}

\begin{equation}
\expected{T_{Q}}=\frac{\lambda\expected{S^{2}}}{2(1-\varrho)}
\end{equation}