\subsection{Scheduling: Preemptive, Non-Size-Based}
\label{sec:Scheduling-Preemptive-Non-Size-Based}

The most common preemptive non-size-based scheduling policies are

\begin{description}
	
	\item [Processor-Sharing (PS)] the server is time-shared among all jobs.
	
	\item [Preemptive Last-Come First-Served (PLCFS)] a new arrival preempts the job in service; when the arrival is completed, the preempted job resumes.
	
	\item [Generalized Foreground-Background (FB)\footnote{also known as \textit{Least-Attained-Service (LAS)}.}] the job with the lowest age gets served; jobs with the same lowest age time-share service.
	
\end{description}




\subsection{Scheduling: PS}
\label{sec:Scheduling-PS}
PS is insensitive to job size variability, because when a job arrives it immediately time-shares with all other jobs.
$M/G/1/PS$ is better in expectation than $M/G/1/FCFS$ when the squared coefficient of job size variation gets high (namely, $C_{G}^{2}>1$).
PS is effective in CPU scheduling: it lets short jobs get out quickly, has slowdown lower than FCFS, and has high throughput in a  multi-resources context.

In $M/G/1/PS$, the following holds:

\begin{equation}
\label{eqn:PS-Mean-Response-Time}
\expected{T(x)}^{\mathit{M/G/1/PS}}=\frac{x}{1-\varrho}
\end{equation}

Hence $M/G/1/PS$ is said to be fair, because every job has the same slowdown, namely:

\begin{equation}
\label{eqn:PS-Mean-Slowdown}
\expected{Slowdown}^{\mathit{PS}}=\frac{1}{1-\varrho}
\end{equation}

We notice that $\variance{T}^{\mathit{PS}}$ cannot be expressed in closed form.

Recall that $M/G/1/PS$ and $M/G/1/FCFS$ have the same mean system jobs

\begin{equation*}
\expected{N}=\frac{\varrho}{a-\varrho}
\end{equation*}

There are many variants of PS, see \cite{aalto2007beyond}.




\subsection{Scheduling: PLCFS}
\label{sec:Scheduling-PLCFS}

In $M/G/1/PLCFS$, the following holds:

\begin{equation}
\label{eqn:PLCFS-Mean-Response-Time}
\expected{T(x)}^{\mathit{M/G/1/PLCFS}}=\frac{x}{1-\varrho}
\end{equation}

Hence $M/G/1/PLCFS$ is said to be fair, because every job has the same slowdown, namely:

\begin{equation}
\label{eqn:PLCFS-Mean-Slowdown}
\expected{Slowdown}^{\mathit{PLCFS}}=\frac{1}{1-\varrho}
\end{equation}

PLCFS has the same mean performances as PS, but with many fewer preemptions\footnote{only 2 preemptions per job.}.




\subsection{Scheduling: FB}
\label{sec:Scheduling-FB}

Recall that if job size distribution has decreasing failure rate, then the greater the job age, the greater its expected remaining service time.
FB gives priority to jobs with low age, this giving preference to the expected shortest jobs.

The performance improvement of FB against PS depends on job size distribution: if the distribution is d.f.r. then it outperforms.

If the job size distribution is DFR, then younger jobs have lower remaining service times, hence $\expected{T}^{\mathit{FB}}<\expected{T}^{\mathit{PS}}$.

If the job size distribution is IFR, then younger jobs have higher remaining service times, hence $\expected{T}^{\mathit{FB}}>\expected{T}^{\mathit{PS}}$.

If the job size distribution is CFR, then the remaining service times are independent of jobs ages, hence $\expected{T}^{\mathit{FB}}=\expected{T}^{\mathit{PS}}<$. 
If the job size distribution is CFR, then $\expected{Slowdown}^{\mathit{FB}}<\expected{Slowdown}^{\mathit{PS}}<$, because, even if job ages and expected remaining service times are independent, they are not of original jobs size. Hence, FB gives slight preference to small sized jobs.

FB needs the job size distribution to be DFR to perform well. Recall that higher DFR is coupled with higher variability. Thus, FB does not work well under low variability.

\begin{equation}
\label{eqn:Scheduling-FB-Response-Time}
\expected{T(x)}^{\mathit{FB}}=
\frac{x(1-\varrho_{\overline{x}})+\frac{1}{2}\lambda\expected{S_{\overline{x}}^{2}}}
{\Big(1-\varrho_{\overline{x}}\Big)^{2}}
\end{equation}

where $\varrho_{\overline{x}}=\lambda\expected{S_{\overline{x}}}$ is the utilization under the transformed distribution.