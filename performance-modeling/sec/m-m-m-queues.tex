\section{M/M/m Queues}
\label{sec:M-M-m-Queues}

A $M/M/m$ is a queue where 
(i) the arrival process is Poissonian with rate $\lambda$,
(ii) the service process is Exponential with rate $\mu$,
(iii) there are $m$ servers,
(iv) the buffer has infinite capacity,
(v) the scheduling policy is FCFS.

%\begin{figure}[tp]
%\label{fig:M-M-m-Queue}	
%	\centering
%	\includegraphics{fig/M-M-m-Queue}
%	\caption{An M/M/m queue and its corresponding CTMC.}
%\end{figure}

The key question in these types of systems is determining the \textit{Queue Probability} $P_{Q}$, that is the probability that an arriving job is enqueued.

\begin{definition}[Utilization]
\label{def:M-M-m-Utilization}
	For any $M/M/m$, the system utilization is
	
	\begin{equation}
	\label{eqn:M-M-m-Utilization}
	\varrho = \frac{\lambda}{m \mu}
	\end{equation}
\end{definition}

\begin{definition}[Resource Requirement]
\label{def:M-M-m-Resource-Requirement}
	For any $M/M/m$, the resource requirement is
	
	\begin{equation}
	\label{eqn:M-M-m-Resource-Requirement}	
	R = \frac{\lambda}{\mu}
	\end{equation}
\end{definition}

The resource requirement $R$ is 
(i) the minimum number of servers needed to keep the system stable,
(ii) the expected number of busy servers, and 
(iii) the expected number of jobs in service.
	
\begin{theorem}[State Probability]
\label{thm:M-M-m-Probability-State}	
	For any $M/M/m$, the state probability is
	
	\begin{equation}
	\label{eqn:M-M-m-Probability-State}
	\pi_{i} = \frac{m^{q} \varrho^{i}}{q!} \cdot \pi_{0}
	\end{equation}
	
	with
	
	\begin{equation}
	\label{eqn:M-M-m-Probability-State-Zero}	
	\pi_{0} = \Big[ \sum_{i=0}^{m-1} \frac{(m \varrho)^{i}}{i!} + \frac{(m \varrho)^{m}}{m! (1- \varrho)} \Big]^{-1}
	\end{equation}
	
	where
	
	$q=\min\{i,m\}$ is the number of busy servers.
	
	\begin{proof}
		For a formal demonstration, see \cite{harchol2013performance} on page 258-259.
	\end{proof}
\end{theorem}

\begin{theorem}[Queue Probability]
	\label{thm:M-M-m-Probability-Queue}	
	For any $M/M/m$, the queue probability is
	
	\begin{equation}
	\label{eqn:M-M-m-Probability-Queue}
	P_{Q} = \frac{(m \varrho)^{m}}{m!(1-\varrho)} \cdot \pi_{0}
	\end{equation}
	
	\begin{proof}
		For a formal demonstration, see \cite{harchol2013performance} on page 260.
	\end{proof}
\end{theorem}

\Cref{thm:M-M-m-Probability-Queue} is called \textit{Erlang-C Formula}.

\begin{theorem}[Mean Queue Jobs]
	\label{thm:M-M-m-Mean-Queue-Jobs}	
	For any $M/M/m$, the expected number of jobs in queue is
	
	\begin{equation}
	\label{eqn:M-M-m-Mean-Queue-Jobs}
	\expected{N_{Q}} = \frac{\varrho}{(1-\varrho)} \cdot P_{Q}
	\end{equation}
	
	\begin{proof}
		For a formal demonstration, see \cite{harchol2013performance} on page 262.
	\end{proof}
\end{theorem}

The remaining metrics ($\expected{N},\expected{T},\expected{T_{Q}}$) could be determined by applying the Little's Law, the basic definitions 
$\expected{N}=\expected{N_{Q}}+\expected{N_{S}}$, 
$\expected{N_{S}}=R$, 
$\expected{T}=\expected{T_{Q}}+\expected{T_{S}}$, and
$\expected{T_{S}}=\frac{1}{\mu}$.

In particular we obtain:

\begin{equation}
\label{eqn:M-M-m-Delay}
\expected{T_{Q}} = \frac{1}{\lambda} P_{Q} \frac{\varrho}{(1-\varrho)}
\end{equation}

\begin{equation}
\label{eqn:M-M-m-Reponse-Time}
\expected{T} = \frac{1}{\lambda} P_{Q} \frac{\varrho}{(1-\varrho)} + \frac{1}{\mu}
\end{equation}

\begin{equation}
\label{eqn:M-M-m-System-Jobs}
\expected{N} = P_{Q} \frac{\varrho}{(1-\varrho)} + m\varrho
\end{equation}




\subsection{Comparison between M/M/m and M/M/1}
\label{sec:Comparison-M-M-m-And-M-M-1}

Let us consider 
(i) a $M/M/m$ with average arrival rate $\lambda$ and average service rate $\mu$, and
(ii) a $M/M/1$ with average arrival rate $\lambda$ and average service rate $m \mu$.

Under light load, in (i) jobs are served by few servers with rate $\mu$; whereas in (ii) jobs are served with rate $m \mu$.
Under high load, in (i) the state is always greater than $m$ and it behaves just like (ii).

Hence, if $\lambda^{M/M/1} = \lambda^{M/M/m}$ and $\mu^{M/M/1} = m \mu^{M/M/m}$, then $M/M/1$ is always at least $m$ time faster than $M/M/m$.




\subsection{Comparison between M/M/m/m and M/M/m}
\label{sec:Comparison-M-M-m-m-And-M-M-m}

It is interesting to compare the block probability of a $M/M/m/m$ with the queue probability of an equivalent $M/M/m$.
The following theorem compares $P_{block}^{M/M/m/m}$ with $P_{Q}^{M/M/m}$.

\begin{theorem}[Relation $P_{block}^{M/M/m/m}$ with $P_{Q}^{M/M/m}$]
\label{thm:Relation-Probability-Block-Probability-Queue}	
	For any $M/M/m/m$ with block probability $P_{block}$ and any $M/M/m$ with queue probability $P_{Q}$, we have that
	 
	\begin{equation}
	\label{eqn:Relation-Probability-Block-Probability-Queue}
	P_{block} = \frac{(1-\varrho)P_{Q}}{1-\varrho P_{Q}}
	\end{equation}
	
	\begin{proof}
		For a formal demonstration, see \cite{harchol2013performance} on page 261.
	\end{proof}
\end{theorem}