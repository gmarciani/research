\section{Formulary of Probability}
\label{sec:Formulary-Probability}




\subsection{Foundations}

\begin{description}
	
	\item [Failure Rate]
		\begin{equation}
		r(x) = \frac{f(x)}{\overline{F}(x)}
		\end{equation}
	
	\item [Coefficient of Variation]
		\begin{equation}
		C_{X}^{2} = \frac{\variance{X}}{\expected{X^{2}}}
		\end{equation}
	
\end{description}




\subsection{Exponential Distribution}

\begin{description}
	
	\item [Exponential p.d.f]	
		\begin{equation}
		f(x) = \left\{\begin{matrix}
		\lambda e^{-\lambda x} & x \geq 0\\ 
		0 & x \leq 0
		\end{matrix}\right.
		\end{equation}
	
	\item [Exponential c.d.f]	
		\begin{equation}
		\begin{aligned}
		F(x) =  
		\left\{\begin{matrix}
		1 - e^{-\lambda x} & x \geq 0\\ 
		0 & x \leq 0
		\end{matrix}\right.
		\end{aligned}
		\end{equation}
	
	\item [Exponential Mean]	
		\begin{equation}
		\expected{X} = \frac{1}{\lambda}
		\end{equation}	
	
	\item [Exponential Variance]	
		\begin{equation}
		\variance{X} = \frac{1}{\lambda^{2}}
		\end{equation}	
	
	\item [Exponential Precedency]	
		\begin{equation}
		\probability{X_{1} < X_{2}} = \frac{\lambda_{1}}{\lambda_{1} + \lambda_{2}}
		\end{equation}
	
	\item [Exponential Minimum]	
		\begin{equation}
		\min(X_{1},X_{2}) \sim Exp(\lambda_{1} + \lambda_{2})
		\end{equation}

\end{description}




\subsection{Poisson Distribution}

\begin{description}
	
	\item [Poisson p.m.f]	
		\begin{equation}
		f(k,t) = \frac{(\lambda t)^{k} e^{-\lambda t}}{k!}
		\end{equation}
	
	\item [Poisson c.d.f]	
		\begin{equation}
		F(k,t) = \sum_{x=0}^{t} f(k,x)
		\end{equation}
	
	\item [Poisson Mean]	
		\begin{equation}
		\expected{N(t)} = \lambda t
		\end{equation}
	
	\item [Poisson Variance]	
		\begin{equation}
		\variance{N(t)} = \lambda t
		\end{equation}
	
	\item [Poisson Inter-arrivals]	
		\begin{equation}
		\probability{N(s+t)-N(s) = n} \sim Poisson(\lambda t)
		\end{equation}	
	
	\item [Poisson Merging]
		Given two independent Poisson process with rate $\lambda_{1}$ and $\lambda_{2}$, the merged process is a Poisson process with rate $(\lambda_{1} + \lambda_{2})$.
	
	\item [Poisson Splitting]
		Given a Poisson process with rate $\lambda$, whose events are partitioned in class-A with probability $p$ and class-B with probability $(1-p)$, the class-A process is a Poisson process with rate $p \lambda$ and the class-B process is a Poisson process with rate $(1-p) \lambda$, and these processes are independent.
	
	\item [Poisson Uniformity]
		If $k$ events of a Poisson process occur by time $t$, then the $k$ events are distributed independently and uniformly in $[0,t]$. 
		
\end{description}




\subsection{Pareto Distribution}

\begin{description}
	
	\item [Pareto distribution]
		$X$ is Pareto distributed with order $\alpha$ (that is, $X \sim Pareto(\alpha)$) if
		\begin{equation}
		\overline{F}(x) = \probability{X > x} = x^{-\alpha} \quad \forall x \geq 1, 0 < \alpha < 2
		\end{equation}
	
	\item [Bounded Pareto distribution]
		$X$ is Bounded Pareto distributed with order $\alpha$ and bounds $k,p$ (that is, $X \sim BP(k,p,\alpha)$) if
		\begin{equation}
		f(x) = \probability{X \leq x} = \alpha x^{-\alpha -1} \cdot \frac{k^{\alpha}}{1-\Big(\frac{k}{p})^{\alpha}} \quad \forall k \leq x \geq p, 0 < \alpha < 2
		\end{equation}
	
\end{description}