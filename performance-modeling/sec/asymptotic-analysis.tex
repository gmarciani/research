\section{Asymptotic Analysis for Closed Systems}
\label{sec:Asymptotic-Analysis-Closed-Systems}

\begin{theorem}[Asymptotic Bounds for Closed Systems]
	\label{thm:Asymptotic-Bounds-Closed-Systems}
	For any closed system with load $N$, we have that
	
	\begin{equation}
	\label{eqn:Asymptotic-Bounds-Closed-Systems}
	\begin{split}
	X & \leq \min \Big\{ \frac{N}{D + \expected{Z}} , \frac{1}{D_{max}} \Big\} \\
	\expected{R} & \geq \max\{D , N \cdot D_{max} - \expected{Z} \}
	\end{split}
	\end{equation}
	
	where the first term in each clause is an asymptote for small $N$, and the second one for large $N$.
	
	\begin{proof}
		See \cite{harchol2013performance} for a formal demonstration.
	\end{proof}
\end{theorem}

It is possible to get even tighter bounds by using the \textit{Balanced Bounds technique}, described in \cite{lazowska1984quantitative}.

The lower and the upper bound intercept in 
\begin{equation}
\label{eqn:Asymptotic-Bounds-Closed-Systems-Interception}
N^{*} = \frac{D + \expected{Z}}{D_{max}}
\end{equation}
That is the point beyond which there must be some queueing in the system, because $\expected{R} > D$.

If $N > N^{*}$, performance can be improved ($X$ increased and $\expected{R}$ decreased) only decreasing $D_{max}$.

If $N >> N^{*}$, decreasing some $D_{i} < D_{max}$ have no effect.
If $N << N^{*}$, decreasing some $D_{i} < D_{max}$ have little effect.

In a batch system ($Z=0$), $N^{*}$ decreases, meaning that the domination of $D_{max}$ occurs with smaller load.

The bottleneck server is the server with demand $D_{max}$. It is the key limiting factor to performance improvement. Thus, the first step in improving performance is to identify the bottleneck server.

The asymptotic bounds hold for open systems only if $\lambda . X \rightarrow \frac{1}{D_{max}}$.