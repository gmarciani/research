\section{M/M/1 Queues}
\label{sec:M-M-1-queues}

A $M/M/1$ is a queue where 
(i) the arrival process is Poissonian with rate $\lambda$,
(ii) the service process is Exponential with rate $\mu$,
(iii) there is one servers,
(iv) the buffer has infinite capacity,
(v) the scheduling policy is FCFS.

%\begin{figure}[tp]
%\label{fig:M-M-1-queue}	
%	\centering
%	\includegraphics{fig/M-M-1-Queue}
%	\caption{An M/M/1 queue and its corresponding CTMC.}
%\end{figure}
	
\begin{theorem}[State Probability]
\label{thm:M-M-1-probability-state}	
	For any $M/M/1$, the state probability is
	
	\begin{equation}
	\label{eqn:M-M-1-probability-state}
	\pi_{i} = \varrho^{i}(1-\varrho)
	\end{equation}
	
	\begin{proof}
		For a formal demonstration, see \cite{harchol2013performance} on page 258-259.
	\end{proof}
\end{theorem}

\begin{theorem}[Mean System Jobs]
\label{thm:M-M-1-System-Jobs}	
	For any $M/M/1$, the expected number of jobs is
	
	\begin{equation}
	\label{eqn:M-M-1-System-Jobs}
	\expected{N} = \frac{\varrho}{(1-\varrho)}
	\end{equation}
	
	\begin{proof}
		\begin{equation*}
		\expected{N} = \sum_{i=0}^{\infty} i \pi_{i} = \frac{\varrho}{(1-\varrho)}
		\end{equation*}
	\end{proof}
\end{theorem}

The remaining metrics ($\expected{N_{Q}},\expected{T},\expected{T_{Q}}$) could be determined by applying the Little's Law, the basic definitions 
$\expected{N}=\expected{N_{Q}}+\expected{N_{S}}$, 
$\expected{N_{S}}=\varrho$, 
$\expected{T}=\expected{T_{Q}}+\expected{T_{S}}$, and
$\expected{T_{S}}=\frac{1}{\mu}$.

In particular we obtain:

\begin{equation}
\label{eqn:M-M-1-Queue-Jobs}
\expected{N_{Q}} = \frac{\varrho^{2}}{(1-\varrho)}
\end{equation}

\begin{equation}
\label{eqn:M-M-1-Delay}
\expected{T_{Q}} = \frac{1}{\mu} \cdot \frac{\varrho}{(1-\varrho)}
\end{equation}

\begin{equation}
\label{eqn:M-M-1-Response-Time}
\expected{T} = \frac{1}{\mu - \lambda}
\end{equation}