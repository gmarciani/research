\section{Scheduling: Non-Preemptive, Non-Size-Based}
\label{sec:Scheduling-Non-Preemptive-Non-Size-Based}

The most common non-preemptive non-size-based scheduling policies are

\begin{description}

	\item [Random Order (RO)] the server chooses the job in queue at random.
	
	\item [First-Come First-Served (FCFS)] the server chooses the job at the head of the queue.
	
	\item [Last-Come First-Served (LCFS)] the server chooses the job at the tail of the queue.

\end{description}

\begin{theorem}
\label{thm:Scheduling--Non-Preemptive-Non-Size-Based}
All non-preemptive service orders that do not make use of job sizes have the same $\expected{N}$ and the same $\expected{T}$.
\end{theorem}

In particular, we have that

\begin{equation}
\expected{N^{\mathit{FCFS}}}=\expected{N^{\mathit{LCFS}}}=\expected{N^{\mathit{RO}}}
\end{equation}

\begin{equation}
\expected{T^{\mathit{FCFS}}}=\expected{T^{\mathit{LCFS}}}=\expected{T^{\mathit{RO}}}
\end{equation}

\begin{equation}
\variance{T^{\mathit{FCFS}}}<\variance{T^{\mathit{RO}}}=\variance{T^{\mathit{LCFS}}}
\end{equation}

The latter holds because LCFS can produce high response time, since the first come job has to wait for the queue to be empty.

Thus, all non-preemptive non-size-based scheduling policies for $M/G/1$ have the same $\expected{T}$ as $M/M/1$, namely

\begin{equation}
\label{eqn:Non-Preemptive-Non-Size-Based-Time}
\expected{T}=\expected{S}+\frac{\lambda\expected{S^{2}}}{2(1-\varrho)}
\end{equation} 

The problem is that under high-variability job sizes the mean response time gets very high.
In particular, the mean slowdown is high for small sized jobs.