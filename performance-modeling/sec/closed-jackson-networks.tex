\section{Closed Jackson Networks}
\label{sec:Closed-Jackson-Networks}

We examine only the Batch Jackson Networks. 
See \cite{harchol2013performance} to examine Interactive Jackson Networks.

\begin{definition}[Batch Jackson Network]
	\label{def:Batch-Jackson-Network}
	A Batch Jackson Network is a closed network of queues such that:
	
	\begin{itemize}
		\item there are $m$ $M/M/1$ queues;
		\item the multiprogramming level is $N$;
		\item the $i$-th server service is Exponentially distributed with rate $\mu_{i}$;
		\item each server may receive only internal arrivals;
		\item when the $i$-th server completes a job, it is routed to the $j$-th server (internal arrival) with probability $P_{i,j}$.
	\end{itemize}
\end{definition}

%\begin{figure}[tp]
%\label{fig:Batch-Jackson-Network}	
%	\centering
%	\includegraphics{fig/Batch-Jackson-Network}
%	\caption{A Batch Jackson Network and its corresponding CTMC.}
%\end{figure}

The \textit{response time} is defined as the time from when the job arrives to the system until it leaves it, including possible multiple visitation to the same server.

The \textit{total arrival rate} $\lambda_{i}$ to the $i$-th server is also the total departure rate from the same server.

In such networks it not possible to determine a unique solution for all the $\lambda_{i}$'s, because we face with a $k-1$-order system of equations. So, we have to choose a $\lambda_{i}$ at random. Typically we fix $\lambda_{1}$ at random and then resolve the others \footnote{Since the $\lambda_{i}$'s rely on a made-up value, the $\varrho_{i}$'s do not represent real servers utilizations}.

\begin{theorem}[Batch Jackson Network Product Form]
\label{thm:Batch-Jackson-Network-Product-Form}	
	A Closed Jackson Network with $m$ servers has the following product form
	
	\begin{equation}
	\label{eqn:Batch-Jackson-Network-Product-Form}
	\pi_{n_{1},...,n_{m}} = C \cdot \prod_{i=1}^{m} \varrho_{i}^{n_{i}} (1-\varrho_{i})
	\end{equation}
	
	where $C$ is the \textit{Normalizing Constant} determined as the solution of 
	
	$
	\sum_{(n_{1},...,n_{m}) \\ \in States} \pi_{n_{1},...,n_{m}} = 1
	$
	
	\begin{proof}
		The demonstration makes use of the \textit{Mean Value Analysis}.
		For a formal demonstration, see \cite{harchol2013performance}.
	\end{proof}
\end{theorem}

The complexity of determining state probabilities with \Cref{eqn:Batch-Jackson-Network-Product-Form} growths exponentially with $N$ and $m$
\footnote{The total number of states in a Closed Jackson Network is equal to the number of ways of dividing $N$ jobs among $m$ servers, that is
	\begin{equation*}
	\label{eqn:Batch-Jackson-Networks-Number-States}
	Number \! of \! States = \binom{N+m-1}{m-1}
	\end{equation*}
}.

When $N$ and/or $m$ get high, it is better to apply the \textit{Mean Value Analysis}.