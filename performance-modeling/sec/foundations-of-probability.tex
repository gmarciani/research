\section{Foundations of Probability}
\label{sec:probability-foundations}

The memoryless property states that the probability to be in some state is independent of all the previous states.

\begin{definition}[Memoryless Property]
	\label{def:memoryless-property}
	A random variable $X$ is memoryless if
	\begin{equation}
	\label{eqn:memoryless-property}
		\probability{X > s+t | X > s} = \probability{X > t} \quad \forall s,t \geq 0
	\end{equation}
\end{definition}

Distributions for which $\probability{X > s+t | X > s}$ increases with $s$ are said to have \textit{increasing failure rate}, whereas if it decreases they are said to have \textit{decreasing failure rate}.

\begin{definition}[Independent Increment]
	A sequence of events has independent increments if the number of events that occur in disjoint time intervals are independent.
\end{definition}

\begin{definition}[Stationary Increment]
	A sequence of events has stationary increments if the number of events during a time interval depends only on the length of the interval, and not on its starting point.
\end{definition}


\subsection{TO BE ADDED}

Una variabile aleatoria discreta X è un’entità che può assumere un numero discreto (finito o
infinito) di valori xi. Ogni valore xi ha probabilità di occorrenza P(X=xi)=pi, dove Σ i pi = 1.
Una variabile aleatoria continua Y è un’entità che può assumere valori y in un sottoinsieme S della
retta reale composto da uno o più intervalli. Ogni intervallo infinitesimo di ampiezza dy ha
probabilità P(y<Y≤y+dy)=p(y)dy che Y assuma un valore all'interno di esso. La funzione p(y) è
detta funzione di densità ed è tale che ∫ p(y)dy = 1 .
S
Un processo stocastico è una variabile aleatoria i cui valori e le relative probabilità sono funzioni
del tempo. L'evoluzione temporale di un processo stocastico può avvenire in modo continuo o
discreto.