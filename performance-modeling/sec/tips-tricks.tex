\section{Tips and Tricks}
\label{sec:tips-and-tricks}

Always take into account the following (often counterintuitive) know hows:
\begin{itemize}
	\item When job variability is high, we prefer many slow server instead of one fast server because we do not want short jobs getting stuck behind long ones.
	
	\item When load is low, we prefer one fast server because we do not want unutilized servers.
	
	\item If jobs are preemptible, we could always use one fast server to simulate many slow servers.
	
	\item In a closed system, the service rate does not influence neither mean time nor throughput, because arrival times depend on completion times.
	
	\item In an open system, the service rate does influence both the mean time and throughput, because arrival times are independent of service completions.
	
	\item Non-preemptive scheduling policies have all the same mean response time as FCFS.
	
	\item In case of finite buffer, $X=\varrho \cdot \mu$ but we need stochastic analysis to determine $\varrho$ because some arrivals can be dropped.
	
	\item For large loads, improvements can be achieved decreasing the demand to bottleneck server.
	
	\item high value of utilization does not imply hard delays, provided that there are sufficient servers.
\end{itemize}