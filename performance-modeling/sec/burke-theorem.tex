\section{Burke's Theorem}
\label{sec:Burke-Theorem}

\begin{theorem}[Burke]
\label{thm:Burke}
	For any $M/M/m$ with arrival rate $\lambda$, we have that:
	
	\begin{enumerate}
		\item The departure process is $Poisson(\lambda)$
		\footnote{that is, inter-departure times are Exponentially distributed with rate $\lambda$.};
		
		\item The number of jobs in the system is independent of the sequence of previous departures
		\footnote{that is, independent of times and patterns}.
	\end{enumerate}
	\begin{proof}
		For a formal demonstration, see \cite{harchol2013performance}.
	\end{proof}
\end{theorem}

The Burke's Theorem allows us to instantly analyze a large class of queueing networks.
Let us start with its application on (i) tandem systems and (ii) acyclic networks with probabilistic routing.




\subsection{Burke Applications: Tandem Systems}
\label{sec:Burke-Application-Tandem-Systems}

By the Burke's Theorem, for any tandem system made by m $M/M/1$ nodes arrival rate $\lambda$, we have that:

\begin{itemize}
	\item the arrival stream to each server is Poisson distributed with rate $\lambda$;
	
	\item the number of jobs in each server is independent of the number of jobs in every other server;
	
	\item the order of the sequence of servers are uninfluential for system performance;
	
	\item the state probability is
	
	\begin{equation}
	\label{eqn:Burke-Application-Tandem-Systems-Probability-State}
	\pi_{n_{1},...,n_{m}} = \prod_{i=1}^{m} \varrho_{i}^{n_{i}} (1 - \varrho_{i})
	\end{equation}
	
	where $\varrho_{i}$ is the utilization of the $i$-th server.
	
	\item the probability to have $n_{i}$ jobs in the $i$-th server is
	
	\begin{equation}
	\label{eqn:Burke-Application-Tandem-Systems-Probability-Jobs-Server}
	\probability{N_{i}=n_{i}} = \varrho_{i}^{n_{i}} (1 - \varrho_{i})
	\end{equation}
	
	where $\varrho_{i}=\frac{\lambda}{\mu_{i}}$ is the utilization of the $i$-th server.
	
	\item the mean number of jobs in the $i$-th server is
	
	\begin{equation}
	\label{eqn:Burke-Application-Tandem-Systems-Expected-Jobs-Server}
	\expected{N_{i}} = \frac{\varrho_{i}}{1-\varrho_{i}}
	\end{equation}
\end{itemize}

%\begin{figure}[tp]
%	\label{fig:tandem-systems}	
%	\centering
%	\includegraphics{fig/Tandem-System}
%	\caption{A Tandem system}
%\end{figure}




\subsection{Burke Application: Acyclic Networks with Probabilistic Routing}
\label{sec:Burke-Application-Acyclic-Networks-Probabilistic-Routing}

By the Burke's Theorem, for any acyclic system with probabilistic routing made by $m$ $M/M/1$ nodes with arrival rate $\lambda$, everything stated in \Cref{sec:Burke-Application-Tandem-Systems} holds, but the arrival stream to each server is (splitted/merged) Poisson distributed with rate proportional to $\lambda$.

%\begin{figure}[tp]
%\label{fig:Acyclic-Networks-Probabilistic-Routing}	
%	\centering
%	\includegraphics{fig/Acyclic-Network-Probabilistic-Routing}
%	\caption{An Acyclic Networks with Probabilistic Routing}
%\end{figure}