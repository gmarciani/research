\section{Scheduling for M/G/1}
\label{sec:Scheduling-M-G-1}

The scheduling policy can hugely influence the mean response time and other metrics.
Scheduling policies can be categorized based on whether the policy is preemptive or not, and whether it assumes knowledge of job size or not.

A \textit{work-conserving} scheduling policy is one that always performs work on some job when there is a job in the system, and does not create new work.
Given an arrival sequence and a time, all work-conserving policies have the same work left in system and server utilization.This does not imply that all work-conserving policies have the same mean response time.

When evaluating a scheduling policy, we take into account the following metrics:

\begin{description}
	
	\item [Response Time ($T$)]
	We denote with $T(x)$ the response time for job size $x$.
	
	\item [Queueing Time ($T_{Q}$)]
	
	\item [System Jobs ($N$)]
	
	\item [Queueing Jobs ($N_{Q}$)]
	
	\item [Slowdown ($Slowdown$)]
	\begin{equation}
	\label{eqn:Slowdown}
	Slowdown=\frac{T}{S}
	\end{equation}
	We denote with $Slowdown(x)$ the slowdown for job size $x$.
	
	\item [Tail Behavior ($Tail-Behavior$)]
	\begin{equation}
	\label{eqn:Tail-Behavior}
	Tail-Behavior=\probability{T>x}
	\end{equation}
	
	\item [Starvation]
	We say that policy $\mathit{P}$ produces \textit{starvation} if for some $x$
	\begin{equation}
	\label{eqn:Starvation}
	\expected{Slowdown^{\mathit{P}}(x)}>\expected{Slowdown^{\mathit{PS}}(x)}
	\end{equation}
	where $\mathit{PS}$ is the Processor-Sharing scheduling policy.
	
	\item [Fairness]
	We say that policy $\mathit{P}$ is fair if for all $x$
	\begin{equation}
	\label{eqn:Fairness}
	\expected{Slowdown^{\mathit{P}}(x)}<\expected{Slowdown^{\mathit{PS}}(x)}
	\end{equation}
	where $\mathit{PS}$ is the Processor-Sharing scheduling policy.
	Alternatively, we say that policy $\mathit{P}$ is fair if $Slowdown^{\mathit{P}}(x)$ is equal for all $x$.
	
\end{description}

More often we consider absolute mean metrics and mean metrics with respect to job sizes.
Sometimes however, variance in response time and variance in slowdown are more important than the respective means.

Few books analyze scheduling policies in stochastic environment: the best are \cite{conway2012theory,kleinrock1976queueing}.
The fairness metric we used is defined in \cite{bansal2001analysis}. The slowdown metric has received attention only recently in \cite{hyytia2012minimizing}.

In the following sections, we evaluate various scheduling policies for $M/G/1$. We base our evaluation on $\expected{T},\expected{T(x)},\expected{Slowdown(x)}$.

Notice that $\expected{Slowdown}\neq\frac{\expected{T}}{\expected{S}}$. We derive the Slowdown as follows:

\begin{equation*}
\expected{Slowdown(x)}=\expected{\frac{T}{S}|S=x}=\expected{\frac{T(x)}{x}}=\frac{\expected{T(x)}}{x}
\end{equation*}

and

\begin{equation*}
\expected{Slowdown}=\int_{x}\expected{Slowdown(x)}f_{S}(x)\partial x=\int_{x}\frac{\expected{T(x)}}{x}f_{S}(x)\partial x
\end{equation*}

The \textit{job age} is the total service it has received so far. If job size distribution has decreasing failure rate, then the greater the job age, the greater its expected remaining service time.